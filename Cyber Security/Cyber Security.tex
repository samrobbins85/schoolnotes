\documentclass{article}[18pt]
\usepackage[utf8]{inputenc}
\usepackage[margin=0.7in]{geometry}
\usepackage{parselines} 
\usepackage{amsmath}
\usepackage{titlesec}
\usepackage{pgfplots}
\usepackage{graphicx}
\usepackage[english]{babel}
\usepackage{fancyhdr}
\usepackage{gensymb}
\usepackage{amssymb}

\pgfplotsset{width=10cm,compat=1.9}

\titlespacing\section{0pt}{14pt plus 4pt minus 2pt}{0pt plus 2pt minus 2pt}
\newlength\tindent
\setlength{\tindent}{\parindent}
\setlength{\parindent}{0pt}
\renewcommand{\indent}{\hspace*{\tindent}}

\pagestyle{fancy}
\fancyhf{}
\rhead{Sam Robbins 13SE}
\lhead{GCHQ Cyberfirst}
\rfoot{Page \thepage}

\usepackage{titlesec}

\setcounter{secnumdepth}{4}

\titleformat{\paragraph}
{\normalfont\normalsize\bfseries}{\theparagraph}{1em}{}
\titlespacing*{\paragraph}
{0pt}{3.25ex plus 1ex minus .2ex}{1.5ex plus .2ex}

\begin{document}
\begin{center}
\underline{\huge Cyber Security Notes}
\end{center}
\section{Security Essentials}
\begin{itemize}
\item Integrity
\item Availability 
\item Confidentiality 
\end{itemize}
\section{Introductory definitions}
\begin{itemize}
\item Information Assets - Data to be protected
\item Authentication - Verifying the identity of a user
\item Non-repudiation - Ensuring a user cannot deny something later or claim something is false
\item Malware - A contraction of malicious software
\item Ransomware - Malware that demands money to stop from doing something
\item Spyware - Malware that records the activity of the user
\item Botnets - Malware that records the activities of the user
\item Vulnerability - A point at which there is potential for a security breach
\item Threat - Some danger that can exploit a vulnerability
\item Countermeasure - An action taken to protect information from threats and vulnerabilities
\end{itemize}
\section{Passwords}
\subsection{Aims of a password}
\begin{itemize}
\item Memorable enough that the user can remember it without writing it down
\item Long and unique enough that no one else can guess it
\end{itemize}
As these two aims are a contradiction, password must be a compromise between the two
\subsection{Transfer of passwords}
Passwords transmitted and stored in plaintext are insecure
\subsection{Securing of passwords}
Passwords are often encrypted using \textbf{SSL}(secure socket layer)\\
\\
\textbf{Hashing}- Encrypting a password using one way encryption, any subsequent password is encrypted using the same method and compared to the stored hashed password.
\subsubsection{Salting}
\textbf{Salting} - Adding a value to the password before encryption.\\
Salting means that even if two people choose identical passwords, the stored password will be different.\\
Salting is only effective if:
\begin{itemize}
\item Salts are truly random
\item The salt is sufficiently long enough to avoid the attacker just adapting their dictionary to include all salted values
\end{itemize}
\subsection{Password managers}
Requirements for a password manager:
\begin{itemize}
\item The password manager should require a password to start it, preventing unwanted access
\item It should lock itself after a period of inactivity
\item The passwords should be encrypted
\end{itemize}
\section{Types of cyber attacks}
\subsection{Virus}
A virus is a self replicating program often intended to cause harm
\subsection{Worms}
Four stages of a worm attack:
\begin{itemize}
\item First stage - Worm probes other machines, looking for a vulnerability to exploit
\item Second stage - Penetrate the machine, exploiting the vulnerability
\item Third stage - The worm downloads itself onto the machine and stores itself there
\item Fourth stage - Probe other machines (back to stage 1)
\end{itemize}
\subsection{Trojans}
\textbf{Trojan} - A seemingly legitimate program that causes damage behind the scenes\\
Trojans are not self replicating
\subsection{Phishing}
\textbf{Phishing} - The process of luring people to disclose confidential information\\
Phishing relies on people trusting official looking messages
\subsection{Spam Messages}
\textbf{SMTP}(Simple Mail Transfer Protocol) defines a standard template of commands for different email programs.\\
This was created to a small number of users so did not include the ability to verify emails, meaning that phishing becomes possible.
\subsection{Spoofing}
Spoofing is where people pretend to be a person or device that they are not
\subsection{Botnets}
Botnets are used to coordinate the activity of many computers, these are often used for further cyber attacks.
\section{Antivirus software}
\textbf{Malware signature} - A distinctive pattern of data, either in memory or in a file\\
\\
\textbf{Heuristics} - The use of rules to identify viruses based on previous exposure to viruses. Heuristics may execute programs in a virtual machine, checking the requests and actions the malware makes to see if it poses a threat to the computer.\\
\\
\textbf{Sandbox} - A way for computers to run programs in a controlled environment. This constrains computing resources, allowing the program to not cause a threat to the computer.\\
\\
\textbf{Signed programs} - The sue of cryptography when companies issue copies of a program, so that the user can check it for authenticity.
\section{How the internet works}
The internet comprises of a hierarchy of individual networks that have been connected to each other.\\
\\
Key factors in the design of the internet:
\begin{itemize}
\item Should not have a central controlling computer. Every computer on the network has the same authority
\item The network should be able to deliver information between any two computers on the network, even if come of the machines in the network have failed. There should be a large number of alternative routes through the network
\end{itemize}
\subsection{Datagrams(packets)}
When a large amount of data is sent over the internet it is split into small, uniformly sized blocks called "datagrams", also called "packets"\\
\\
\textbf{Header} - Sender and recipient's address, unique number, data stamp and error correcting information\\
\textbf{Payload} - The actual information being delivered
\subsection{Wireless networks}
WiFi allows devices to be connected together wirelessly to form a LAN\\
WiFi refers to the wireless LAN standard from the Institute of Electrical and Electronic Engineers (IEEE) called the 802.11 family.\\
\\
\textbf{SSID} - The name of the network (Service Set IDentifier)\\
The SSID allows nodes on a wireless LAN to distinguish themselves from nodes on other wireless LANs in the same physical space.
\section{Network security challenges}
\textbf{Packet Sniffing} - The copying of datagrams without the recipient knowing
\subsection{Security risks of wireless networking}
A wireless network should ensure that an eavesdropper is not able to convert wireless signals into the original message. This ensures \textbf{confidentiality}.
\subsubsection{Man in the middle attacks}
A man in the middle attack is where malicious users interpose themselves between the sender and receiver to modify or destroy the messages being sent. This compromises the \textbf{integrity} of the data being transferred.
\subsubsection{Denial of service attacks}
An attacker could transmit lots or random data on the frequency used by the wireless network, congesting the network and so preventing others from sending data. This compromises the \textbf{availability} of the network.
\newpage
\subsection{How encryption helps prevent security issues in wireless networks}
Encryption helps ensure:
\begin{itemize}
\item \textbf{Confidentiality} - Encryption keys are needed to decrypt information, meaning attackers can't recover the information
\item \textbf{Integrity} - Encryption prevents messages from being modified without the receivers knowledge
\item \textbf{Authentication} - Encryption proves the identities of the sender and receiver
\end{itemize}
\subsection{Implementation of encryption in WiFi}
\subsubsection{WEP(Wired Equivalent Privacy)}
WEP has many serious problems as the encryption key can be computed in a few minutes. Many devices still support this to ensure compatibility but it should not be used.
\subsubsection{WPA2(WiFi Protected Access 2)}
This uses a more secure key to encrypt data that WEP, this is the default for WiFi networks. All WiFi devices must support it to be compliant with the 802.11 standard.
\section{The role of standards in the internet}
\subsection{Why standards are needed on the internet}
Standards are needed to ensure that all devices can communicate with each other.
\subsection{TCP/IP Protocols}
\subsubsection{TCP}
TCP ensures that data can be sent reliably over the internet. This works through software ports to keep data separate on the same computer.\\
The port decides how the data is handled when it reaches it's destination.\\
\\
Common TCP Ports:
\begin{itemize}
\item 20 and 21 - FTP - for sending and receiving files(20) and control(21)
\item 22 - SSH for secure logins
\item 25 - SMTP(Simple Mail transfer protocol) - Email
\item 80 - HTTP(HyperText Transfer Protocol) - Web Pages
\end{itemize}
\textbf{Traffic type} - The type of data and the port associated with it. For example SSH or HTTP.
\\
TCP also ensures all data sent from a computer is received by the destination. It does this by waiting for acknowledgements of receipt from the remote computer, and if the data was not received, sending the message again.\\
\\
Applications where latency is more important than accuracy, such as video conferencing, use \textbf{UDP} to send and receive data.
\subsubsection{IP}
\textbf{IP} - Internet Protocol\\
IP is only concerned with moving data, it doesn't check the data has arrived(handled by TCP).\\
\\
When IP receives data from TCP in the computer it wraps the TCP datagram in an IP datagram with senders and receivers address, along with other information.\\
\\
When IP receives data from the internet, it removes the IO datagram and passes it to TCP.
\section{Test}
This is some test text
\paragraph{Test}
How is this

\end{document}