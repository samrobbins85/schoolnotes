\documentclass{article}[18pt]
\usepackage[utf8]{inputenc}
\usepackage[margin=0.7in]{geometry}
\usepackage{parselines} 
\usepackage{amsmath}
\usepackage{titlesec}
\usepackage{pgfplots}
\usepackage{graphicx}
\usepackage[english]{babel}
\usepackage{fancyhdr}
\usepackage{gensymb}

\pgfplotsset{width=10cm,compat=1.9}

\titlespacing\section{0pt}{14pt plus 4pt minus 2pt}{0pt plus 2pt minus 2pt}
\newlength\tindent
\setlength{\tindent}{\parindent}
\setlength{\parindent}{0pt}
\renewcommand{\indent}{\hspace*{\tindent}}

\pagestyle{fancy}
\fancyhf{}
\rhead{Sam Robbins 13SE}
\lhead{A Level Maths - S2}
\rfoot{Page \thepage}


\begin{document}
\begin{center}
\underline{\huge Hypothesis testing}
\end{center}
\section{Tests of hypotheses}
\textbf{Statistical hypothesis} - An assertion or conjecture concerning a population.\\
\\
To test the validity of a statement a random sample is taken from the population and that data can them be used to provide evidence that either supports or does not support the hypothesis.\\
\\
\textbf{Null hypothesis} - $H_0$ - A hypothesis assumed to be true
\textbf{Alternative hypothesis} - $H_1$ - The situation if $H_0$ is false.

If the data leads to rejection of the null hypothesis the alternative hypothesis will be accepted.

The sample data is used to evaluate the \textbf{test statistic}, probabilities related to it can be calculated using the null hypothesis.

Critical values are the range of values that if the test statistic is found in the null hypothesis will be rejected.


 


\end{document}