\documentclass{article}[18pt]
\usepackage[utf8]{inputenc}
\usepackage[margin=0.7in]{geometry}
\usepackage{parselines} 
\usepackage{amsmath}
\usepackage{titlesec}
\usepackage{pgfplots}
\usepackage{graphicx}
\usepackage[english]{babel}
\usepackage{fancyhdr}
\usepackage{gensymb}
\usepackage{amssymb}

\pgfplotsset{width=10cm,compat=1.9}

\titlespacing\section{0pt}{14pt plus 4pt minus 2pt}{0pt plus 2pt minus 2pt}
\newlength\tindent
\setlength{\tindent}{\parindent}
\setlength{\parindent}{0pt}
\renewcommand{\indent}{\hspace*{\tindent}}

\pagestyle{fancy}
\fancyhf{}
\rhead{Sam Robbins 13SE}
\lhead{A Level Maths - S2}
\rfoot{Page \thepage}


\begin{document}
\begin{center}
\underline{\huge Hypothesis testing}
\end{center}
\section{Tests of hypotheses}
\textbf{Statistical hypothesis} - An assertion or conjecture concerning a population.\\
\\
To test the validity of a statement a random sample is taken from the population and that data can them be used to provide evidence that either supports or does not support the hypothesis.\\
\\
\textbf{Null hypothesis} - $H_0$ - A hypothesis assumed to be true
\textbf{Alternative hypothesis} - $H_1$ - The situation if $H_0$ is false.

If the data leads to rejection of the null hypothesis the alternative hypothesis will be accepted.

The sample data is used to evaluate the \textbf{test statistic}, probabilities related to it can be calculated using the null hypothesis.

If the test statistic is found in the \textbf{critical region} the null hypothesis will be rejected.

The \textbf{boundary values} of the critical region are called the critical values.
\section{Method}
\begin{enumerate}
\item Establish the null and alternative hypothesis ($H_0$ and $H_1$)
\item Define distribution under $H_0$
\item Decide on the significance level
\item Collect data, state the test statistic, X=
\item Calculate the probability of obtaining the test statistic or a more extreme result (same direction as $H_1$)
\item Compare this to the sig level as a decimal
\begin{itemize}
\item If \textbf{greater} than the sig level, it is a \textbf{non significant} result, it is not in the critical region and we \textbf{do not} reject $H_0$
\item If \textbf{less} than sig level, it is a \textbf{significant result}, it is in the critical region and we \textbf{reject} $H_0$ 
\end{itemize}
\item Interpret the results in terms of the original claim
\end{enumerate}
\subsection{Example}
\textbf{Establish the null and alternative hypothesis}\\
$H_0:p=0.5$\\
$H_1:p>0.5$\\
\\
\textbf{Define the distribution under $\mathbf{H_O}$}\\
Under $H_0$ $X\sim B(15,0.5)$\\
\\
\textbf{Decide on the significance level}\\
$5\%$\\
\\
\textbf{Collect data, state the test statistic}\\
X=12\\
\\
\textbf{Calculate the probability of obtaining the test statistic or a more extreme result}\\
\begin{tabular}{r l}
$P(X\geqslant 12)$&$=1-P(X\leqslant 11)$\\
&$=1-0.9824$\\
&$=0.0176$\\
\end{tabular}
\\
\textbf{Compare this to the sig level as a decimal}
$0.0176<0.05$\\
\\
\textbf{Interpret the results in terms of the original claim}\\
There is evidence to reject $H_0$ in favour of $H_1$. The test is significant.
\subsection{Finding critical values}
We require a value c such that:\\
$P(X\geqslant c)<0.05$\\
$1-P(X\leqslant c-1)<0.05$\\
$P(X\leqslant c-1)>0.95$\\
\\
\textbf{Test against tables}\\
$P(X<11)=0.9824$\\
\\
$c-1=11$\\
$c=12$
\subsection{Two tailed tests}


\end{document}