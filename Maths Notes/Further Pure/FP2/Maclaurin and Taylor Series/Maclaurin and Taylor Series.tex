\documentclass{article}[18pt]
\usepackage[utf8]{inputenc}
\usepackage[margin=0.7in]{geometry}
\usepackage{parselines} 
\usepackage{amsmath}
\usepackage{titlesec}
\usepackage{pgfplots}
\usepgfplotslibrary{fillbetween}
\usepackage{graphicx}
\usepackage[english]{babel}
\usepackage{fancyhdr}
\usepackage{gensymb}
\usepackage{relsize}
\pgfplotsset{width=10cm,compat=1.9}
\usepackage[super]{nth}
\titlespacing\section{0pt}{14pt plus 4pt minus 2pt}{0pt plus 2pt minus 2pt}
\newlength\tindent
\setlength{\tindent}{\parindent}
\setlength{\parindent}{0pt}
\renewcommand{\indent}{\hspace*{\tindent}}

\pagestyle{fancy}
\fancyhf{}
\rhead{Sam Robbins 13SE}
\lhead{A Level Maths - FP2}
\rfoot{Page \thepage}

\usepackage{cancel}

\begin{document}
\begin{center}
\underline{\huge Maclaurin and Taylor Series}
\end{center}
\section{Maclaurin's expansion}
$$f(x)=f(0)+f'(0)x+f''(0)\frac{x^2}{2!}+f'''(0)\frac{x^3}{3!}+...+f^r(0)\frac{x^r}{r!}...$$
\\
For the continuous function, f, given by $f:x\Rightarrow f(x)$(where $x$ is real), then providing $f(0),f'(0),f''(0)$ etc all have finite values. This is an infinite series.
\subsection{Example}
\textit{Given that $f(x)=e^x$ can be written as an infinite series in the form:}
$$f(x)=e^x=a_0+a_1x+a_2x^2+a_3x^3+...+a_rx^4+...$$
\textit{And that it is valid to differentiate an infinite series term by term, show that:}
$$e^2=1+x+\frac{x^2}{2!}+\frac{x^3}{3!}+...+\frac{x^r}{r!}+...$$
\\
\\
\textcolor{red}{Find up to the third differential of $f(x)$ and the value of zero for each}\\
\begin{tabular}{l l}
$f(x)=e^x$&$f(0)=1$\\
$f'(x)=e^x$&$f''(0)=1$\\
$f''(x)=e^x$&$f''(0)=1$\\
$f'''(x)=e^x$&$f'''(0)=1$
\end{tabular}
\\
\\
$$f(x)=1+1\times x+1\times\frac{x^2}{2!}+1\times\frac{x^3}{3!}$$
$$f(x)=1+x+\frac{x^2}{2!}+\frac{x^3}{3!}+...+\frac{x^r}{r!}+...$$
\subsection{Standard results}
Standard results are given on the data sheet, these can then be used for adapted forms of the results also. Remember to consider the limits where appropriate.
\newpage
\section{Taylor expansion}
The conditions of the Maclaurin expansion mean that some functions, such as $\ln x$ cannot be expanded as a series in ascending powers of x.\\
\\
The construction of the Maclaurin expansion focuses on $x=0$ and values of x very close to zero. The Taylor expansion focuses on $x=a$.\\
\\
Considering the functions f and g, where $f(x+a)\equiv g(x)$ then:\\
$f^r(a)=g^r(0)$\\
\\
Turning the Maclaurin expansion for g from:
$$g(x)=g(0)+g'(0)x+\frac{g''(0)}{2!}x^2...$$
Into
$$f(x+a)=f(a)+f'(a)x+\frac{f''(a)}{2!}x^2+...+\frac{f^r(a)}{r!}x^r$$
Replacing x by x-a gives
$$f(x)=f(a)+f'(a)(x-a)+\frac{f''(a)}{2!}(x-a)^2+...+\frac{f^r(a)}{r!}(x-a)^r$$
These are the two forms of the Taylor expansion, when a=0, they both become the Maclaurin expansion.
\end{document}