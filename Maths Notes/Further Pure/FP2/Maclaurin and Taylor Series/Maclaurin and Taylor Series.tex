\documentclass{article}[18pt]
\usepackage[utf8]{inputenc}
\usepackage[margin=0.7in]{geometry}
\usepackage{parselines} 
\usepackage{amsmath}
\usepackage{titlesec}
\usepackage{pgfplots}
\usepgfplotslibrary{fillbetween}
\usepackage{graphicx}
\usepackage[english]{babel}
\usepackage{fancyhdr}
\usepackage{gensymb}
\usepackage{relsize}
\pgfplotsset{width=10cm,compat=1.9}
\usepackage[super]{nth}
\titlespacing\section{0pt}{14pt plus 4pt minus 2pt}{0pt plus 2pt minus 2pt}
\newlength\tindent
\setlength{\tindent}{\parindent}
\setlength{\parindent}{0pt}
\renewcommand{\indent}{\hspace*{\tindent}}

\pagestyle{fancy}
\fancyhf{}
\rhead{Sam Robbins 13SE}
\lhead{A Level Maths - FP2}
\rfoot{Page \thepage}

\usepackage{cancel}

\begin{document}
\begin{center}
\underline{\huge Maclaurin and Taylor Series}
\end{center}
\section{Maclaurin's expansion}
$$f(x)=f(0)+f'(0)x+f''(0)\frac{x^2}{2!}+f'''(0)\frac{x^3}{3!}+...+f^r(0)\frac{x^r}{r!}...$$
\\
For the continuous function, f, given by $f:x\Rightarrow f(x)$(where $x$ is real), then providing $f(0),f'(0),f''(0)$ etc all have finite values. This is an infinite series.
\subsection{Example}
\textit{Given that $f(x)=e^x$ can be written as an infinite series in the form:}
$$f(x)=e^x=a_0+a_1x+a_2x^2+a_3x^3+...+a_rx^4+...$$
\textit{And that it is valid to differentiate an infinite series term by term, show that:}
$$e^2=1+x+\frac{x^2}{2!}+\frac{x^3}{3!}+...+\frac{x^r}{r!}+...$$
\\
\\
\textcolor{red}{Find up to the third differential of $f(x)$ and the value of zero for each}\\
\begin{tabular}{l l}
$f(x)=e^x$&$f(0)=1$\\
$f'(x)=e^x$&$f''(0)=1$\\
$f''(x)=e^x$&$f''(0)=1$\\
$f'''(x)=e^x$&$f'''(0)=1$
\end{tabular}
\\
\\
$$f(x)=1+1\times x+1\times\frac{x^2}{2!}+1\times\frac{x^3}{3!}$$
$$f(x)=1+x+\frac{x^2}{2!}+\frac{x^3}{3!}+...+\frac{x^r}{r!}+...$$
\subsection{Standard results}
Standard results are given on the data sheet, these can then be used for adapted forms of the results also. Remember to consider the limits where appropriate.
\newpage
\section{Taylor expansion}
The conditions of the Maclaurin expansion mean that some functions, such as $\ln x$ cannot be expanded as a series in ascending powers of x.\\
\\
The construction of the Maclaurin expansion focuses on $x=0$ and values of x very close to zero. The Taylor expansion focuses on $x=a$.\\
\\
Considering the functions f and g, where $f(x+a)\equiv g(x)$ then:\\
$f^r(a)=g^r(0)$\\
\\
Turning the Maclaurin expansion for g from:
$$g(x)=g(0)+g'(0)x+\frac{g''(0)}{2!}x^2...$$
Into
$$f(x+a)=f(a)+f'(a)x+\frac{f''(a)}{2!}x^2+...+\frac{f^r(a)}{r!}x^r$$
Replacing x by x-a gives
$$f(x)=f(a)+f'(a)(x-a)+\frac{f''(a)}{2!}(x-a)^2+...+\frac{f^r(a)}{r!}(x-a)^r$$
These are the two forms of the Taylor expansion, when a=0, they both become the Maclaurin expansion.
\subsection{Example}
\textit{Find the Taylor expansion of $\cos2x$, in ascending powers of $(x-\frac{\pi}{4})$ up to $(x-\frac{\pi}{4})^5$}
\textcolor{red}{Find the differentials of $f(x)$ up to the fifth derivative, and the associated values when $x=\frac{\pi}{4}$}\\
\begin{tabular}{l l}
$f(x)=\cos2x$&$f(\frac{\pi}{4})=0$\\
$f'(x)=-2\sin2x$&$f(\frac{\pi}{4})=-2$\\
$f''(x)=-4\cos2x$&$f(\frac{\pi}{4})=0$\\
$f'''(x)=8\sin2x$&$f(\frac{\pi}{4})=8$\\
$f''''(x)=16\cos2x$&$f(\frac{\pi}{4})=0$\\
$f'''''(x)=-32\sin2x$&$f(\frac{\pi}{4})=-32$
\end{tabular}\\
\textcolor{red}{Substitute in the associated values into the formula}
$$\cos2x=0-2\Big(x-\frac{\pi}{4}\Big)+0+\frac{8}{3!}\Big(x-\frac{\pi}{4}\Big)^3+0-\frac{32}{5!}\Big(x-\frac{\pi}{4}\Big)^5$$
$$\cos2x=-2\Big(x-\frac{\pi}{4}\Big)+\frac{4}{5}\Big(x-\frac{\pi}{4}\Big)^3-\frac{4}{15}\Big(x-\frac{\pi}{4}\Big)^5$$



\newpage
\section{Finding the solution, in the form of a series to a differential equation using the Taylor series method}
Suppose you have a first order differential equation of the form $\frac{dy}{dx}=f(x,y)$ and you know the initial condition that at $x=x_0$,$y=y_0$, then you can calculate \Large{$(\frac{dy}{dx})_{x_0}$} \normalsize by substituting $x_0$ and $y_0$ into the original differential equation.\\
\\
By successive differentiation of the original differential equation, the values of {\Large $(\frac{d^2y}{dx^2})_{x_0}$} and {\Large $(\frac{d^3y}{dx^3})_{x_0}$} and so on can be found by substituting previous results into the derived equations.\\
\\
The series solution to the differential equation is found using the Taylor series in the form:
$$y=y_0+(x-x_0)\bigg(\frac{dy}{dx}\bigg)_{x_0}+\frac{(x-x_0)^2}{2!}\Bigg(\frac{d^2y}{dx^2}\Bigg)_{x_0}+\frac{(x-x_0)^3}{3!}\Bigg(\frac{d^3y}{dx^3}\Bigg)_{x_0}+...$$
In the common situation where $x_0=0$ then this reduces to the Maclaurin series
$$y=y_0+x\bigg(\frac{dy}{dx}\bigg)_{x_0}+\frac{x^2}{2!}\Bigg(\frac{d^2y}{dx^2}\Bigg)_{x_0}+\frac{x^3}{3!}\Bigg(\frac{d^3y}{dx^3}\Bigg)_{x_0}+...$$
\subsection{Example}
\textit{Using the Taylor method to find a series solution, in ascending powers of x up to and including the term in $x^3$, of:}
$$\frac{d^2y}{dx^2}=y-\sin x$$
\textit{Given that when $x=0$,$y=1$ and $\frac{dy}{dx}=2$}
\\
\textcolor{red}{Use the formula to find $\frac{d^2y}{dx^2}$}
$$\frac{d^2y}{dx^2}=1-\sin(0)=1$$
$$\mathbf{\frac{d^2y}{dx^2}=1}$$
\textcolor{red}{Differentiate the formula to obtain a formula for $\frac{d^3y}{dx^3}$}
$$\frac{d^3y}{dx^3}=\frac{dy}{dx}-\cos x=2-\cos(0)=1$$
$$\mathbf{\frac{d^3y}{dx^3}=1}$$
\textcolor{red}{Substitute the known values into the Maclaurin formula}
$$y=1+x\times2+\frac{x^2}{2!}\times1+\frac{x^3}{3!}\times1$$
$$y=1+2x+\frac{x^2}{2}+\frac{x^3}{6}$$
\end{document}