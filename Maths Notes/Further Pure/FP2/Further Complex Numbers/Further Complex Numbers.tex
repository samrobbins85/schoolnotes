\documentclass{article}[18pt]
\usepackage[utf8]{inputenc}
\usepackage[margin=0.7in]{geometry}
\usepackage{parselines} 
\usepackage{amsmath}
\usepackage{titlesec}
\usepackage{pgfplots}
\usepgfplotslibrary{fillbetween}
\usepackage{graphicx}
\usepackage[english]{babel}
\usepackage{fancyhdr}
\usepackage{gensymb}
\usepackage{relsize}
\pgfplotsset{width=10cm,compat=1.9}
\usepackage[super]{nth}
\titlespacing\section{0pt}{14pt plus 4pt minus 2pt}{0pt plus 2pt minus 2pt}
\newlength\tindent
\setlength{\tindent}{\parindent}
\setlength{\parindent}{0pt}
\renewcommand{\indent}{\hspace*{\tindent}}

\pagestyle{fancy}
\fancyhf{}
\rhead{Sam Robbins 13SE}
\lhead{A Level Maths - FP2}
\rfoot{Page \thepage}

\usepackage{cancel}

\begin{document}
\begin{center}
\underline{\huge Further Complex Numbers}
\end{center}
\section{Expressions of complex numbers}
\subsection{$\mathbf{x+iy}$}
This expresses the coordinate of the point at the end of the vector on the argand diagram.
\\
\begin{center}

\begin{tikzpicture}[scale=0.6]
\begin{axis}[axis lines=middle,ymin=-2.5,ymax=2.5,xmin=-2.5,xmax=2.5,
xtick={-2},xticklabels={$\sqrt{3}$},ytick={1},yticklabels={1},ytick pos=right]
\pgfplotsset{every tick label/.append style={font=\Large}}

\addplot coordinates { (0,0) (-2,1) };
\addplot [dashed] coordinates { (-2,0) (-2,1) };
\addplot [dashed] coordinates { (0,1) (-2,1) };
\end{axis}
\end{tikzpicture}

\end{center}
\subsection{$\mathbf{r(\cos\theta+i\sin\theta)}$}
This expresses the length of the line and the angle anticlockwise from the positive x axis
\begin{center}

\begin{tikzpicture}[scale=0.6]
\begin{axis}[axis lines=middle,ymin=-2.5,ymax=2.5,xmin=-2.5,xmax=2.5,ticks=none]
\pgfplotsset{every tick label/.append style={font=\Large}}

\addplot coordinates { (0,0) (-2,1) };
\draw (axis cs: 0.7,0) arc[radius=80, start angle= 0, end angle= 150];
    \node[] at (axis cs: 0.7,0.7) {$\mathlarger{\frac{5\pi}{6}}$};
\node[blue] at (axis cs: -0.9,0.7) {$\mathlarger{2}$};

\end{axis}
\end{tikzpicture}
\end{center}
$$2\Bigg(\cos\Big(\frac{5\pi}{6}\Big)+i\sin\Big(\frac{5\pi}{6}\Big)\Bigg)$$
\subsection{$re^{i\theta}$}
This uses the same parameters as $r(\cos\theta+i\sin\theta)$
\newpage
\section{Multiplying and dividing complex numbers}
\subsection{Multiplying}
\subsubsection{Trigonometric form}
$$Z_1Z_2=r_1(\cos\theta_1+i\sin\theta_1)\times r_2(\cos\theta_2+i\sin\theta_2)$$
$$=r_1r_2(\cos\theta_1\cos\theta_2-\sin\theta_1\sin\theta_2+i\cos\theta_1\sin\theta_2+i\sin\theta_1\cos\theta_2)$$
Apply the cos addition formula to the first two terms
$$=r_1r_2(\cos(\theta_1+\theta_2)+i\cos\theta_1\sin\theta_2+i\sin\theta_1\cos\theta_2)$$
Apply the sin addition formula to the last two terms
$$=r_1r_2(\cos(\theta_1+\theta_2)+\sin(\theta_1+\theta_2))$$
\subsubsection{Exponential form}
$$\mathlarger{Z_1Z_2=r_1e^{i\theta_1}\times r_2e^{i\theta_2}}$$
Apply laws of indices 
$$\mathlarger{Z_1Z_2=r_1r_2e^{i(\theta_1+\theta_2)}}$$
\subsection{Dividing}

\end{document}