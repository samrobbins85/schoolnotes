\documentclass{article}[18pt]
\usepackage[utf8]{inputenc}
\usepackage[margin=0.7in]{geometry}
\usepackage{parselines} 
\usepackage{amsmath}
\usepackage{titlesec}
\usepackage{pgfplots}
\usepackage{tabularx}
\usepgfplotslibrary{fillbetween}
\usepackage{graphicx}
\usepackage[english]{babel}
\usepackage{fancyhdr}
\usepackage{gensymb}
\usepackage{relsize}
\pgfplotsset{width=10cm,compat=1.9}
\usepackage[super]{nth}
\titlespacing\section{0pt}{14pt plus 4pt minus 2pt}{0pt plus 2pt minus 2pt}
\newlength\tindent
\setlength{\tindent}{\parindent}
\setlength{\parindent}{0pt}
\renewcommand{\indent}{\hspace*{\tindent}}

\pagestyle{fancy}
\fancyhf{}
\rhead{Sam Robbins 13SE}
\lhead{A Level Maths - FP2}
\rfoot{Page \thepage}

\usepackage{cancel}

\begin{document}
\begin{center}
\underline{\huge Further Complex Numbers}
\end{center}
\section{Expressions of complex numbers}
\subsection{$\mathbf{x+iy}$}
This expresses the coordinate of the point at the end of the vector on the argand diagram.
\\
\begin{center}

\begin{tikzpicture}[scale=0.6]
\begin{axis}[axis lines=middle,ymin=-2.5,ymax=2.5,xmin=-2.5,xmax=2.5,
xtick={-2},xticklabels={$\sqrt{3}$},ytick={1},yticklabels={1},ytick pos=right]
\pgfplotsset{every tick label/.append style={font=\Large}}

\addplot coordinates { (0,0) (-2,1) };
\addplot [dashed] coordinates { (-2,0) (-2,1) };
\addplot [dashed] coordinates { (0,1) (-2,1) };
\end{axis}
\end{tikzpicture}

\end{center}
\subsection{$\mathbf{r(\cos\theta+i\sin\theta)}$}
This expresses the length of the line and the angle anticlockwise from the positive x axis
\begin{center}

\begin{tikzpicture}[scale=0.6]
\begin{axis}[axis lines=middle,ymin=-2.5,ymax=2.5,xmin=-2.5,xmax=2.5,ticks=none]
\pgfplotsset{every tick label/.append style={font=\Large}}

\addplot coordinates { (0,0) (-2,1) };
\draw (axis cs: 0.7,0) arc[radius=80, start angle= 0, end angle= 150];
    \node[] at (axis cs: 0.7,0.7) {$\mathlarger{\frac{5\pi}{6}}$};
\node[blue] at (axis cs: -0.9,0.7) {$\mathlarger{2}$};

\end{axis}
\end{tikzpicture}
\end{center}
$$2\Bigg(\cos\Big(\frac{5\pi}{6}\Big)+i\sin\Big(\frac{5\pi}{6}\Big)\Bigg)$$
\subsection{$re^{i\theta}$}
This uses the same parameters as $r(\cos\theta+i\sin\theta)$
\newpage
\section{Multiplying and dividing complex numbers}
\subsection{Multiplying}
\subsubsection{Trigonometric form}
$$Z_1Z_2=r_1(\cos\theta_1+i\sin\theta_1)\times r_2(\cos\theta_2+i\sin\theta_2)$$
$$=r_1r_2(\cos\theta_1\cos\theta_2-\sin\theta_1\sin\theta_2+i\cos\theta_1\sin\theta_2+i\sin\theta_1\cos\theta_2)$$
Apply the cos addition formula to the first two terms
$$=r_1r_2(\cos(\theta_1+\theta_2)+i\cos\theta_1\sin\theta_2+i\sin\theta_1\cos\theta_2)$$
Apply the sin addition formula to the last two terms
$$=r_1r_2(\cos(\theta_1+\theta_2)+\sin(\theta_1+\theta_2))$$
\subsubsection{Exponential form}
$$\mathlarger{Z_1Z_2=r_1e^{i\theta_1}\times r_2e^{i\theta_2}}$$
Apply laws of indices 
$$\mathlarger{Z_1Z_2=r_1r_2e^{i(\theta_1+\theta_2)}}$$
\subsection{Dividing}
\subsubsection{Trigonometric form}
$$\frac{Z_1}{Z_2}=\frac{r_1(\cos\theta_1+i\sin\theta_1)}{r_2(\cos\theta_2+i\sin\theta_2)}$$
Multiply by the complex conjugate 
$$\frac{Z_1}{Z_2}=\frac{r_1(\cos\theta_1+i\sin\theta_1)}{r_2(\cos\theta_2+i\sin\theta_2)}\times\frac{\cos\theta_2-i\sin\theta_2}{\cos\theta_2-i\sin\theta_2}$$
Expand
$$\frac{Z_1}{Z_2}=\frac{r_1}{r_2}\times\frac{cos\theta_1\cos\theta_2-i\cos\theta_1\sin\theta_2+i\sin\theta_1\cos\theta_2+\sin\theta_1\sin\theta_2}{\cos^2\theta_2-i\cos\theta_2\sin\theta_2+i\sin\theta_2\cos\theta_2+\sin^2\theta_2}$$
Simplify
$$\frac{Z_1}{Z_2}=\frac{r_1}{r_2}\times\Big(\cos(\theta_1-\theta_2)+i\sin(\theta_1-\theta_2)\Big)$$
\subsubsection{Exponential form}
$$\mathlarger{\frac{Z_1}{Z_2}=\frac{r_1}{r_2}\times\frac{e^{i\theta_1}}{e^{i\theta_2}}}$$
$$\mathlarger{\frac{Z_1}{Z_2}=\frac{r_1}{r_2}\times e^{i(\theta_1-\theta_2)}}$$
\subsection{Comparison}
\begin{tabularx}{\textwidth}{|X|X|}
\hline
Multiplying&Dividing\\
\hline	
Multiply modulus, add arguments&Divide modulus, subtract arguments\\
\hline
\end{tabularx}
\newpage
\section{De Moivre's Theorem}
\textcolor{red}{$$\mathlarger{[r(\cos\theta+i\sin\theta)]^n=r^n(\cos(n\theta)+i\sin(n\theta))}$$}
\subsection{Positive proof}
Prove true for n=1
$$r(\cos\theta+i\sin\theta)=r(\cos\theta+i\sin\theta)$$
\begin{center}
True for n=1
\end{center}
Assume true for n=k
$$\mathlarger{[r(\cos\theta+i\sin\theta)]^k=r^k(\cos(k\theta)+i\sin(k\theta))}$$
Prove true for n=k+1
$$[r(\cos\theta+i\sin\theta)]^{k+1}$$
$$[r(\cos\theta+i\sin\theta)]^k\times(r(\cos\theta+i\sin\theta))^1$$
$$r^k(\cos(k\theta)+i\sin(k\theta))\times r(\cos\theta+i\sin\theta)$$
$$r^kr(\cos(k\theta+\theta)+i\sin(k\theta+\theta))$$
$$r^{k+1}(\cos((k+1)\theta)+i\sin((k+1)\theta))$$
True
\subsection{Negative proof}
n=-m
$$[r(\cos\theta+i\sin\theta)]^{-m}$$
Multiply by complex conjugate
$$\frac{1}{[r(\cos\theta+i\sin\theta)]^m}\times\frac{[r(\cos\theta-i\sin\theta)]^m}{[r(\cos\theta-i\sin\theta)]^m}$$
Apply positive De Moivre's Theorem
$$\mathlarger{\frac{r^m(\cos(m\theta)-i\sin(m\theta))}{r^m(\cos(m\theta)+i\sin(m\theta))\times r^m(\cos(m\theta)-i\sin(m\theta))}}$$
Simplify and expand
$$\mathlarger{\frac{\cos(m\theta)-i\sin(m\theta)}{r^m(\cos^2m\theta-i\cos m\theta\sin m\theta+i\cos m\theta\sin m\theta+\sin^2m\theta}}$$
Simplify
$$\frac{\cos m\theta-i\sin m\theta}{r^m}=r^{-m}(\cos m\theta-i\sin m\theta)$$
Rewrite
$$r^{-m}(\cos(-m\theta)+i\sin(-m\theta))$$
Replace -m with n
$$r^n(\cos(n\theta)+i\sin(n\theta))$$

\end{document}