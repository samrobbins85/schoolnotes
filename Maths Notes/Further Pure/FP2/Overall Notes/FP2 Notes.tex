\documentclass{article}[18pt]
\usepackage{../../../../format}
\lhead{A Level Maths - FP2}

\begin{document}
\begin{center}
\underline{\huge FP2}
\end{center}
\section{First order differential equations}
\subsection{Solving first order DE using an integrating factor}
Solving $\dfrac{dy}{dx}+P(x)y=Q(x)$\\
\\
IF(Integrating factor) is found by finding $\mathlarger{e^{\int p(x) \ dx}}$ And multiplying the DE by the IF.\\
\\
This will result in the DE being in the form:
$$f(x)\frac{dy}{dx}+f'(x)y$$
This form can then be shortened by integrating:
$$\int f'(x)g(x)+f(x)g'(x) dx=f(x)g(x)+c$$
Integrate both sides then simplify
\section{Further complex numbers}
\subsection{Converting between forms}
When converting from $x+iy$ to a form in r and $\theta$, take the angle from the positive $x$ axis
\subsection{Multiplying and dividing complex numbers}
It is easiest to use the exponential form, then convert if needed
\subsubsection{Multiplying}
$$Z_1Z_2=r_1e^{i\theta_1}\times r_2e^{i\theta_2}=r_1r_2e^{i(\theta_1+\theta_2)}$$
\subsubsection{Dividing}
$$\frac{Z_1}{Z_2}=r_1e^{i\theta_1}\div r_2e^{i\theta_2}=\frac{r+1}{r+2} e^{i(\theta_1-\theta_2)}$$
\subsection{De Moivre's theorem}
This is given on the data sheet
\subsubsection{Z formulas}
If $z=\cos\theta+i\sin\theta$
$$z+\frac{1}{z}=2\cos\theta$$
$$z-\frac{1}{z}=2i\sin\theta$$
$$z^n+\frac{1}{z^n}=2\cos n\theta$$
$$z^n-\frac{1}{z^n}=2i\sin n\theta$$
\subsection{Loci on the complex plane}
\begin{tabularx}{\textwidth}{|X|X|}
\hline
Equation&Description\\
\hline
{\Large$|z-z_1|=r$}&A circle centre $(x_1,y_1)$ with a radius r\\
\hline
{\Large$|z-z_1|=|z-z_2|$}&A perpendicular bisector of the line segment joining points $z_1$ and $z_2$\\
\hline
{\Large$\arg(z-z_1)=\theta$}&The half line from a fixed point $z_1$, making an angle $\theta$ with the positive real axis\\
\hline
{\Large$\arg(\frac{z-z_1}{z-z_2})=\theta$}&An arc between the points $z_1$ and $z_2$ where the angle the lines from $z_1$ and $z_2$ to any point on the arc is $\theta$\\
\hline
\end{tabularx}
\subsection{Translations}
\begin{itemize}
\item $w=z+a+ib$ represents a translation with translation vector $\begin{pmatrix}
a\\b
\end{pmatrix}$
\item $w=kz$ represents an enlargement with scale factor $k$ centre $(0,0)$
\item $w=kz+a+ib$ represents an enlargement scale factor $k$ centre $(0,0)$ followed by a translation with translation vector $\begin{pmatrix}
a\\b
\end{pmatrix}$
\item $w=z^2$ multiply a shape by itself, for example a circle of radius 4 would go to radius 16
\end{itemize}
\section{Inequalities}
We can build upon our previous algebraic skills in order to solve more complex inequalities\\
Remember:
\begin{itemize}
\item Don't multiply anything that could be negative - use "squared" things
\item Find the critical values (f(x)=0)
\item Sketch the graph to solve
\end{itemize}
\section{Maclaurin and Taylor Series}
Use the formulas on the data sheet
\subsection{Solving differential equations using the Taylor expansion}
From the differential equation, calculate the values of $\frac{dy}{dx}$, $\frac{d^2y}{dx^2}$ etc up to whatever is needed.\\
Then substitute those values into the Taylor series to solve the differential equation



\end{document}