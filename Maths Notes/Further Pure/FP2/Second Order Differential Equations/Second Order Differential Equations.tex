\documentclass{article}[18pt]
\usepackage{/home/sam/Documents/School_Notes/format}
\lhead{A Level Maths - FP2}

%File specific Preable
\usepackage[super]{nth} %Provides 2nd etc


\begin{document}
\begin{center}
\underline{\huge Second Order Differential Equations}
\end{center}
In FP2 we are interested in solving 2nd ODEs of the form:
$$a\frac{d^2y}{dx^2}+b\frac{dy}{dx}+cy=f(x) \quad \quad \textrm{a,b,c are constants}$$
\\
We consider three distinct cases:\\
$b^2>4ac\quad$  (Two real solutions)\\
$b^2=4ac\quad$  (One repeated solution)\\
$b^2<4ac\quad$  (Two complex solutions)\\
\\
To solve \nth{2} ODEs of this form we first consider solutions to:
$$a\frac{d^2y}{dx^2}+b\frac{dy}{dx}+cy=0$$
The process of solving a \nth{2} ODE starts with a general solution to a \nth{1} ODE of form:
$$b\frac{dy}{dx}+cy=0$$
$$\mathlarger{\int}\frac{1}{b} \ dy=\mathlarger{\int}\frac{1}{-cy} \ dy$$
$$b\ln(y)=-cx+k$$
$$y=Ae^{-\frac{c}{b}x}$$
$$y=Ae^{mx}$$
\textcolor{red}{This was suggested to be a solution to the \nth{2} ODE as well}\\
\\
We take $y=e^{mx}$ as a starting point for finding general solutions to:
$$(1) \quad a\frac{d^2y}{dx^2}+b\frac{dy}{dx}+cy=0$$\\
If $y=e^{mx}$ is a solution to (1):
$$\frac{dy}{dx}=me^{mx}$$
$$\frac{d^2y}{dx^2}=m^2e^{mx}$$
Then substitute this into (1)
$$am^2e^{mx}+bme^{mx}+ce^{mx}=0$$
Factor out $e^{mx}$
$$e^{mx}(am^2+bm+c)$$
As $e^x$ must be greater than zero $am^2+bm+c=0$\\
This is a solvable quadratic called the \textbf{Auxiliary equation}
\newpage
\section{Two real roots $\mathbf{b^2>4ac}$}
$$(1) \quad a\frac{d^2y}{dx^2}+b\frac{dy}{dx}+cy=0$$\\
The general solution to (1) is in the form:
$$\mathlarger{y=Ae^{\alpha x}+Be^{\beta x}}$$
Where A and B are constants and $\alpha$ and $\beta$ are the roots to the AE
\subsection{Example}
$$(1) \quad 2\frac{d^2y}{dx^2}+5\frac{dy}{dx}+3y=0$$
Find the auxiliary equation
$$2m^2e^{mx}+5me^{mx}+3e^{mx}=0$$
$$e^{mx}(2m^2+5m+3=0$$
$m=-\dfrac{3}{2}\quad m=-1$
$$\textrm{General solution:} y=Ae^{\alpha x}+Be^{\beta x}$$
$$GS=Ae^{-\frac{3}{2}x}+Be^{-x}$$
\section{1 Real, Repeated root $\mathbf{b^2=4ac}$}
$$\textrm{General solution}: \mathlarger{(A+bx)e^{\alpha x}}$$
A and B are constants and $\alpha$ is the root of the AE
\subsection{Example}
\textit{Find the general solution of:}
$$\frac{d^2y}{dx^2}+8\frac{dy}{dx}+16y=0$$
Find the auxiliary equation
$$e^{mx}(m^2+8m+16)=0$$
Find the solution
$$m=-4$$
Substitute into the general solution formula
$$y=(A+Bx)e^{-4x}$$
\section{Imaginary only roots $\mathbf{b^2<4ac}$}
This is when the AI has roots of form $\pm \alpha i$
$$\textrm{General solution:} \quad \mathlarger{y=A\cos(\alpha x)+B\sin(\alpha x)}$$
\newpage
\section{Complex roots $\mathbf{b^2<4ac}$}
This is used when the root is in the form $\beta\pm\alpha i$
$$\textrm{General solution:} \quad \mathlarger{y=e^{\beta x}(A\cos(\alpha x)+B\sin(\alpha x))}$$
\subsection{Example}
\textit{Find the general solution of:}
$$\frac{d^2y}{dx^2}-6\frac{dy}{dx}+34y=0$$
Find the auxiliary equation
$$m^2-6m+34=0$$
Solve to find roots
$$\textrm{Roots}=\frac{6\pm\sqrt{36-4\times1\times34}}{2}=\frac{6\pm10i}{2}=3\pm5i$$
Substitute into general solution formula
$$\mathlarger{y=e^{3x}(A\cos(5x)+B\sin(5x))}$$
\section{Solving \nth{2} ODE = f(x)}
Of the type:
$$a\frac{d^2y}{dx^2}+b\frac{dy}{dx}+cy=f(x)$$
There are set forms of f(x)\\
\\
The LHS will be solved in the standard way and the general solution of the LHS will be called the \textbf{complementary solution} (CS)\\
\\
Solving the RHS will give us a \textbf{particular integral} (PI)
\begin{center}
Full general solution=Complementary function+Particular integral
\end{center}
\subsection{Standard forms of f(x)}
$f(x)=\lambda$\\
$f(x)=\lambda+\mu x$\\
$f(x)=\lambda+\mu x+\nu x^2$\\
$f(x)=ke^{px}$\\
$f(x)=m\cos\omega x$\\
$f(x)=m\sin\omega x$\\
$f(x)=m\cos\omega x\pm n\sin\omega x$
\newpage
\subsection{Examples}
$$\frac{d^2y}{dx^2}-5\frac{dy}{dx}+6y=f(x)$$
Find complementary function
$$m^2-5m+6=0$$
$$m=2 \quad m=2$$
$$\textrm{Complementary funtion}=Ae^{3x}+Be^{2x}$$
\subsubsection{\nth{2} ODE=$\lambda$}
$f(x)=3$\\
Start with $y=\lambda$
$$y=\lambda$$
$$\frac{dy}{dx}=0$$
$$\frac{d^2y}{dx^2}=0$$
Substitute values into LHS
$$0-5\times0+6\lambda=3$$
$$\lambda=\frac{1}{2}$$
Add Complementary function to particular integral
$$y=Ae^{3x}+Be^{2x}+\frac{1}{2}$$
\subsubsection{\nth{2} ODE=$\lambda+\mu x$}
$f(x)=2x$\\
Start with $y=\lambda+\mu x$
$$y=\lambda+\mu x$$
$$\frac{dy}{dx}=\mu$$
$$\frac{d^2y}{dx^2}=0$$
Substitute into LHS
$$0-5\mu+6(\lambda+\mu x)=2x$$
Equate x terms
$$6\mu x=2x$$
$$\mu=\frac{1}{3}$$
Equate constant terms
$$-\frac{5}{3}+6\lambda=0$$
$$6\lambda=\frac{5}{3}$$
$$\lambda=\frac{5}{18}$$
Substitute into form for the particular integral
$$y=\frac{1}{3}x+\frac{5}{18}$$
Add the PI and CF to find the general solution
$$y=Ae^{3x}+Be^{2x}+\frac{1}{3}x+\frac{5}{18}$$
\newpage
\subsubsection{\nth{2} ODE=$\lambda+\mu x+\nu x^2$}
$f(x)=3x^2$\\
Start with $y=\lambda+\mu x+\nu x^2$
$$y=\lambda+\mu x+\nu x^2$$
$$\frac{dy}{dx}=\mu+2\nu x$$
$$\frac{d^2y}{dx^2}=2\nu$$
Substitute into the LHS
$$2\nu-5(\mu+2\nu x)+6(\lambda+\mu x+\nu x^2)=3x^2$$
Equate $x^2$ terms
$$6\nu=3 \quad \nu=\frac{1}{2}$$
Equate x terms
$$-10\times\frac{1}{2}\times x+6\mu x=0$$
$$-5+6\mu=0$$
$$\mu=\frac{5}{6}$$
Equate constant coefficients
$$1-5\times\frac{5}{6}+6\lambda=0$$
$$6\lambda=\frac{19}{6}$$
$$\lambda=\frac{19}{36}$$
Substitute into the particular integral form
$$y=\frac{1}{2}x^2+\frac{5}{6}x+\frac{19}{36}$$
Add CF and PI to get the general solution
$$y=Ae^{3x}+Be^{2x}+\frac{1}{2}x^2+\frac{5}{6}x+\frac{19}{36}$$
\newpage
\subsubsection{Trigonometric f(x)}
General forms:\\
If $f(x)=m\cos\omega x$
$$PI: \ y=P\cos\omega x+Q\sin\omega x$$
If $f(x)=n\sin\omega x$
$$PI: \ y=P\cos\omega x+Q\sin\omega x$$
If $f(x)=m\cos\omega x\pm n\sin\omega x$
$$PI: \ y=P\cos\omega x+Q\sin\omega x$$
\paragraph{$f(x)=m\sin\omega x$ or $n\sin\omega x$}
$f(x)=13\sin4x$
$$PI: \ y=P\cos\omega x+Q\sin\omega x$$
$$\frac{dy}{dx}=-\omega P\sin\omega x+\omega Q\cos\omega x$$
$$\frac{d^2y}{dx^2}=-\omega^2P\cos\omega x-\omega^2Q\sin\omega x$$
$$\textcolor{red}{\frac{d^2y}{dx^2}-5\frac{dy}{dx}+6y=13\sin3x}$$
Substitute into LHS
$$-\omega^2P\cos\omega x-\omega^2Q\sin\omega x-5(-\omega P\sin\omega x+\omega Q\cos\omega x)+6(P\cos\omega x+Q\sin\omega x)=13\sin3x$$
Equate cos terms
$$-\omega^2 P\cos\omega x-5\omega Q\cos\omega x+6P\cos\omega x=0$$
Substitute $\omega=3$ and divide by $\cos3x$
$$-9P-15Q+6P=0$$
Simplify
\begin{equation}\label{eq:cos}
-3P-15Q=0
\end{equation}
Equate $\sin$ terms
$$-\omega^2Q\sin\omega x+5\omega P\sin\omega x+6Q\sin\omega x=13\sin3x$$
Substitute $\omega=3$ and divide by $\sin3x$
$$-9Q+15P+6Q=13$$
Simplify
\begin{equation}\label{eq:sin}
15P-3Q=13
\end{equation}
Multiply \eqref{eq:cos} by 5
$$-15P-75Q=0$$
Add the multiplied \eqref{eq:cos} and \eqref{eq:sin}
$$-78Q=13$$
Simplify
$$Q=-\frac{1}{6}$$
Substitute to find P
$$15\times\frac{1}{6}=-3P \quad P=\frac{5}{6}$$
Substitute and add to the CF to find the PI
$$y=Ae^{3x}+Be^{2x}+\frac{5}{6}\cos3x-\frac{1}{6}\sin3x$$
\newpage
\section{Clash of terms between CF and PI}
\textit{Solve:}
$$\frac{d^2y}{dx^2}-5\frac{dy}{dx}+6y=e^{2x}$$
Find the Complementary function
$$m^2-5m+6=0$$
$$CF: \ y=Ae^{3x}+Be^{2x}$$
Here there will be a clash of terms between the CF and the PI so a different PI must be used, this will be given to you.
$$\textrm{Use PI} \ y=\lambda xe^{2x}$$
Differentiate twice
$$\frac{dy}{dx}=\lambda e^{2x}+2\lambda xe^{2x}$$
$$\frac{d^2y}{dx^2}=2\lambda e^{2x}+2\lambda e^{2x}+4\lambda xe^{2x}$$
Substitute 
$$2\lambda e^{2x}+2\lambda e^{2x}+4\lambda xe^{2x}-5(\lambda e^{2x}+2\lambda xe^{2x})+6\lambda xe^{2x}=e^{2x}$$
$$\lambda=-1$$
Substitute
$$PI: \ y=-xe^{2x}$$
\section{Applications of boundary conditions}
If DE is in $\frac{d^2y}{dx^2}$ form then the numerical values for x,y and $\frac{dy}{dx}$ will be given for a value of x.\\
\\
If (as is common in exams) DE is in $\frac{d^2x}{dt^2}$ form then numerical values of x,t and $\frac{dx}{dt}$ will be given for a value of x.\\
\\
Used to find A and B in the CF. This gives a particular solution.
\subsection{Example}
\textit{When y=1 x=0 $\frac{dy}{dx}=0$}
\\
$$\frac{d^2y}{dx^2}+5\frac{dy}{dx}+6y=12e^x$$
Find the complementary function
$$y=Ae^{-3x}+Be^{-2x}$$
Find the general form of the particular integral
$$PI: \ \lambda e^x$$
Differentiate the particular integral twice
$$\frac{dy}{dx}=\lambda e^x$$
$$\frac{d^2y}{dx^2}=\lambda e^x$$
Substitute into the initial formula
$$\lambda e^5+5\lambda e^x+6\lambda e^x=12e^x$$
Simplify to find lambda
$$12\lambda e^x=12e^x \quad	\lambda=1$$
Add this to the complementary function to find the general solution
$$GS:\ Ae^{-3x}+Be^{-2x}+e^x$$
Substitute y=1 and x=0 into the general solution
$$1=A+B+1$$
Simplify
$$A+B=0$$
Substitute $\frac{dy}{dx}=0$ and $x=0$ into the differentiated form of the general solution\\
Differentiate the general solution
$$\frac{dy}{dx}=-3Ae^{-3x}-2Be^{-2x}+e^x$$
Substitute values
$$0=-3A-2B+1$$
Solve simultaneous equations
$$A=1 \quad B=-1$$
Substitute into the general solution to find the particular solution
$$PS: \ y=e^{-3x}-e^{-2x}+e^x$$
\newpage
\section{Substitution}
\setcounter{equation}{0}
\subsection{Example 1}
\textit{Show that the substitution $x=e^u$ transforms}
\begin{equation}\label{eq:1}
x^2\frac{d^2y}{dx^2}+x\frac{dy}{dx}+y=0
\end{equation}
to
\begin{equation}\label{eq:2}
\frac{d^2y}{dx^2}+y=0
\end{equation}
\\
Find $\frac{dx}{du}$
$$\textrm{As} \ x=e^u, \ \frac{dx}{du}=e^u=x$$
Find $\frac{dy}{du}$ using the chain rule
\begin{equation}\label{eq:3}
\frac{dy}{du}=\frac{dy}{dx}\times\frac{dx}{du}=e^u\frac{dy}{dx}=x\frac{dy}{dx}
\end{equation}
Find a simplification of $\frac{d^2y}{du^2}$
$$\frac{d^2y}{du^2}=\frac{d}{du}\Bigg(\frac{dy}{du}\Bigg)$$
Simplify using known expression for $\frac{dy}{du}$
$$\frac{d^2y}{dx^2}=\frac{d}{du}\Bigg(e^u\frac{dy}{dx}\Bigg)$$
Apply the product rule,using the chain rule to differentiate $\frac{dy}{dx}$ with respect to u
$$\frac{d^2y}{dx^2}=e^u\frac{dy}{dx}+e^u\frac{d^2y}{dx^2}\frac{dx}{du}$$
Replace $e^u\frac{dy}{dx}$ with $\frac{dy}{du}$ and $e^u$ and $\frac{dx}{du}$ each with x
$$\frac{d^2y}{dx^2}=\frac{dy}{du}+x^2\frac{d^2y}{dx^2}$$
Rearrange to find $x^2\frac{d^2y}{dx^2}$
\begin{equation}\label{eq:4}
x^2\frac{d^2y}{dx^2}=\frac{d^2y}{du^2}-\frac{dy}{du}
\end{equation}
Substitute the results from \eqref{eq:3} and \eqref{eq:4} into \eqref{eq:1}.
$$\frac{d^2y}{dx^2}-\frac{dy}{du}+\frac{dy}{du}+y=0$$
Simplify to find answer
$$\frac{d^2y}{dx^2}+y=0$$
\textit{Solve this to find the general solution}
$$m^2+1=0$$
$$m=\pm i$$
Substitute into the general form
$$y=A\cos u+B\sin u$$
Substitute $u=\ln x$ to give GS in terms of x
$$y=A\cos(\ln(x))+B\sin(\ln(x))$$
\end{document}