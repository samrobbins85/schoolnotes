\documentclass{article}[18pt]
\usepackage[utf8]{inputenc}
\usepackage[margin=0.7in]{geometry}
\usepackage{parselines} 
\usepackage{amsmath}
\usepackage{titlesec}
\usepackage{pgfplots}
\usepackage{graphicx}
\usepackage[english]{babel}
\usepackage{fancyhdr}
\usepackage{gensymb}
\usepackage{relsize}
\pgfplotsset{width=10cm,compat=1.9}
\usepackage[super]{nth}
\titlespacing\section{0pt}{14pt plus 4pt minus 2pt}{0pt plus 2pt minus 2pt}
\newlength\tindent
\setlength{\tindent}{\parindent}
\setlength{\parindent}{0pt}
\renewcommand{\indent}{\hspace*{\tindent}}

\pagestyle{fancy}
\fancyhf{}
\rhead{Sam Robbins 13SE}
\lhead{A Level Maths - FP2}
\rfoot{Page \thepage}


\begin{document}
\begin{center}
\underline{\huge Second Order Differential Equations}
\end{center}
In FP2 we are interested in solving 2nd ODEs of the form:
$$a\frac{d^2y}{dx^2}+b\frac{dy}{dx}+cy=f(x) \quad \quad \textrm{a,b,c are constants}$$
\\
We consider three distinct cases:\\
$b^2>4ac\quad$  (Two real solutions)\\
$b^2=4ac\quad$  (One repeated solution)\\
$b^2<4ac\quad$  (Two complex solutions)\\
\\
To solve \nth{2} ODEs of this form we first consider solutions to:
$$a\frac{d^2y}{dx^2}+b\frac{dy}{dx}+cy=0$$
The process of solving a \nth{2} ODE starts with a general solution to a \nth{1} ODE of form:
$$b\frac{dy}{dx}+cy=0$$
$$\mathlarger{\int}\frac{1}{b} \ dy=\mathlarger{\int}\frac{1}{-cy} \ dy$$
$$b\ln(y)=-cx+k$$
$$y=Ae^{-\frac{c}{b}x}$$
$$y=Ae^{mx}$$
\textcolor{red}{This was suggested to be a solution to the \nth{2} ODE as well}\\
\\
We take $y=e^{mx}$ as a starting point for finding general solutions to:
$$(1) \quad a\frac{d^2y}{dx^2}+b\frac{dy}{dx}+cy=0$$\\
If $y=e^{mx}$ is a solution to (1):
$$\frac{dy}{dx}=me^{mx}$$
$$\frac{d^2y}{dx^2}=m^2e^{mx}$$
Then substitute this into (1)
$$am^2e^{mx}+bme^{mx}+ce^{mx}=0$$
Factor out $e^{mx}$
$$e^{mx}(am^2+bm+c)$$
As $e^x$ must be greater than zero $am^2+bm+c=0$\\
This is a solvable quadratic called the \textbf{Auxiliary equation}
\newpage
\section{Two real roots $\mathbf{b^2>4ac}$}
$$(1) \quad a\frac{d^2y}{dx^2}+b\frac{dy}{dx}+cy=0$$\\
The general solution to (1) is in the form:
$$\mathlarger{y=Ae^{\alpha x}+Be^{\beta x}}$$
Where A and B are constants and $\alpha$ and $\beta$ are the roots to the AE
\subsection{Example}
$$(1) \quad 2\frac{d^2y}{dx^2}+5\frac{dy}{dx}+3y=0$$
Find the auxiliary equation
$$2m^2e^{mx}+5me^{mx}+3e^{mx}=0$$
$$e^{mx}(2m^2+5m+3=0$$
$m=-\dfrac{3}{2}\quad m=-1$
$$\textrm{General solution:} y=Ae^{\alpha x}+Be^{\beta x}$$
$$GS=Ae^{-\frac{3}{2}x}+Be^{-x}$$
\section{1 Real, Repeated root $\mathbf{b^2=4ac}$}
$$\textrm{General solution}: \mathlarger{(A+bx)e^{\alpha x}}$$
A and B are constants and $\alpha$ is the root of the AE
\subsection{Example}
\textit{Find the general solution of:}
$$\frac{d^2y}{dx^2}+8\frac{dy}{dx}+16y=0$$
Find the auxiliary equation
$$e^{mx}(m^2+8m+16)=0$$
Find the solution
$$m=-4$$
Substitute into the general solution formula
$$y=(A+Bx)e^{-4x}$$
\section{Imaginary only roots $\mathbf{b^2<4ac}$}
This is when the AI has roots of form $\pm \alpha i$
$$\textrm{General solution:} \quad \mathlarger{y=A\cos(\alpha x)+B\sin(\alpha x)}$$
\newpage
\section{Complex roots $\mathbf{b^2<4ac}$}
This is used when the root is in the form $\beta\pm\alpha i$
$$\textrm{General solution:} \quad \mathlarger{y=e^{\beta x}(A\cos(\alpha x)+B\sin(\alpha x))}$$
\subsection{Example}
\textit{Find the general solution of:}
$$\frac{d^2y}{dx^2}-6\frac{dy}{dx}+34y=0$$
Find the auxiliary equation
$$m^2-6m+34=0$$
Solve to find roots
$$\textrm{Roots}=\frac{6\pm\sqrt{36-4\times1\times34}}{2}=\frac{6\pm10i}{2}=3\pm5i$$
Substitute into general solution formula
$$\mathlarger{y=e^{3x}(A\cos(5x)+B\sin(5x))}$$
\end{document}