\documentclass{article}[18pt]
\usepackage{/home/sam/Documents/School_Notes/format}
\lhead{A Level Maths - FP2}

%File specific Preamble
\usepackage{cancel} %Strikethrough fractions

\begin{document}
\begin{center}
\underline{\huge Series}
\end{center}
\section{The method of differences}
This method is used when large parts of a series cancel, allowing the series to be expressed in a simple form in terms of n.
\subsection{Example}
\textit{Express the below summation in terms of partial fractions}
$$\sum_{r=1}^n \ \frac{1}{r(r+1)}$$
Write using unknown numerators
$$\frac{1}{r(r+1)}=\frac{A}{r}+\frac{B}{r+1}$$
Solve
$$A(r+1)+B(r)=1$$
$$r=0 \ A=1$$
$$r=-1 \ B=-1$$
Substitute
$$\sum_{r=1}^n \ \frac{1}{r(r+1)}=\sum_{r=1}^n \ \frac{1}{r}-\frac{1}{r+1}$$
\textit{Find a formula for this without summation}\\
\\
Write down values for the start and end of the summation, crossing out cancelling entries:\\
\\
\begin{tabular}{l l l l}
=&$\frac{1}{1}$&-&\cancel{$\frac{1}{2}$}\\
\\
+&\cancel{$\frac{1}{2}$}&-&\cancel{$\frac{1}{3}$}\\
\\
+&\cancel{$\frac{1}{3}$}&-&\cancel{$\frac{1}{4}$}\\
\\
+&...&-&...\\
\\
+&\cancel{$\frac{1}{n-1}$}&-&\cancel{$\frac{1}{n}$}\\
\\
+&\cancel{$\frac{1}{n}$}&-&$\frac{1}{n+1}$
\\
\end{tabular}\\
\\
\\
Write down remaining values as the answer:
$$1-\frac{1}{n+1}$$
\section{Summations not starting at 1}
Remember:
$$\mathlarger{\sum_{r=21}^n=\sum_{r=1}^n-\sum_{r=1}^{20}}$$
\end{document}