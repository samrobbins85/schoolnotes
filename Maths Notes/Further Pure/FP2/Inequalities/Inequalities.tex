\documentclass{article}[18pt]
\usepackage{../../../../format}
\lhead{A Level Maths - FP2}

%File Specific Preamble
\usepgfplotslibrary{fillbetween}

\begin{document}
\begin{center}
\underline{\huge Inequalities}
\end{center}
\section{Solving inequalities}
We can build upon our previous algebraic skills in order to solve more complex inequalities\\
Remember:
\begin{itemize}
\item Don't multiply anything that could be negative - use "squared" things
\item Find the critical values (f(x)=0)
\item Sketch the graph to solve
\item Things only need squaring when multiplying, not subtracting or adding
\end{itemize}
\subsection{Examples}
\subsubsection{Example 1}
$$2x^2<x+3$$
Move all terms to one side
$$2x^2-x-3<0$$
Factorise
$$(2x-3)(x+1)<0$$
Solve to find critical values
$$\textrm{CVs:} \ x=\frac{3}{2},-1$$
Draw graph to find inequality\\
\\
\\
\begin{tikzpicture}[scale=0.5]
\begin{axis}[axis lines=middle,xtick={
        -1,1.5
    },
    xticklabels={
        -1,$\dfrac{3}{2}$
    },ytick style={draw=none},ytick=\empty
]
\pgfplotsset{every tick label/.append style={font=\Large,color=red}}

\addplot[name path=A,color=black,domain=-1.5:2]{2*x^2-x-3};
\end{axis}
\end{tikzpicture}\\
Write inequality for when the graph is below the x axis
$$-1<x<\frac{3}{2}$$
\newpage
\subsubsection{Example 2}
$$\frac{x}{x+1}<\frac{2}{x+2}$$
Multiply both sides by $(x+1)^2(x+3)^2$
$$x(x+1)(x+3)^2<2(x+3)(x+1)^2$$
Put all terms on one side
$$x(x+1)(x+3)^2-2(x+3)(x+1)^2<0$$
Simplify
$$(x+1)(x+3)(x(x+3)-2(x+1))<0$$
$$(x+1)(x+3)(x+2)(x-1)<0$$
Plot graph\\
\begin{tikzpicture}[scale=0.5]
\begin{axis}[axis lines=middle,xtick={
        -3,-2,-1,1
    },
    xticklabels={
        -3,-2,-1,1
    },ytick style={draw=none},ytick=\empty,ymax=2,xmax=3,samples=100
]
\pgfplotsset{every tick label/.append style={font=\Large,color=red}}

\addplot[name path=A,color=black,domain=-4:4]{(x+1)*(x+3)*(x+2)*(x-1)};
\end{axis}
\end{tikzpicture}\\
\\
Write the inequality for when the graph is below the x axis
$$-3<x<-2, \ -1<x<1$$
\section{Modulus inequalities}
When doing inequalities using modulus signs, they can be solved by squaring each side and rearranging, but this will often result in difficult to solve equations.\\
It is often easier to find critical values by removing the modulus signs and setting them equal to each other to find one set, then , multiply one side by -1 to find the other set. 



\end{document}