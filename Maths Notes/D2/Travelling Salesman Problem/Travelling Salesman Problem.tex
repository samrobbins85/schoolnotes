\documentclass{article}[18pt]
\usepackage[utf8]{inputenc}
\usepackage[margin=0.7in]{geometry}
\usepackage{parselines} 
\usepackage{amsmath}
\usepackage{titlesec}
\usepackage{pgfplots}
\usepackage{graphicx}
\usepackage[english]{babel}
\usepackage{fancyhdr}
\usepackage{gensymb}
\usepackage{enumerate}
\usepackage{amssymb}
\pgfplotsset{width=10cm,compat=1.9}
\usepackage{tikz}
\usetikzlibrary{calc}
\titlespacing\section{0pt}{14pt plus 4pt minus 2pt}{0pt plus 2pt minus 2pt}
\newlength\tindent
\setlength{\tindent}{\parindent}
\setlength{\parindent}{0pt}
\renewcommand{\indent}{\hspace*{\tindent}}	
\newcommand{\cred}[1]{\color{red}#1}
\newcommand{\cblue}[1]{\color{blue}#1}
\newcommand{\cgreen}[1]{\color{green}#1}
\hyphenpenalty=10000
\pagestyle{fancy}
\fancyhf{}
\rhead{Sam Robbins 13SE}
\lhead{A Level Maths - D2}
\rfoot{Page \thepage}
\usepackage{float}

\newcommand{\tikzmark}[1]{\tikz[overlay,remember picture] \node (#1) {};}

\newcommand{\DrawVLine}[3][]{%
    \begin{tikzpicture}[overlay,remember picture]
        \draw [#1] ($(#2.north)$) -- ($(#3.north)$);
    \end{tikzpicture}%
}%
\newcommand{\DrawLine}[3][]{%
    \begin{tikzpicture}[overlay,remember picture]
        \draw [#1] ($(#2)+(-0.4,0.6ex)$) -- ($(#3)+(0.6,0.6ex)$);
    \end{tikzpicture}%
}%




\begin{document}
\begin{center}
\underline{\huge Travelling Salesman Problem}
\end{center}
\section{Definitions}
The travelling salesman problem looks for a \textbf{walk} that gives the minimum \textbf{tour}.\\
\\
\textbf{Walk} - A finite series of edges so that the end of one vertex is the start of the next\\
\\
\textbf{Tour} - A walk that visits every vertex and returns to the starting vertex
\\
\textbf{Add more information about upper and lower bounds here when you understand it more}
\section{The differences between classical and practical problems}
\textbf{Classical Problem} - Must visit each vertex \textbf{only once} before returning to the start.\\
\textbf{Practical Problem} - Must visit each vertex \textbf{at least once} before returning to the start
\section{Converting a network into a complete network of least distances}
If a network is converted into a complete network of least distances, the classical and practical problem are the same.\\
\\
To create a complete network of least distances, you must ensure the \textbf{triangle inequality} holds for all triangles in the network.\\
\\
\textbf{Triangle inequality:}\\
The longest side of any triangle $\leqslant$ The sum of the two shorter sides\\
\\
In a network where the triangle inequality does not hold, replace the longest arc with the sum of the two shorter ones.
\section{Using a Minimum Spanning tree to find the upper bound of the travelling salesman problem}
Method:
\begin{itemize}
\item Find the minimum spanning tree(Prim's or Kruskal's). This guarantees all vertexes are included.
\item Double the length of the minimum spanning tree as the route includes going there and back.
\item Find "short cuts"(using the non included arcs to bypass repeated edges.
\end{itemize}
This algorithm gives the initial upper bound
\end{document}
