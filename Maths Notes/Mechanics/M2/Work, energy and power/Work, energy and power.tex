\documentclass{article}[18pt]
\usepackage{/home/sam/Documents/School_Notes/format}
\lhead{A Level Maths - M2}

%File specific Preamble
\usetikzlibrary{positioning,scopes,decorations.text}

\begin{document}
\begin{center}
\underline{\huge Work, energy and power}
\end{center}
\begin{obeylines}
\section{Work}
$\mathbf{Work(J)=Force(N)\times Distance(m)}$
For work done against  gravity, use vertical distance.
\subsection{Work done against friction}
Work done against friction is:
$$WD=\mu R\times \textrm{Distance}$$
Where the distance is in the direction of motion
\section{Energy}
$\mathbf{E_K=\frac{1}{2}mv^2}$
$ $
$\mathbf{GPE=E_P=mgh}$
\section{Conservation of energy}
A particle's total energy is constant if it is subject only to gravity (i.e. smooth surfaces)
If there are frictional forces to consider then the loss of energy is the work done by friction.
\section{Power}
$\mathbf{Power=Force\times Velocity}$
Usually given in kW
\end{obeylines}
\newpage
\begin{center}
\underline{\huge Work, Energy and Power Example - Conservation of energy}
\end{center}
\textit{A block of mass 10 kg is pulled along a straight horizontal road by a constant horizontal force
of magnitude 70 N in the direction of the road. The block moves in a straight line passing
through two points A and B on the road, where AB = 50 m. The block is modelled as a particle and the road is modelled as a rough plane. The coefficient of friction between the block and the road is $\frac{4}{7}$}\\
\\
\textit{Calculate the work done against friction in moving the block from A to B. }\\
\\
Calculate the force of friction
$$F=\mu R=\frac{4}{7}\times10g=56N$$
Multiply force by distance to find work done
$$50\times56=2800J$$
\\
\textit{The block passes through A with a speed of $2ms^{-1}$.\\
Find the speed of the block at B.}
\\
\\
Subtract the work done against friction from the work done by the pulling force
$$70\times50-2800=700J$$
Write an expression for the change in kinetic energy
$$\frac{1}{2}\times10\times v^2-\frac{1}{2}\times10\times2^2$$
Set the excess energy from the work done equal to the change in kinetic energy and simplify
$$700=5v^2-20$$
$$v=12$$
\newpage
\begin{center}
\underline{\huge Work, Energy and Power Example - Power}
\end{center}
\textit{A lorry of mass 2000 kg is moving down a straight road inclined at angle $\alpha$ to the horizontal,where $\sin\alpha=\frac{1}{25}$. The resistance to motion is modelled as a constant force of magnitude 1600 N. The lorry is moving at a constant speed of $14ms^{-1}$.\\
Find, in kW, the rate at which the lorry's engine is working.}
\\
\\
Draw a diagram to represent the information
\def\iangle{35} % Angle of the inclined plane

\def\down{-90}
\def\arcr{0.5cm} % Radius of the arc used to indicate angles
\begin{tikzpicture}[
    force/.style={>=latex,draw=blue,fill=blue},
    calculatedforce/.style={>=latex,draw=red,fill=red},
    axis/.style={densely dashed,gray,font=\small},
    M/.style={rectangle,draw,fill=lightgray,minimum size=0.5cm,thin},
    m/.style={rectangle,draw=black,fill=lightgray,minimum size=0.3cm,thin},
    plane/.style={draw=black,fill=blue!10},
    string/.style={draw=red, thick},
    pulley/.style={thick},
    scale=2
]



    %% Free body diagram of M
    \begin{scope}[rotate=\iangle]
        \node[M,transform shape] (M) {};
        % Draw axes and help lines

        {[axis,-]
            \draw (M.center) -- (0,-1) node[right] {};

            % Indicate angle. The code is a bit awkward.

            \draw[solid,shorten >=0.5pt] (\down-\iangle:\arcr)
                arc(\down-\iangle:\down:\arcr);
            \node at (\down-0.5*\iangle:1.3*\arcr) {$\alpha$};
        }
        % Forces
        {[force,->]
            % Assuming that Mg = 1. The normal force will therefore be cos(alpha)
            \draw (M.south) -- ++(0,{cos(\iangle)*1.5}) node[above right] {$R$};
            \draw (M.west) -- ++(-1,0) node[left] {$T_r$};
            \draw (M.east) -- ++(1,0) node[above] {$1600$};   
            }
        {[calculatedforce,->]       
        
        \draw (0,-1) -- ++(-0.7,0) node[pos=0,right,sloped] {};
        }
        \draw[black, thick] (-3,-0.25) -- (2,-0.25);
		\node at (0,-1.13) (nodeA) {};
		\node at (-0.7,-1.13) (nodeB) {};
		\draw [decoration={text along path,text={{\footnotesize $2000g\sin\theta$}{}},text align={center}},decorate]  (nodeB) -- (nodeA);
    \end{scope}
    % Draw gravity force. The code is put outside the rotated
    % scope for simplicity. No need to do any angle calculations. 
    \draw[force,->] (M.center) -- ++(0,-1.5) node[below] {$2000g$};
    \draw[black, thick] (-2.31,-1.93) -- (1.76,-1.93);
    \draw[thick, -] (-1.8,-1.93) arc (0:36:0.5);
    \node[] at (-1.93,-1.8) {$\alpha$};
\\
;
\end{tikzpicture}
\\
\\
Equate forces
$$T_r+2000g\sin\alpha=1600$$
Solve
$$T_r=1600-2000g\times\frac{1}{25}=816N$$
Find power by multiplying force and velocity
$$P=Fv=816\times14=11.4kW$$
\end{document}