\documentclass{article}[18pt]
\usepackage[utf8]{inputenc}
\usepackage[margin=0.7in]{geometry}
\usepackage{parselines} 
\usepackage{amsmath}
\usepackage{titlesec}
\usepackage{pgfplots}
\usepackage{graphicx}
\usepackage[english]{babel}
\usepackage{fancyhdr}
\usepackage{gensymb}

\pgfplotsset{width=10cm,compat=1.9}

\titlespacing\section{0pt}{14pt plus 4pt minus 2pt}{0pt plus 2pt minus 2pt}
\newlength\tindent
\setlength{\tindent}{\parindent}
\setlength{\parindent}{0pt}
\renewcommand{\indent}{\hspace*{\tindent}}

\pagestyle{fancy}
\fancyhf{}
\rhead{Sam Robbins 13SE}
\lhead{A Level Maths - M2}
\rfoot{Page \thepage}


\begin{document}
\begin{center}
\underline{\huge Work, energy and power}
\end{center}
\begin{obeylines}
\section{Work}
$\mathbf{Work(J)=Force(N)\times Distance(m)}$
For work done against  gravity, use vertical distance.
\subsection{Work done against friction}
Work done against friction is:
$$WD=\mu R\times \textrm{Distance}$$
Where the distance is in the direction of motion
\section{Energy}
$\mathbf{E_K=\frac{1}{2}mv^2}$
$ $
$\mathbf{GPE=E_P=mgh}$
\section{Conservation of energy}
A particle's total energy is constant if it is subject only to gravity (i.e. smooth surfaces)
If there are frictional forces to consider then the loss of energy is the work done by friction.
\section{Power}
$\mathbf{Power=Force\times Velocity}$
Usually given in kW
\end{obeylines}
\end{document}