\documentclass{article}[18pt]
\usepackage[utf8]{inputenc}
\usepackage[margin=0.7in]{geometry}
\usepackage{parselines} 
\usepackage{amsmath}
\usepackage{titlesec}
\usepackage{pgfplots}
\usepackage{graphicx}
\usepackage[english]{babel}
\usepackage{fancyhdr}
\usepackage{gensymb}

\pgfplotsset{width=10cm,compat=1.9}

\titlespacing\section{0pt}{14pt plus 4pt minus 2pt}{0pt plus 2pt minus 2pt}
\newlength\tindent
\setlength{\tindent}{\parindent}
\setlength{\parindent}{0pt}
\renewcommand{\indent}{\hspace*{\tindent}}

\pagestyle{fancy}
\fancyhf{}
\rhead{Sam Robbins 13SE}
\lhead{A Level Maths - M2}
\rfoot{Page \thepage}


\begin{document}
\begin{center}
\underline{\huge Statics of rigid bodies}
\end{center}
\section{Equilibrium of rigid bodies}
When calculating moments, only use the component of the force acting perpendicular to the rod.\\
\\
A rigid body is in equilibrium if:
\begin{itemize}
\item The vector sum of the forces is zero
\item The sum of the moments around any point is zero
\end{itemize}

\end{document}