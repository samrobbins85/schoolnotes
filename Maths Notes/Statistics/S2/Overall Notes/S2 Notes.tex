\documentclass{article}[18pt]
\usepackage{../../../../format}
\lhead{A Level Maths - S2}


\begin{document}
\begin{center}
\underline{\huge S2 Notes}
\end{center}
\section{Continuous random variables}
\subsection{Probability Density Function}
\begin{itemize}
\item $f(x)\geqslant 0$ for all values of $x$, so that no probabilities are negative.
\item $\mathlarger{\int_{\infty}^{-\infty}} f(x) dx=1$  (The sum of all probabilities is 1)
\item $P(a<x<b)=\mathlarger{\int_{a}^{b}} f(x) dx$
\end{itemize}
\subsection{Median}
When finding the median of a CRV with multiple ranges, set F(x) to 0.5 and rearrange. If x lies outside the range, change to a different range until one works
\section{Continuous distributions}
\subsection{Continuity correction}
{\renewcommand{\arraystretch}{1.5}
\begin{tabular}{|c|c|}
\hline
Probability&Corrected Probability\\
\hline
$P(X=n)$&$P(n-0.5<X<n+0.5)$\\
\hline
$P(X>n)$&$P(X>n+0.5)$\\
\hline
$P(X\leqslant n)$ &$P(X<n+0.5)$\\
\hline
$P(X<n)$&$P(X<n-0.5)$\\
\hline
$P(X\geqslant n)$&$P(X>n-0.5)$\\
\hline
\end{tabular}}
\section{Discrete distributions}
\subsection{Samples}
When asked to list all the possible samples remember:
$$\textrm{Number of samples}=\textrm{Number of options}^\textrm{Sample size}$$
\section{Hypothesis tests}
Remember to split the significance level for a two tailed test
\subsection{Method}
\begin{enumerate}
\item Establish the null and alternative hypothesis ($H_0$ and $H_1$)
\item Define distribution under $H_0$
\item Decide on the significance level
\item Collect data, state the test statistic, X=
\item Calculate the probability of obtaining the test statistic or a more extreme result (same direction as $H_1$)
\item Compare this to the sig level as a decimal
\begin{itemize}
\item If \textbf{greater} than the sig level, it is a \textbf{non significant} result, it is not in the critical region and we \textbf{do not} reject $H_0$
\item If \textbf{less} than sig level, it is a \textbf{significant result}, it is in the critical region and we \textbf{reject} $H_0$ 
\end{itemize}
\item Interpret the results in terms of the original claim
\end{enumerate}
\end{document}