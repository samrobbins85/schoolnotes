\documentclass{article}[18pt]
\usepackage{../../../../format}
\lhead{A Level Maths - S2}


\begin{document}
\begin{center}
\underline{\huge Binomial and Poisson Distributions}
\end{center}
\begin{obeylines}
\section{The Binomial Distribution}
\subsection{Introduction to the binomial distribution}
The binomial distribution is a \textbf{discrete} distribution.
Conditions for a binomial distribution:
\begin{itemize}
\item There are a fixed number of trials, \textbf{n}
\item There are two outcomes (success and failure)
\item Each trial is independent
\item The probability of success is constant, \textbf{p}
\end{itemize}
Formula:
For $X~B(n,p)$

$P(X=r)=\binom{n}{r}p^r(1-p)^{n-r}$

Example:
If 25 dice are thrown, find the probability three sixes are obtained.
$X~B(25,\frac{1}{6})$
$P(X=3)=\binom{25}{3}(\frac{1}{6})^3(\frac{5}{6})^{22}=0.1929$
\subsection{Use of tables}
Tables give "less than or equal to" probabilities.

Example:
$X~B(5,0.35)$
$P(x\leq3)=0.9640$ (in the table where n=5, p=0.35 and x=3)
$P(x<4)=0.9640$ (the data is discrete)
$P(x=3)=P(x\leq3)-P(x\leq2)$
$P(x=3)=0.9640-0.7648$
$ $
$P(x\geq3)=1-P(X\leq2)$
$P(x\geq3)=1-0.7648$
$P(x\geq3)=0.2352$
\subsection{Dealing with $P>\frac{1}{2}$}
Example:
In the production of a car it is found that 85 \% are without defects.
The cars are produces in batches of 50
\textit{Find the probability there are at least 40 defect-free cars in a batch}
X=Number of cars without defects
$X~B(50,0.85)$
$P(X\geq40)$
Y=Number of cars with defects
$Y~B(50,0.15)$
$P(Y\leq10)=0.8801=P(x\geq40)$
\subsection{Mean and variance of the binomial distribution}
$E(X)=np$
$Var(x)=np(1-p)$
$\sigma(x)=\sqrt{Var(x)}$
Example 1:
$X~B(80,0.4)$ \textit{Find $E(x)$ and $\sigma$}
$E(x)=80\times0.4=32$
$\sigma=80\times0.4(1-0.4)=4.38$
$ $
\newpage
Example 2:
$E(x)=8$
$Var(x)=6.4$
\textit{Find n and p}
$np-np^2=6.4$
$8-np^2=6.4$
$np^2=1.6$
$ $
$p=\dfrac{np^2}{np}=\dfrac{1.6}{8}=0.2$
$ $
$n=\dfrac{np}{p}=\dfrac{8}{0.2}=40$
\section{The Poisson Distribution}
\subsection{Introduction to the Poisson Distribution}
The conditions for a poisson distribution are:
\begin{itemize}
\item Events occur at random
\item Events occur independently of each other
\item The average rate of occurrences remains constant
\item There is zero probability of simultaneous occurrences 
\end{itemize}
$P(x=r)=\dfrac{e^{-\lambda}\lambda^r}{r!}$
$\lambda$ represents the mean number of occurrences in the time period.
\textbf{Example}
$x\sim P_o(4)$
$P(x=2)=\dfrac{e^{-4}\times 4^2}{2!}=0.1465$
\subsection{Mean and variance of the poisson distribution}
If $X~P_o(\lambda)$ then $E(x)=\mu=\lambda$ and $Var(x)=\lambda$
\subsection{Using the Poisson as an approximation of the binomial}
You can use the Poisson distribution as an approximation if:
\begin{itemize}
\item P is small $(<0.1)$
\item n is large $(>50)$
\end{itemize}
$\lambda=np$
$X\sim B(n,p)\approx Y\sim P_o(np)$
\textbf{Example}
$X\sim B(250,0.01)\approx X\sim P_o(2.5)$









\end{obeylines}
\end{document}