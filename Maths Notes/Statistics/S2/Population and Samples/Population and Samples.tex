\documentclass{article}[18pt]
\usepackage{/home/sam/Documents/School_Notes/format}
\lhead{A Level Maths - S2}

\begin{document}
\begin{center}
\underline{\huge Population and Samples}
\end{center}
\begin{tabularx}{\textwidth}{|X|X|}
\hline
  \textbf{Term} & \textbf{Definition} \\
\hline
Population&A collection of individual people or items\\
\hline
Finite population&Countable\\
\hline
Infinite population&Impossible to count (different from infinite number, e.g. grains of sand)\\
\hline
Census&Information obtained from all members of the population\\
\hline
Advantages of a census&
\begin{itemize}
\item Every single member of the population is used
\item It is unbiased
\item It gives an accurate answer
\end{itemize}\\
\hline
Disadvantages of a census&
\begin{itemize}
\item Takes a long time
\item Costly
\item Difficult to ensure the whole population is surveyed
\end{itemize}\\
\hline
Sample&A subset of the population\\
\hline
Sampling units&Individual units of the population (e.g. people)\\
\hline
Sampling frame&Individually named or numbered sampling units to form a list or other representation of data\\
\hline
Advantages of sampling&
\begin{itemize}
\item If the population is large and well mixed the sample will be representative
\item Cheaper than a census
\item Advantageous when testing results in destruction
\item Data is more readily available
\end{itemize}\\
\hline
Disadvantages of sampling&
\begin{itemize}
\item Uncertainty as there is variation between samples
\item Could have bias
\end{itemize}\\
\hline
Bias&Anything which prevents a sample becoming truly representative\\
\hline
Statistic&A quantity calculated solely from the observations in a sample. It does not involve any unknown parameters\\
\hline
Simple random sampling&A sample taken so every possible sample has equal chance of being selected\\
\hline
\end{tabularx}

\end{document}