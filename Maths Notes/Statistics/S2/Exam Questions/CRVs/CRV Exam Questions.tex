\documentclass{article}[18pt]
\usepackage[utf8]{inputenc}
\usepackage[margin=0.7in]{geometry}
\usepackage{parselines} 
\usepackage{amsmath}
\usepackage{titlesec}
\usepackage{pgfplots}
\usepackage{graphicx}
\usepackage[english]{babel}
\usepackage{fancyhdr}
\usepackage{gensymb}
\usepackage{amssymb}

\pgfplotsset{width=10cm,compat=1.9}

\titlespacing\section{0pt}{14pt plus 4pt minus 2pt}{0pt plus 2pt minus 2pt}
\newlength\tindent
\setlength{\tindent}{\parindent}
\setlength{\parindent}{0pt}
\renewcommand{\indent}{\hspace*{\tindent}}

\pagestyle{fancy}
\fancyhf{}
\rhead{Sam Robbins 13SE}
\lhead{A Level Maths - S2}
\rfoot{Page \thepage}


\begin{document}
\begin{center}
\underline{\huge CRV Exam Questions}
\end{center}
\section{Averages}
\subsection{Mean}
Use the formula on the data sheet:
$$\mu=\int_a^bxf(x) dx$$
Where a is the start of the range and b is the end.\\
If the variable has multiple ranges add all the means together to find the total mean.
\subsection{Variance}
Use the formula on the data sheet:
$$Var(X)=\int_a^b x^2f(x) \ dx-\mu^2$$
Where a is the start of the range and b is the end.\\
Find the mean using the method above


\subsection{Median}
To find the median set F(x) to 0.5. To do this you may need to convert the P.D.F to the C.D.F.\\
If the C.D.F has different equations for different ranges substitute in for any range. If the result is outside the range try a different range.
\subsubsection{Example}
{\renewcommand{\arraystretch}{1.5}
$
  F(x)=\left\{
  \begin{array}{@{}ll@{}}
    0, & \ x<0\\
    \frac{1}{4}x^2, &  0\leqslant x \leqslant 1\\
    \frac{1}{20}x^4+\frac{1}{5}, &  1\leqslant x \leqslant 2\\
    1 & x>2
  \end{array}\right.
$}\\
Testing between 0 and 1:
$$\frac{1}{4}x^2=0.5$$
$$x=1.41$$
As x is greater than 1 move to the next range
$$\frac{1}{20}x^4+\frac{1}{5}=0.5$$
$$x=1.57$$
This is in the range so is correct
\subsection{Mode}
With difficult problems differentiate f(x) and set to zero. Most will be able to be solved with logic though
\section{Conversions between the P.D.F and C.D.F}
\subsection{P.D.F to C.D.F}
To convert the P.D.F to the C.D.F it must be integrated.\\
This integration is done as normal but needs no +C.\\
If there are multiple ranges add the maximum value of the previous range to find the C.D.F.
\subsubsection{Example}
{\renewcommand{\arraystretch}{1.5}
$
  f(x)=\left\{
  \begin{array}{@{}ll@{}}
    \frac{1}{2}x, & \ 0\leqslant x\leqslant1\\
    \frac{1}{5}x^3, &  1\leqslant x \leqslant 2\\
    0 & \textrm{Otherwise}
  \end{array}\right.
$}\\


\end{document}
