\documentclass{article}[18pt]
\usepackage[utf8]{inputenc}
\usepackage[margin=0.7in]{geometry}
\usepackage{parselines} 
\usepackage{amsmath}
\usepackage{titlesec}
\usepackage{pgfplots}
\usepackage{graphicx}
\usepackage[english]{babel}
\usepackage{fancyhdr}
\usepackage{gensymb}
\usepackage{amssymb}

\pgfplotsset{width=10cm,compat=1.9}

\titlespacing\section{0pt}{14pt plus 4pt minus 2pt}{0pt plus 2pt minus 2pt}
\newlength\tindent
\setlength{\tindent}{\parindent}
\setlength{\parindent}{0pt}
\renewcommand{\indent}{\hspace*{\tindent}}

\pagestyle{fancy}
\fancyhf{}
\rhead{Sam Robbins 13SE}
\lhead{A Level Maths - S2}
\rfoot{Page \thepage}


\begin{document}
\begin{center}
\underline{\huge CRV Exam Questions}
\end{center}
\section{Averages}
\subsection{Mean}
Use the formula on the data sheet:
$$\mu=\int_a^bxf(x) dx$$
Where a is the start of the range and b is the end.\\
If the variable has multiple ranges add all the means together to find the total mean.
\subsection{Variance}
Use the formula on the data sheet:
$$Var(X)=\int_a^b x^2f(x) \ dx-\mu^2$$
Where a is the start of the range and b is the end.\\
Find the mean using the method above


\subsection{Median}
To find the median set F(x) to 0.5. To do this you may need to convert the P.D.F to the C.D.F.\\
If the C.D.F has different equations for different ranges substitute in for any range. If the result is outside the range try a different range.
\subsubsection{Example}
{\renewcommand{\arraystretch}{1.5}
$
  F(x)=\left\{
  \begin{array}{@{}ll@{}}
    0, & \ x<0\\
    \frac{1}{4}x^2, &  0\leqslant x \leqslant 1\\
    \frac{1}{20}x^4+\frac{1}{5}, &  1\leqslant x \leqslant 2\\
    1 & x>2
  \end{array}\right.
$}\\
Testing between 0 and 1:
$$\frac{1}{4}x^2=0.5$$
$$x=1.41$$
As x is greater than 1 move to the next range
$$\frac{1}{20}x^4+\frac{1}{5}=0.5$$
$$x=1.57$$
This is in the range so is correct
\subsection{Mode}
With difficult problems differentiate f(x) and set to zero. Most will be able to be solved with logic though
\newpage
\section{Conversions between the P.D.F and C.D.F}
\subsection{P.D.F to C.D.F}
To convert the P.D.F to the C.D.F it must be integrated.\\
This integration is done as normal but needs no +C.\\
If there are multiple ranges add the maximum value of the previous range to find the C.D.F.
\subsubsection{Example}
{\renewcommand{\arraystretch}{1.5}
$
  f(x)=\left\{
  \begin{array}{@{}ll@{}}
    \frac{1}{2}x, & \ 0\leqslant x\leqslant1\\
    \frac{1}{5}x^3, &  1\leqslant x \leqslant 2\\
    0 & \textrm{Otherwise}
  \end{array}\right.
$}\\
\\
Start by integrating the first range
$$\int_0^x \frac{1}{2}x \ dx=\frac{1}{4}x^2$$
Integrate the second range, remembering to add the area under the first range.
$$\int_1^x \frac{1}{5}x^3 dx+\int_0^1\frac{1}{2}x \ dx=\frac{1}{20}x^4+\frac{1}{5}$$
Write this in function form\\
\\
{\renewcommand{\arraystretch}{1.5}
$
  F(x)=\left\{
  \begin{array}{@{}ll@{}}
  0 & x<0\\
    \frac{1}{4}x^2, & \ 0\leqslant x\leqslant1\\
    \frac{1}{20}x^4+\frac{1}{5}, &  1\leqslant x \leqslant 2\\
    1 & x>2
  \end{array}\right.
$}\\
\subsection{C.D.F to P.D.F}
{\renewcommand{\arraystretch}{1.5}
$
  F(x)=\left\{
  \begin{array}{@{}ll@{}}
  0 & x<2\\
    \frac{1}{20}(x^2-4), & \ 2\leqslant x\leqslant4\\
    \frac{1}{5}(2x-5), & \ 4< x\leqslant5\\
    1 & x>5\\
  \end{array}\right.
$}\\
\\
Differentiate $\frac{1}{20}(x^2-4)$
$$\frac{1}{10}x$$
Differentiate $\frac{1}{5}(2x-5)$
$$\frac{2}{5}$$
Write in function form\\
{\renewcommand{\arraystretch}{1.5}
$
  F(x)=\left\{
  \begin{array}{@{}ll@{}}
    \frac{1}{10}x, & \ 2\leqslant x\leqslant4\\
    \frac{2}{5}, & \ 4< x\leqslant5\\
    0 & \textrm{Otherwise}\\
  \end{array}\right.
$}\\
\newpage
\section{Finding unknown constants}
To find unknown constants set one of the formulas to a value it is known to be and rearrange.\\
\\
For example with a C.D.F find the uppermost value and set it equal to one to find any constants.
\section{Finding probabilities}
For questions where a probability must be found, either an inequality or specific value, use the P.D.F or the C.D.F.
\subsection{Specific value}
To find the probability of a specific value substitute that value into the P.D.F.
\subsection{Inequality}
To find the probability of an inequality use the C.D.F. The C.D.F gives a less than or equal probability, so to find greater than probabilities the probability found must be subtracted from one.





\end{document}
