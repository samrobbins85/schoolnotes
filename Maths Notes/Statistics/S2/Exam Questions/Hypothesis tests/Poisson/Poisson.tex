\documentclass{article}[18pt]
\usepackage{../../../../../../format}
\lhead{A Level Maths - S2}

\begin{document}
\begin{center}
\underline{\huge Hypothesis tests - Poisson - Exam Questions}
\end{center}
\section{One tailed test}
\textit{An effect of a certain disease is that a small number of the red blood cells are deformed. Emily
has this disease and the deformed blood cells occur randomly at a rate of 2.5 per ml of her
blood. Following a course of treatment, a random sample of 2 ml of Emily’s blood is found to
contain only 1 deformed red blood cell.\\
\\
Stating your hypotheses clearly and using a 5\% level of significance, test whether or not there
has been a decrease in the number of deformed red blood cells in Emily’s blood.}\\
\\
Write the null and alternate hypothesis, using information from the text
$$H_0:\ \lambda=5$$
$$H_1:\ \lambda<5$$
Write down the distribution the hypothesis test will be done on
$$X\sim P_o(5)$$
Write down a probability to test based on the data, in this case 1 one deformed red blood cell
$$P(X\leqslant1)$$
Find this value in the tables, or using an approximation if needed
$$P(X\leqslant1)=0.0404$$
Compare this to the significance level
$$0.0404<0.05$$
Write the conclusion, if in the critical region, reject $H_0$, if not accept it
\begin{center}
As $0.0404<0.05$ the result is significant, so reject $H_0$\\
There is evidence in a \textbf{decrease} in the mean \textbf{rate} of \textbf{deformed blood cells}
\end{center}
\section{Finding critical values}
\textit{A test statistic has a Poisson distribution with parameter $\lambda$.}\\
\textit{Given that:}
$$H_0: \ \lambda=9,\ H_1: \ \lambda\neq9$$ 
\textit{Find the critical region for the test statistic such that the probability in each tail is as close
as possible to 2.5\%.}\\
\\
Write down the distribution
$$X\sim P_o(9)$$
Find the lower value by looking on the tables at the column related to $\lambda$ and look for the closest value to the probability wanted (in this case 2.5\%)
$$P(X\leqslant3)=0.0212\approx0.025$$
To find the upper value look for the value closest to (1-wanted probability). Then add one to this value as:
\textcolor{red}{$$P(X\geqslant c)=1-P(X\leqslant c-1)$$}
$$P(X\leqslant15)=0.9780\approx0.975$$
$$P(X\geqslant16)=1-0.9780=0.022\approx0.025$$
Critical values are 3 and 16\\
\\
\textit{Write down the actual significance}
\\
Actual significance is the probability of a value being found in the critical region.
$$0.0212+0.022=0.0432$$
\newpage
\section{Two tailed test}
\textit{Bacteria are randomly distributed in a river at a rate of 5 per litre of water. A new factory opens
and a scientist claims it is polluting the river with bacteria. He takes a sample of 0.5 litres of
water from the river near the factory and finds that it contains 7 bacteria. Stating your
hypotheses clearly test, at the 5\% level of significance, the claim of the scientist.}\\
\\
Write down the null and alternative hypothesis
$$H_0: \ \lambda=2.5$$
$$H_1 \ \lambda\neq2.5$$
Write down the distribution to be tested
$$X\sim P_o(2.5)$$
Find a probability to test from the data
$$P(X\geqslant7)=1-P(X\leqslant6)$$
Look up the probability on the tables
$$1-P(X\leqslant6)=1-0.9858=0.0142$$
Compare with the significance level, remembering to split it
$$0.0142<0.025$$
Write the conclusion
\begin{center}
There is significant evidence that the factory is polluting the river with bacteria
\end{center}


\end{document}
