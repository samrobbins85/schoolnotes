\documentclass{article}[18pt]
\usepackage[utf8]{inputenc}
\usepackage[margin=0.7in]{geometry}
\usepackage{parselines} 
\usepackage{amsmath}
\usepackage{titlesec}
\usepackage{pgfplots}
\usepackage{graphicx}
\usepackage{tabularx}
\usepackage[english]{babel}
\usepackage{fancyhdr}
\usepackage{gensymb}
\usepackage{relsize}
\pgfplotsset{width=10cm,compat=1.9}
\usepackage{enumitem}
\titlespacing\section{0pt}{14pt plus 4pt minus 2pt}{0pt plus 2pt minus 2pt}
\newlength\tindent
\setlength{\tindent}{\parindent}
\setlength{\parindent}{0pt}
\renewcommand{\indent}{\hspace*{\tindent}}
\hyphenpenalty=10000
\pagestyle{fancy}
\fancyhf{}
\rhead{Sam Robbins 13SE}
\lhead{A Level Maths - S3}
\rfoot{Page \thepage}


\begin{document}
\begin{center}
\underline{\huge Sampling}
\end{center}
\begin{tabularx}{\textwidth}{|c|X|X|X|}
\hline
&What it is&Advantages&Disadvantages\\
\hline
Census&A collection of data from an entire population&Gives a completely accurate result&
\begin{itemize}[noitemsep,topsep=0pt,leftmargin=*]
\item Time consuming+Expensive
\item Can not be used when testing involves destruction
\item Large volume of data to process
\end{itemize}\\
\hline
Sample survey&A survey of a small sample of the population&
\begin{itemize}[noitemsep,topsep=0pt,leftmargin=*]
\item Cheaper
\item Quicker
\item Easier to process
\end{itemize}&
\begin{itemize}[noitemsep,topsep=0pt,leftmargin=*]
\item Data may not be accurate
\item Data may not be large enough to represent small sub groups
\end{itemize}\\
\hline
Random Sampling&Each thing has an equal chance of being selected&
\begin{itemize}[noitemsep,topsep=0pt,leftmargin=*]
\item Numbers truly random and free from bias
\item Easy to use
\item Each number has a known equal chance of being selected
\end{itemize}&
\begin{itemize}[noitemsep,topsep=0pt,leftmargin=*]
\item Not suitable when population is large
\end{itemize}\\
\hline
Lottery sampling&Each element of a population put on a ticket. Tickets drawn randomly from container(without displacement)&
\begin{itemize}[noitemsep,topsep=0pt,leftmargin=*]
\item Tickets are drawn at random
\item It is easy to use
\item Each ticket has a known chance of selection
\end{itemize}&
\begin{itemize}[noitemsep,topsep=0pt,leftmargin=*]
\item It is not suitable when the population is large
\item A sampling frame is needed
\end{itemize}\\
\hline
Systematic sampling&Required elements are chosen at regular intervals in an ordered list&
\begin{itemize}[noitemsep,topsep=0pt,leftmargin=*]
\item Simple to use
\item Suitable for large samples
\end{itemize}&
\begin{itemize}[noitemsep,topsep=0pt,leftmargin=*]
\item Only random if ordered list is truly random
\item Can introduce bias
\end{itemize}\\
\hline
Stratified sampling&Population is divided into groups and a simple random sample is carried out in each group&
\begin{itemize}[noitemsep,topsep=0pt,leftmargin=*]
\item It can give more accurate estimates than simple random sampling where clear strata are present
\item Reflects the population structure
\end{itemize}&
\begin{itemize}[noitemsep,topsep=0pt,leftmargin=*]
\item Within the strata, the problems are than same as for any simple random sample
\item If the strata are not clearly defined they may overlap
\end{itemize}\\
\hline
Quota sampling&The population is divided into groups by gender etc. A quota of people in each group is set to try and reflect the group's proportion in the whole population&
\begin{itemize}[noitemsep,topsep=0pt,leftmargin=*]
\item Enables fieldwork to be done quickly because a small sample size is taken.
\item Costs kept to a minimum
\item Administering test is easy
\end{itemize}&
\begin{itemize}[noitemsep,topsep=0pt,leftmargin=*]
\item Not possible to estimate the sampling errors
\item Interviewers may not be able to judge characteristics easily
\item Non responses are not recorded
\item Can introduce interview bias
\end{itemize}\\
\hline
\end{tabularx}
\newpage
\begin{center}
\underline{\huge Types of data}
\end{center}
\begin{tabularx}{\textwidth}{|c|X|X|}
\hline
Type of data&Advantages&Disadvantages\\
\hline
Primary&
\begin{itemize}[noitemsep,topsep=0pt,leftmargin=*]
\item The collection method is known
\item The accuracy is known
\item The exact data needed is collected
\end{itemize}&
\begin{itemize}[noitemsep,topsep=0pt,leftmargin=*]
\item It is costly in time and effort
\end{itemize}\\
\hline
Secondary&
\begin{itemize}[noitemsep,topsep=0pt,leftmargin=*]
\item They are cheap to obtain
\item A large quantity of data is available
\item Much of the data has been collected for years and can be used to plot trends
\end{itemize}&
\begin{itemize}[noitemsep,topsep=0pt,leftmargin=*]
\item Bias is not always recognised
\item It can be in a form that is difficult to deal with
\end{itemize}\\
\hline





\end{tabularx}



\end{document}