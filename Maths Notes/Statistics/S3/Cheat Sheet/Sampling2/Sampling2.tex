\documentclass{article}[5pt]
\usepackage[utf8]{inputenc}
\usepackage[lmargin=0.1in,rmargin=0.1in,bmargin=0.1in,tmargin=0.6in,landscape,a4paper]{geometry}
\usepackage{parselines} 
\usepackage{amsmath}
\usepackage{titlesec}
\usepackage{pgfplots}
\usepackage{graphicx}
\usepackage[english]{babel}
\usepackage{fancyhdr}
\usepackage{gensymb}
\usepackage{pbox}
\pgfplotsset{width=10cm,compat=1.9}
\usepackage{mathtools}
\titlespacing\section{0pt}{14pt plus 4pt minus 2pt}{0pt plus 2pt minus 2pt}
\newlength\tindent
\setlength{\tindent}{\parindent}
\setlength{\parindent}{0pt}
\renewcommand{\indent}{\hspace*{\tindent}}
\usepackage{relsize}
\pagestyle{fancy}
\fancyhf{}
\rhead{Sam Robbins 13SE}
\lhead{A Level Maths - S3}
\newcommand{\tabitem}{~~\llap{\textbullet}~~}
\usepackage{tabularx}
%File specific Preamble
\usepackage{enumitem} %Helps with lists in tables
\setitemize{noitemsep,topsep=-5pt,leftmargin=*}%Compress list
\usepackage[none]{hyphenat}
\begin{document}
\setcounter{section}{2}
\setcounter{subsection}{1}
\subsection{Sampling}
\begin{tabularx}{\textwidth}{|c|X|X|X|X|}
\hline
&What it is&When to use&Advantages&Disadvantages\\
\hline
Census&A collection of data from an entire population&Gives a completely accurate result&
\begin{itemize}
\item Small population
\item Easy to collect data
\item Large variation of opinion
\end{itemize}&

\begin{itemize}
\item Time consuming+Expensive
\item Can not be used when testing involves destruction
\item Large volume of data to process
\end{itemize}\\
\hline


Random Sampling
&Each thing has an equal chance of being selected&

\begin{itemize}
\item Large population
\item Have a sampling frame
\end{itemize}&

\begin{itemize}
\item Numbers truly random and free from bias
\item Easy to use
\item Each number has a known equal chance of being selected
\end{itemize}&
\begin{itemize}
\item Needs a sampling frame
\end{itemize}\\
\hline


Systematic sampling
&Required elements are chosen at regular intervals in an ordered list&
\begin{itemize}
\item Time constraint
\end{itemize}&

\begin{itemize}
\item Simple to use
\item Suitable for large samples
\end{itemize}&
\begin{itemize}
\item Only random if ordered list is truly random
\item Can introduce bias
\end{itemize}\\
\hline


Stratified sampling
&Population is divided into groups and a simple random sample is carried out in each group&

\begin{itemize}
\item More accurate when strata are present
\item Reflects population structure
\end{itemize}&

\begin{itemize}
\item It can give more accurate estimates than simple random sampling where clear strata are present
\item Reflects the population structure
\end{itemize}&
\begin{itemize}
\item Within the strata, the problems are than same as for any simple random sample
\item If the strata are not clearly defined they may overlap
\end{itemize}\\
\hline

Quota sampling
&The population is divided into groups by gender etc. A quota of people in each group is set to try and reflect the group's proportion in the whole population&
\begin{itemize}
\item There is no sampling frame
\end{itemize}&

\begin{itemize}
\item Enables fieldwork to be done quickly because a small sample size is taken.
\item Costs kept to a minimum
\item Administering test is easy
\end{itemize}&
\begin{itemize}
\item Not possible to estimate the sampling errors
\item Interviewers may not be able to judge characteristics easily
\item Non responses are not recorded
\item Can introduce interview bias
\end{itemize}\\
\hline
\end{tabularx}



\end{document}
