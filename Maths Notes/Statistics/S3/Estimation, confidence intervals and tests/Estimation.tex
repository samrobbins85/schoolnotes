\documentclass{article}[18pt]
\usepackage{../../../../format}
\lhead{A Level Maths - S3}

%File specific Preamble
%Normal Distribution
\pgfmathdeclarefunction{gauss}{2}{%
  \pgfmathparse{1/(#2*sqrt(2*pi))*exp(-((x-#1)^2)/(2*#2^2))}%
}

\begin{document}
\begin{center}
\underline{\huge Estimation, confidence intervals and tests}
\end{center}
\section{Statistics}
A statistic is defined as follows:\\
If $X_1,X_2,X_3,...,X_n$ is a random sample of size n from some population, then a statistic T is a random variable consisting of the $X_i$ than involves no other quantities.\\
\\
Since it is possible to repeat the process of taking a sample, if all samples are taken, they will form a probability distribution called the \textbf{sampling distribution of T}. This will usually depend on the distribution of the population X.\\
\\
A random sample is defined as follows:\\
A random sample of size n consists of the observations $X_1,X_2,X_3,...,X_n$ from a population where the $X_i$
\begin{itemize}
\item Are independent random variables
\item Have the same distribution as the population
\end{itemize}


\section{Estimators}
A statistic that is used to estimate a population parameter is an \textbf{estimator}.\\
A particular individual value is an \textbf{estimate}.\\
\textbf{Bias} is how far the estimator is from the true value\\
\\
If a statistic T is used as an estimator for a population parameter $\theta$ then the bias is:
$$E(T)-\theta$$
If $E(T)=\theta$ then the statistic is unbiased\\
\\
{\large $\mathlarger{\hat{\theta}}$} represents an estimator of $\theta$, this is true for all population parameters, such as $\hat{\mu}$
\subsection{Proving $\mathbf{E(\overline{X})=\mu}$}
\textit{Prove that $\overline{X}$ is an unbiased estimator for $\mu$ when the population is normally distributed}\\
\\
A random sample $X_1,X_2,X_3...X_N$ is taken for a population with $X\sim N(\mu,\sigma^2)$
$$\overline{X}=\frac{1}{n}\sum X$$
$$E(\overline{X})=E\Big(\frac{1}{n}\sum X\Big)$$
$$E(\overline{X})=\frac{1}{n}\big(E(X_1)+E(X_2)+E(X_3)...+E(X_N)\big)$$
$$E(\overline{X})=\frac{1}{n}(\mu+\mu+\mu...+\mu)$$
$$E(\overline{X})=\frac{1}{n}(n\mu)$$
$$E(\overline{X})=\mu$$
\subsection{Variance}
$$Var(\overline{X})=\frac{\sigma^2}{\textrm{Sample size}}$$
\\
We use $S^2$ as an estimator for $\sigma^2$
$$S^2=\frac{1}{n-1}\Big(\Sigma X^2-n\overline{X}^2\Big)$$
\\
In formula book
$$S^2=\frac{\Sigma(X_i-\overline{X})^2}{n-1}$$
Replace the top of the fraction with $S_{xx}=\Sigma x^2_i-\frac{(\Sigma x_i)^2}{n}$
\begin{center}
{\huge$S^2=\frac{\Sigma x_i^2-\frac{(\Sigma x_i)^2}{n}}{n-1}$}
\end{center}
\newpage
\subsection{Uniform Distributions}
\textit{Random variable X is continuously uniform $[0,\alpha]$. A sample $X_1,X_2..X_N$ is taken}\\
\textit{Show that $\overline{X}$ is a biased estimate and state the bias}
$$\overline{X}=\frac{0+\alpha}{2}=\frac{\alpha}{2}$$
$$E(\overline{X})=E(\frac{\alpha}{2})=\frac{1}{2}E(\alpha)=\frac{\alpha}{2}$$
$$\textrm{Bias}=\frac{\alpha}{2}-\alpha=-\frac{\alpha}{2}$$
\newpage
\section{Standard error}
$$\textrm{Standard error}=\frac{\sigma}{\sqrt{n}} \ \textrm{or} \ \frac{s}{\sqrt{n}}$$
\section{Central limit theorem}
The central limit theorem (C.L.T.) states that if $X_1,X_2,X_3...$ is a random sample, from any distribution with mean $\mu$ and variance $\sigma^2$. Then the sample means, $\overline{X}$ are distributed with a normal distribution.
$$\overline{X}\sim N(\mu,\frac{\sigma^2}{n})$$ 
For this, the sample size must be greater than 50
\section{Confidence intervals}
A 95\% confidence interval is two values in which there is a probability of 0.95 that it will contain the population mean.\\
\begin{tikzpicture}
\begin{axis}[
  no markers, domain=0:8, samples=100,
  every axis y label/.style={at=(current axis.above origin),anchor=south},
  every axis x label/.style={at=(current axis.right of origin),anchor=west},
  height=5cm, width=12cm,
  xtick={2,6},xticklabels={0.025,0.025}, ytick=\empty,
  enlargelimits=false, clip=false,   ]
  \addplot [fill=cyan!20, draw=none, domain=0:2] {gauss(4,1)} \closedcycle;
  \addplot [fill=cyan!20, draw=none, domain=6:8] {gauss(4,1)} \closedcycle;
  \addplot [very thick,cyan!50!black] {gauss(4,1)};
  \node[] at (axis cs: 4,0.1) {95\%};
\end{axis}
\end{tikzpicture}
\\
Look up the value of one of the tails on the percentage points table and substitute into the formula.\\
For example:\\
At a 95\% confidence interval each of the tails is 0.025 and the corresponding z value is 1.96.\\
$$\overline{x}\pm1.96\times\frac{\sigma}{\sqrt{n}}$$
\section{Hypothesis testing}
The test statistic for the population mean $\mu$ is $Z=\dfrac{X-\mu}{\frac{\sigma}{\sqrt{n}}}$\\
\\
A hypothesis test can be done quicker by finding the critical values from the percentage points table.
\subsection{Example}
$\mu=0.58 \ \sigma=0.015$\\
Measured\\
$n=50 \ \mu=0.577 \ \sigma=0.015$\\
\\
$H_0: \mu=0.58$\\
$H_1: \mu\neq0.58$\\
$$\overline{D}\sim N(0.58,\frac{0.015^2}{50})$$
$$Z=\frac{0.577-0.58}{\sqrt{\frac{0.015^2}{50}}}=-1.41$$
At 0.5\%  per level Z=2.5758
$$-1.41>-2.5758$$
Not in critical region so not significant, no evidence to reject $H_0$, the diameter of the bolts is not changed.
\newpage
\subsection{Hypothesis test for the difference between means}
If $\overline{X}\sim N\Big(\mu_x,\frac{\sigma^2_x}{n}\Big)$ and $\overline{Y}\sim N\Big(\mu_y,\frac{\sigma^2_y}{n}\Big)$
$$\overline{X}-\overline{Y}\sim N\Bigg(\mu_x-\mu_y,\frac{\sigma^2_x}{n}+\frac{\sigma^2_y}{n}\Bigg)$$
$$z=\frac{\overline{X}-\overline{Y}-(\mu_x-\mu_y)}{\sqrt{\frac{\sigma^2_x}{n}+\frac{\sigma^2_y}{n}}}$$
\subsubsection{Example}
\textit{The weight of boys and girls in a certain school are known to be normally distributed with standard deviations of 5kg and 8kg respectively. A random sample of 25 boys had a mean weight of 48kg and a random sample of 30 girls had a mean weight of 45kg.\\
Stating your hypothesis clearly test, at the 5\% level of significance, whether or not there is evidence that the mean weight of boys in the school is greater than the mean weight of girls.}\\
\\
Write down the two distributions
$$\overline{X_B}\sim N\Big(\mu_B,\frac{5^2}{25}\Big) \qquad \qquad \overline{X_G}\sim N\Big(\mu_G,\frac{8^2}{30}\Big)$$
State the two hypotheses
$$H_0: \ \mu_B=\mu_G$$
$$H_1: \ \mu_B>\mu_G$$
Find the $z$ value
$$z=\frac{48-45-(0)}{\sqrt{\frac{5^2}{25}+\frac{8^2}{30}}}=1.6947$$
Find the critical value from the percentage points table
$$C.V.=1.6449$$
Compare the critical value to the z value
$$1.6449<1.6947$$
Write conclusion
As z is not in the critical region there is not enough evidence to reject $H_0$, there is not enough evidence to suggest that the mean weight of boys is greater than the mean weight of girls.

\end{document}