\documentclass{article}[18pt]
\usepackage[utf8]{inputenc}
\usepackage[margin=0.7in]{geometry}
\usepackage{parselines} 
\usepackage{amsmath}
\usepackage{titlesec}
\usepackage{pgfplots}
\usepackage{graphicx}
\usepackage[english]{babel}
\usepackage{fancyhdr}
\usepackage{gensymb}


\pgfplotsset{width=10cm,compat=1.9}

\titlespacing\section{0pt}{14pt plus 4pt minus 2pt}{0pt plus 2pt minus 2pt}
\newlength\tindent
\setlength{\tindent}{\parindent}
\setlength{\parindent}{0pt}
\renewcommand{\indent}{\hspace*{\tindent}}

\pagestyle{fancy}
\fancyhf{}
\rhead{Sam Robbins 13SE}
\lhead{A Level Maths - M2}
\rfoot{Page \thepage}


\begin{document}
\begin{center}
\underline{\huge Centres of mass}
\end{center}
\begin{obeylines}
\section{Centre if mass of a discrete mass distribution}
$\bar{x}=\dfrac{\Sigma m_ix_i}{\Sigma m_i}$\\
$\bar{y}=\dfrac{\Sigma m_iy_i}{\Sigma m_i}$\\
\textbf{Example}\\
\end{obeylines}
\begin{tabular}{|c|c|c|c|}
 \hline
 Mass&$2$&$3$&$2$\\
 \hline
 $x$&2&3&-3\\
 \hline
 $y$&3&6&2\\
 \hline

\end{tabular}
\\
\\
\begin{obeylines}
$\bar{x}=\dfrac{2\times 2+3\times3+2\times-3}{2+3+2}=1$
$\bar{y}=\dfrac{2\times3+3\times6+2\times2}{2+3+2}=4$
\section{Uniform laminae}
For a triangular lamina the centre of mass is $\frac{2}{3}$ along the line from the vertex to the middle of the line opposite.
For a sector of a circle, radius r, where the angle at the centre is $2\alpha$ the centre of mass is $\dfrac{2r\sin\alpha}{3\alpha}$
\section{Rods}
In a circular arc, radius r, where the angle at the centre is $2\alpha$, the centre of mass is $\dfrac{r\sin\alpha}{\alpha}$ away from the centre.
\section{Equilibrium}
To avoid tipping, the lime of action of the weight must be within the side of the lamina in contact with the plane.
If a lamina is suspended from a fixed point, the centre of mass will be vertically below the point of suspension.
Assumptions made in equilibrium calculations:
\begin{itemize}
\item No friction at the point of suspension
\item The mass of each area is uniform
\item The mass is uniform at the join
\end{itemize}

Test



\end{obeylines}



\end{document}