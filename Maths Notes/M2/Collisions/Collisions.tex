\documentclass{article}[18pt]
\usepackage[utf8]{inputenc}
\usepackage[margin=0.7in]{geometry}
\usepackage{parselines} 
\usepackage{amsmath}
\usepackage{titlesec}
\usepackage{pgfplots}
\usepackage{graphicx}
\usepackage[english]{babel}
\usepackage{fancyhdr}
\usepackage{gensymb}

\pgfplotsset{width=10cm,compat=1.9}

\titlespacing\section{0pt}{14pt plus 4pt minus 2pt}{0pt plus 2pt minus 2pt}
\newlength\tindent
\setlength{\tindent}{\parindent}
\setlength{\parindent}{0pt}
\renewcommand{\indent}{\hspace*{\tindent}}

\pagestyle{fancy}
\fancyhf{}
\rhead{Sam Robbins 13SE}
\lhead{A Level Maths - M2}
\rfoot{Page \thepage}


\begin{document}
\begin{center}
\underline{\huge Collisions}
\end{center}
\begin{obeylines}
\section{Impulse and momentum}
Impulse=$mv-mu$=$Ft$\\
Total momentum before collision=total momentum after\\
\section{Coefficient of restitution}
This tells us how well something bounces, it is given the symbol \textbf{e}.\\
If $e=1$ the ball returns to it's original height\\
If $e=0$ the ball doesn't bounce\\
$e=\dfrac{\text{Speed of seperation}}{\text{Speed of approach}}$
\subsection{Alternate form of coefficient of restitution formula}
$mgh=\frac{1}{2}mv^2$\\
$v=\sqrt{2gh}$\\
$e=\dfrac{\sqrt{2gh_2}}{\sqrt{2gh_1}}$\\
$e=\dfrac{\sqrt{h_2}}{\sqrt{h_1}}$\\
$h_2$ - the height the ball bounces back to\\
$h_1$ - the height the ball is dropped from\\
\subsection{Calculations involving coefficient of restitution}
When doing calculations involving the coefficient of restitution both the calculation for CoR and conservation of momentum will be needed.
\textbf{Conservation of momentum:}
$m_1u_1+m_2u_2=m_1v_1+m_2v_2$\\
\textbf{Coefficient of restitution}
$e=\dfrac{v_1}{u_1}$
\end{obeylines}
\end{document}