\documentclass{article}[18pt]
\usepackage[utf8]{inputenc}
\usepackage[margin=0.7in]{geometry}
\usepackage{parselines} 
\usepackage{amsmath}
\usepackage{titlesec}
\usepackage{pgfplots}
\usepackage{graphicx}
\usepackage{tabularx}
\usepackage[english]{babel}
\usepackage{fancyhdr}
\usepackage{gensymb}
\usepackage{relsize}
\pgfplotsset{width=10cm,compat=1.9}

\titlespacing\section{0pt}{14pt plus 4pt minus 2pt}{0pt plus 2pt minus 2pt}
\newlength\tindent
\setlength{\tindent}{\parindent}
\setlength{\parindent}{0pt}
\renewcommand{\indent}{\hspace*{\tindent}}

\pagestyle{fancy}
\fancyhf{}
\rhead{Sam Robbins 13SE}
\lhead{A Level Maths - S3}
\rfoot{Page \thepage}


\begin{document}
\begin{center}
\underline{\huge Regression and Correlation}
\end{center}
\section{Spearman's rank}
Spearman's rank shows whether two sets of data agree or disagree.\\
\\
$r_s$ can take values from 1 to -1 (inclusive)
\subsection{$\mathbf{r_s=1}$}
The total sets of data are in perfect rank order.\\
The gradient between each successive point is positive\\
\\
\begin{tikzpicture}[scale=0.5]
\begin{axis}[ymin=0,ticks=none]
\addplot[color=black,domain=-2:2]{exp(x)};
\end{axis}
\end{tikzpicture}
\\
\subsection{$\mathbf{r_s=-1}$}
The gradient between each point is negative\\
\begin{tikzpicture}[scale=0.5]
\begin{axis}[ymin=0,ticks=none]
\addplot[color=black,domain=0.1:2]{1/x};
\end{axis}
\end{tikzpicture}
\subsection{Formula}
$$r_s=1-\dfrac{6\Sigma d^2}{n(n^2-1)}$$
\subsection{Example}
\begin{tabularx}{\textwidth}{|X|X|X|X|X|X|}
\hline
Height of a sunflower(cm)&Rank&Width of stem(mm)&Rank&d&$d^2$\\
\hline
183&\textbf{4}&21&\textbf{4}&\textbf{0}&\textbf{0}\\
\hline
134&\textbf{3}&14&\textbf{1}&\textbf{2}&\textbf{4}\\
\hline
234&\textbf{6}&24&\textbf{5}&\textbf{1}&\textbf{1}\\
\hline
256&\textbf{7}&32&\textbf{7}&\textbf{0}&\textbf{0}\\
\hline
190&\textbf{5}&29&\textbf{6}&\textbf{1}&\textbf{1}\\
\hline
89&\textbf{1}&18&\textbf{2}&\textbf{1}&\textbf{1}\\
\hline
112&\textbf{2}&20&\textbf{3}&\textbf{1}&\textbf{1}\\
\hline
\end{tabularx}
$\Sigma d^2=8$\\
\\
$1-\dfrac{6\times8}{7(7^2-1)}=\dfrac{6}{7}$
\section{Hypothesis Test For $\mathbf{r_s}$}
\underline{Step 1}\\
State $H_0$ and $H_1$\\
\\
$H_0: \rho_s=0$ (always)\\
\\
$H_1: \rho_s\neq0$ or\\
$H_1: \rho_s>0$ or\\
$H_1: \rho<0$\\
\\
\underline{Step 2}\\
State critical value from tables\\
\\
\underline{Step 3}\\
Calculate $r_s$\\
\\
\underline{Step 4}\\
Compare $r_s$ to the critical value. If in the critical region, testis significant, reject $H_0$ in favour of $H_1$\\
\\
\underline{Step 5}\\
Make conclusion in context of question
\section{Tied ranks}
If there are tied ranks and average of the positions they would take up





\end{document}