\documentclass{article}[18pt]
\usepackage{../../../../format}
\lhead{A Level Maths - C3}
\begin{document}
\begin{center}
\underline{\huge C3 Cheat Sheet}
\end{center}
\section{Algebraic fractions}
Algebraic long division can help to simplify a fraction, remember fractions can be expressed as:
$$Q(x)+\frac{R(x)}{D(x)}$$
Where $Q(x)$ is the quotient, $R(x)$ is the remainder and $D(x)$ is the divisor
\section{Functions}
\textbf{Domain} - Inputs\\
\textbf{Range} - Outputs\\
Reasons for a restricted domain:
\begin{itemize}
\item The denominator of the fraction can't be zero
\item You can't square root a negative number
\item You can't log numbers $\leqslant0$
\end{itemize}
Reasons for a restricted range:
\begin{itemize}
\item A restricted domain
\item Asymptotes
\item Minimum or maximum of a quadratic/trig graph
\end{itemize}
If the function doesn't have any obvious restrictions, still remember to put $x\in\mathbb{R}$, or $f(x)\in\mathbb{R}$
\subsection{Finding the inverse of a function}
Rearrange the function to make x the subject, then swap y for x and x for $f^{-1}(x)$\\
\\
The inverse function is a reflection in the line $y=x$ of the original function.\\
\\
If asked to find the values where the inverse function equals the original function, find where $f(x)=x$ as this will not have been transformed by the reflection\\
\\
A function will not have an inverse if it is a many to one function 
\subsection{Finding a composite function}
Substitute the inner function as x into the outer function
\subsection{What to do when not given an explicit function}
When not given an explicit function, remember that given a pair of coordinates $f(x)=f^{-1}(y)$, this can be used to find the function
\section{Exponential and log functions}
$$\log a+\log b=\log(ab)$$
$$\log a-\log b=\log(\frac{a}{b})$$
$$\log a^b=b\log a$$
Remember that you can only apply e to both sides when there is one term on each side, combine terms
\subsection{Quadratic functions}
Some questions with varying powers of $e$ are usually best solved as a quadratic, multiplying through by $e^{ax}$ then substituting $y=e^x$ and solving for $y$ then finding the natural log of the answer.
\subsection{Drawing graphs of $e^x$ and $\ln(x)$}
$\ln(x)$ is the inverse function of $e^x$, so just reflect in the line $y=x$, other than that, apply graph transformations as normal, remember to draw in asymptotes.
\section{Numerical methods}
To prove a root (turning point) is in a range, there will be a change of sign of the gradient.\\
Use the structure of the equation you are trying to rearrange to to help your method.
\section{Transforming graphs of functions}
\subsection{Modulus graphs}
$y=f(|x|)$ - Reflect in y axis (there can't be the correct values for negative x values)\\
$y=|f(x)|$ - Reflect in x axis (there can't be negative y values)\\
Remember that for curved graphs, the point of transformation will likely be sharp, rather than smooth
\subsection{Solving modulus equations}
To solve a modulus equation, use both the positive and negative versions, for example:
$$|3x-2|=x+4|$$
$$3x-2=x+4 \quad \textrm{and} -3x+2=x+4$$
Though remember to substitute the values found back into the original equation to check they are valid
\section{Trigonometry}
$$\sin(x)=\sin(180-x)$$
$$\cos(x)=\cos(360-x)$$
$$\sin \textrm{and} \cos \textrm{repeat every 360}\degree$$
$$\tan \textrm{repeats every 180}\degree$$
$$\sin(x)=\cos(90-x)$$
Remember to get the other identities, divide:
$$\sin^2x+\cos^2x=1$$
By either $\sin^2x$ or $\cos^2x$
\subsection{The R formula}
$a\sin\theta\pm b\cos\theta$ can be expressed in the form $R\sin(\theta\pm\alpha)$\\
$a\cos\theta\pm b\sin\theta$ can be expressed in the form $R\sin(\theta\mp\alpha)$\\
Where: 
$$R = \sqrt{a^2+b^2}$$
$$\alpha=\arctan(\frac{b}{a})$$
\newpage
\section{Differentiation}
Product rule:
$$y=f(x)g(x)\quad \textrm{then} \quad\frac{dy}{dx}=f'(x)g(x)+f(x)g'(x)$$
Quotient rule:
$$y=\frac{f(x)}{g(x)} \quad \textrm{then} \quad\frac{dy}{dx}=\frac{f'(x)g(x)-f(x)g'(x)}{(g(x))^2}$$
Chain rule:
$$y=[f(x)]^n \quad \textrm{then} \quad \frac{dy}{dx}=n[f(x)]^{n-1}f'(x)$$
$$y=f[g(x)] \quad \textrm{then} \quad \frac{dy}{dx}=f'[g(x)]g'(x)$$
$$\frac{dy}{dx}=\frac{dy}{du}\times\frac{du}{dx}$$
\subsection{Parallel to the y axis}
For a line to be parallel to the y axis, the gradient must be infinite, this would arise from dividing by zero, so if the function does this, the value at x at which this occurs is the tangent
\end{document}




































\end{document}