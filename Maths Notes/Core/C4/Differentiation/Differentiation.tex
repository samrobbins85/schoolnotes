\documentclass{article}[18pt]
\usepackage{/home/sam/Documents/School_Notes/format}
\lhead{A Level Maths - C4}

\begin{document}
\begin{center}
\underline{\huge Differentiation}
\end{center}
\section{Implicit differentiation}
An implicit function is in terms of $x$ and $y$ and cannot easily be written as $y=?$
\subsection{Formula}
$\dfrac{d}{dx}(y^n)=ny^{n-1}\times\dfrac{dy}{dx}$\\
This is a formula you need to remember for the exam
\subsection{Example}
$x^3+x+y^3+3y=6$\\
$\dfrac{d}{dx}(x^3+x+y^3+3y)=\dfrac{d}{dx}(6)$\\
\\
\textbf{Differentiate each term}\\
$3x^2+1+3y^2\frac{dy}{dx}+3\frac{dy}{dx}=0$\\
\\
\textbf{Collect terms}\\
$\dfrac{dy}{dx}(3y^2+3)=-3x^2-1$\\
\\
\textbf{Isolate $\mathbf{\frac{dy}{dx}}$}\\
$\dfrac{dy}{dx}=\dfrac{-3x^2-1}{3y^2+3}$
\\
\subsection{Example with combined xy terms}
$\dfrac{d}{dx}(x,y)=x'y+xy'$\\
\\
\textbf{Example}
\\
$\dfrac{d}{dx}xy$\\
\\
$\dfrac{d}{dx}x=1$\\
$\dfrac{d}{dx}y=\dfrac{dy}{dx}$\\
\\
$\dfrac{d}{dx}=1\times y+x\times\dfrac{dy}{dx}$\\
\\
\textbf{Example 2}\\
\textit{Find $\dfrac{dy}{dx}$ of $4xy^2+\dfrac{6x^2}{y}$}\\
\\
\textbf{Simplify}\\
$4xy^3+6x^2-10y=0$\\
\\
\textbf{Differentiate each term}\\
$\dfrac{d}{dx}=\bigg[1\times4y^3+4x\times3y^2\frac{dy}{dx}\bigg]+12x-10\frac{dy}{dx}$\\
\\
\textbf{Collect terms}\\
$\dfrac{dy}{dx}(12xy^2-10)=-(4y^3+12x)$\\
\\
\textbf{Isolate $\mathbf{\dfrac{dy}{dx}}$}\\
$\dfrac{dy}{dx}=\dfrac{-(4y^3+12x)}{12xy^2-10}$
\section{Differentiating $\mathbf{y=a^x}$}
$y=a^x$\\
\\
\textbf{Write in terms of logs}\\
$\ln(y)=x\ln(a)$\\
\\
\textbf{Differentiate both sides}\\
$\dfrac{d}{dx}\ln(y)=\dfrac{d}{dx}x\ln(a)$\\
\\
$\dfrac{1}{y}\dfrac{dy}{dx}=\ln(a)$\\
\\
\textbf{Simplify}\\
$\dfrac{dy}{dx}=y\ln(a)$\\
\\
\textbf{Sub y with $\mathbf{a^x}$}\\
$\dfrac{dy}{dx}=a^x\ln(a)$\\
\subsection{Differentiating $\mathbf{y=a^{f(x)}}$}
$\dfrac{dy}{dx}=a^{f(x)}f'(x)\ln(a)$\\
\section{Related rates of change}
You can use the chain rule to connect related rates of change\\
For example:\\
$\dfrac{dx}{dt}=\dfrac{dV}{dt}\times\dfrac{dx}{dV}$\\
Where $x$ is side length, v is volume and t is time.\\
\\
We need to be able to interpret numerical information as a rate of change, for example:\\
$2cm^2s^{-1}=\dfrac{dA}{dt}$\\
\\
\textbf{Example 1}\\
A cylinder is expanding under heat\\
After t seconds:\\
The radius is $x$ cm\\
The length is $5x$ cm\\
The cross sectional area is increasing at a constant rate of $0.037cm^2s^{-1}$\\
\\
\textit{Find $\frac{dx}{dt}$ when $r=4$}\\
\\
\textbf{Find the area in terms of x}\\
$A=\pi x^2$\\
\\
\textbf{Differentiate the area with respect to x}\\
$\dfrac{dA}{dx}=2\pi x$\\
\\
\\
\\
\textbf{Re write the rate of change}\\
$\dfrac{dA}{dt}=0.037$\\
\\
\textbf{Use the chain rule}
$\dfrac{dx}{dt}=\dfrac{dx}{dA}\times\dfrac{dA}{dt}$\\
\\
$\dfrac{dx}{dt}=\dfrac{0.037}{2\pi x}$\\
\\
\textbf{Sub x=4}\\
$\dfrac{dx}{dt}=0.00147cms^{-1}$\\
\\
\textit{Find the rate of change of the volume when x=4}\\
$V=5\pi x^3$\\
\\
\textbf{Differentiate the volume with respect to x}\\
$\dfrac{dV}{dx}=15\pi x^2$\\
\\
\textbf{Use the chain rule}\\
$\dfrac{dV}{dt}=\dfrac{dV}{dx}\times\dfrac{dx}{dt}=15\pi x^2\times 0.00147$\\
\\
\textbf{Sub x=4}\\
$1.11cm^3s^{-1}$


\end{document}