\documentclass{article}[18pt]
\usepackage{/home/sam/Documents/School_Notes/format}
\lhead{A Level Maths - C4}

%File specific Preamble
\usepackage{polynom} %Provides long division


\begin{document}
\begin{center}
\underline{\huge Partial fractions}
\end{center}
\section{Partial fractions}
This requires us to split difficult algebraic fractions:\\
\\
$\dfrac{4}{(x+1)(x+2)}\rightarrow\dfrac{A}{x+1}+\dfrac{B}{x+2}$\\
This allows us to:
\begin{itemize}
\item Do binomial expansion
\item Integrate using difficult fractions
\end{itemize}
\subsection{Normal Example}

$\dfrac{4}{(x+1)(x+2)}\rightarrow\dfrac{A}{x+1}+\dfrac{B}{x+2}$\\
\\
\textbf{Recombine:}\\
\\
$\dfrac{A(x+2)+B(x+1)}{(x+1)(x+2)}$\\
\\
\textbf{Equate:}\\
$A(x+2)+B(x+1)=4$\\
\\
\textbf{Eliminate one:}\\
\textit{Sub x=-1}\\
$A(-1+2)=4, \underline{A=4}$\\
\\
\textit{Sub x=-2}\\
$B(-2+1), \underline{B=-4}$\\
\\
\textbf{Write as partial fractions:}\\
$\dfrac{4}{x+1}-\dfrac{4}{x-1}$
\subsection{Example with a cubic denominator}
$\dfrac{6x^2+5x-2}{x(x+1)(2x+1)}$\\
\textbf{Expand}\\
$\dfrac{A}{x}+\dfrac{B}{x-1}+\dfrac{C}{2x+1}$\\
\textbf{Equate}\\
$A(x-1)(2x+1)+B(x)(2x+1)+C(x)(x-1)=6x^2+5x-2$\\
\textbf{Eliminate one}\\
\textit{Sub x=0}\\
$-A=-2, \underline{A=2}$\\
\\
\textit{Sub x=1}\\
$3B=9$\\
$\underline{B=3}$\\
\\
\textit{Sub $x=-\frac{1}{2}$}\\
$-\dfrac{3}{4}C=-3, \underline{C=-4}$\\
\textbf{Write as partial fractions}\\
$\dfrac{2}{x}+\dfrac{3}{x-1}-\dfrac{4}{2x+1}$
\subsection{Example with a Repeated root denominator}
$\dfrac{6x^2-29x-29}{(x+1)(x-3)^2}$\\
\\
\textbf{Expand}\\
$\dfrac{A}{x+1}+\dfrac{B}{x-3}+\dfrac{C}{(x-3)^2}$\\
\\
\textbf{Equate}\\
$A(x-3)^2+B(x+1)(x-3)+C(x+1)=6x^2-29x-29$\\
\\
\textbf{Eliminate one}\\
\textit{Sub x=-1}\\
$A(-1-3)^2+B(-1+1)(-1-3)+C(-1+1)=6(-1)^2-29(-1)-29$\\
$A(-4)^2+B(0)(-1-3)+C(0)=6(-1)^2-29(-1)-29$\\
$16A=6, \underline{A=\frac{3}{8}}$\\
\\
\textit{Sub x=3}\\
$4C=-62, \underline{C=-\dfrac{31}{2}}$\\
\\
\textit{Sub x=0}\\
$-29=9\times\frac{3}{8}-3B-\frac{31}{2}$\\
$\underline{B=\dfrac{45}{8}}$\\
\\
\textbf{Write as partial fractions}\\
$\dfrac{3}{8(x+1)}+\dfrac{45}{8(x-3)}-\dfrac{31}{2(x-3)^2}$
\subsection{Example with partial fractions with same or higher denominator}
$\dfrac{3x^2-3x-2}{(x-1)(x-2)}$\\
\\
\textbf{Long division to find remainder}\\
$\polylongdiv{3x^2-3x-2}{x^2-3x+2}$\\
\\
\textbf{Re-write with remainder}\\
$3+\dfrac{6x-8}{(x-1)(x-2)}$\\
\\
\textbf{Expand}\\
$\dfrac{A}{x-1}+\dfrac{B}{x-2}$\\
\\
\textbf{Equate}\\
$A(x-2)+B(x-1)=6x-8$\\
\\
\textbf{Eliminate one}\\
\textit{Sub x=1}\\
$-A=-2$\\
$\underline{A=2}$\\
\\
\textit{Sub x=2}\\
$3B=12$\\
$\underline{B=4}$\\
\\
\textit{Write as partial fractions}\\
$3+\dfrac{2}{x-1}+\dfrac{4}{x-2}$
\section{Binomial expansion}
Expansion can be done using the $(1+x)^n$ expansion, including with $(1+ax)^n$
\subsection{Negative powers}
\textbf{Example}
To expand $\dfrac{1}{1+x}$ turn it into $(1+x)^{-1}$ an use the formula from the book.\\
$1-x+x^2-x^3+x^4-x^5...$\\
As n is not a positive integer there will be no x coefficient equalling zero, meaning the expansion is infinite and convergent.\\
This gives valid values when $|x|<1$
\subsection{Fractional powers}
$\sqrt{1-3x}$\\
\textbf{Simplify}\\
$(1-3x)^{\frac{1}{2}}$\\
\\
\textbf{Find n and x}\\
$n=\frac{1}{2}$\\
$x=-3x$\\
\\
\textbf{Substitute into the formula}\\
$1+\frac{1}{2}\times-3x+\dfrac{\frac{1}{2}(\frac{1}{2}-1)}{1\times2}\times(-3x)^2$\\
\\
\textbf{Simplify}\\
$1-\frac{3}{2}x-\frac{9}{8}x^2$\\
\\
\textbf{Write conclusion}\\
Convergent and infinite when:
$|3x|<1$ $|x|<\frac{1}{3}$
\subsection{Applying $\mathbf{(1+x)^n}$ to $\mathbf{(a\pm bx)^n}$}
$(a\pm bx)^n$ can be rewritten as $a^n(1\pm\frac{b}{a}x)^n$
\subsubsection{Example}
\textit{Expand $\sqrt{4+x}$ to the $x^3$ term}\\
\textbf{Turn square root into power}\\
$(4-x)^{\frac{1}{2}}$\\
\\
\textbf{Rewrite with a 1 in the bracket}\\
$4^{\frac{1}{2}}(1+\frac{1}{4}x)^\frac{1}{2}$\\
\\
\textbf{Find n and x}\\
n=$\dfrac{1}{2}$\\
x=$\frac{1}{4}x$\\
\\
\textbf{Substitute into the formula}\\
\\
$2\Bigg[1+\dfrac{1}{2}\times\dfrac{1}{4}x+\dfrac{\frac{1}{2}(\frac{1}{2}-1)}{2!}\Bigg(\dfrac{1}{4}x\Bigg)^2+\dfrac{\frac{1}{2}(\frac{1}{2}-1)(\frac{1}{2}-2)}{3!}\Bigg(\dfrac{1}{4}x\Bigg)^3\Bigg]$
\\
\textbf{Simplify}\\
$2\Bigg[1+\dfrac{x}{8}-\dfrac{x^2}{128}+\dfrac{x^3}{1024}\Bigg]$\\
\\
$2+\dfrac{x}{4}-\dfrac{x^2}{64}+\dfrac{x^3}{512}$\\
\\
\textbf{Write conclusion}\\
Valid if $\Big|\dfrac{x}{4}\Big|<1$ so valid if $|x|<4$
\subsection{Unknown coefficient type}
$(a+bx)^{-2}$ can be approximated by\\
$a(1+\frac{b}{a}x)^{-2}$\\
$\dfrac{1}{a^2}(1-2\frac{b}{a}x)$\\
\subsection{Fractional type}
Expand up to $x^3$ $\dfrac{1+x}{2+x}$\\
\\
\textbf{Re-Write using powers}\\
$(1+x)(2+x)^{-1}$\\
\\
\textbf{Ensure there is only a 1 in the bracket}\\
$2(1+\frac{1}{2}x)^{-1}$\\
\\
\textbf{Find n and x}\\
$n=-1$\\
$x=\frac{1}{2}x$\\
\\
\textbf{Substitute into the formula}\\
$\dfrac{1}{2}\Bigg(1+-1\times\dfrac{1}{2}x\Bigg)+\dfrac{-1(-1-1)}{2!}\Bigg(\dfrac{1}{2}(x)^2\Bigg)^2+\dfrac{-1(-1-1)(-1-2}{3!}\Bigg(\dfrac{1}{2}x\Bigg)^3$\\
\\
\textbf{Simplify}\\
$(1+x)\Bigg(\dfrac{1}{2}-\dfrac{1}{4}x+\dfrac{1}{8}x^2-\dfrac{1}{16}x^3\Bigg)$\\
$\dfrac{1}{2}+\dfrac{1}{4}x-\dfrac{1}{8}x^2+\dfrac{1}{16}x^3$\\
\\
\textbf{Write conclusion}\\
Valid if $x\neq2$\\
\newpage
\subsection{Approximating roots}
\textit{Find the expansion of $\sqrt{1-2x}$ up to $x^3$}\\
\textbf{Re-Write using powers}\\
$(1-2x)^{\frac{1}{2}}$\\
\\
\textbf{Find n and x}\\
$n=\dfrac{1}{2}$\\
$x=-2x$\\
\\
\textbf{Substitute into the formula}\\
$1+\Bigg(\dfrac{1}{2}\times-2x\Bigg)+\dfrac{\frac{1}{2}(\frac{1}{2}-1)}{2!}\times(-2x)^2+\dfrac{\frac{1}{2}(\frac{1}{2}-1)(\frac{1}{2}-2)}{3!}\times(-2x)^3$\\
\\
\textbf{Simplify}
$1-x+\dfrac{1}{2}x^2-\dfrac{1}{3}x^3$\\
\\
\textit{By substituting $x=0.01$, find a suitable approximation of $\sqrt{2}$}\\
\textbf{Substitute values}
$\sqrt{1-\dfrac{2}{100}}=1-\dfrac{1}{100}-\dfrac{(\frac{1}{100})^2}{2}-\dfrac{(\frac{1}{100})^3}{2}$\\
\\
\textbf{Simplify}
$\sqrt{\dfrac{98}{100}}=\dfrac{\sqrt{98}}{10}=\dfrac{7\sqrt{2}}{10}$\\
\\
\textbf{Rearrange}\\
$\sqrt{2}\approx\dfrac{10}{7}\Bigg(1-\dfrac{1}{100}-\dfrac{(\frac{1}{100})^2}{2}-\dfrac{(\frac{1}{100})^3}{2}\Bigg)$\\
$\sqrt{2}=1.414213571$















\end{document}