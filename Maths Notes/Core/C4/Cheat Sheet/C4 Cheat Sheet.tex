\documentclass{article}[18pt]
\usepackage{../../../../format}
\lhead{A Level Maths - C4}

%File specific Preamble
\usepackage{polynom} %Provides long division

\begin{document}
\begin{center}
\underline{\huge C4 Cheat Sheet}
\end{center}
\section{Partial fractions}
\subsection{Split a fraction whose denominator is a product of linear expressions}
Set the fraction equal to the sum of constants over each linear expression, then multiply by the denominator and solve by substitution.
\subsection{Split a fraction where one or more of the factors in the denominator are squared}
Each squared term gets two terms in the summation, one of the non squared term, and one of the squared term.
\subsection{Deal with top heavy fractions}
Use long division to simplify, for example simplifying $\dfrac{3x^2-3x-2}{(x-1)(x-2)}$\\
\textbf{Long division to find remainder}\\
$\polylongdiv{3x^2-3x-2}{x^2-3x+2}$\\
\\
\textbf{Re-write with remainder}\\
\\
$3+\dfrac{6x-8}{(x-1)(x-2)}$\\
\textbf{Partial fractions can then be applied normally}
\section{Parametric equations}
$$\frac{dy}{dx}=\frac{(\frac{dy}{dt})}{(\frac{dx}{dt})}$$
$$\int y \ dx=\int y\frac{dx}{dt}dt$$
Remember to make use of $\sin^2x+\cos^2x=1$ when converting between parametric and Cartesian
\section{Binomial expansion}
$$(a+bx)^n=a^n(a+\frac{b}{a}x)^n$$
Remember that fractions can be simplified for binomial expansion using negative powers
\section{Differentiation}
On C3 data sheet:
$$a^x=e^{\ln(a^x)}=e^{x\ln(a)}$$
$$\frac{d}{dx}(a^x)=\ln a\times e^{x\ln a}=a^x\ln a$$
\subsection{Implicit differentiation}
$$\frac{d}{dx}(y^2)=2y\frac{dy}{dx}$$
\subsection{Setting up differential equations}
\textbf{Example:}\\
\textit{Temperature falls at a rate proportional to its current temperature}
$$\frac{dT}{dt}=-kT$$
\subsection{Related rates of change}
When given a rate and asked to find a different rate, use equations like
$$\frac{dA}{dx}=\frac{dA}{dt}\times\frac{dt}{dx}$$
Remember that finding relations between length, area and volume can be found using known formulas
\section{Vectors}
\subsection{Point of intersection}
Set the two vector lines equal to each other, find $\lambda$ etc, substitute in to see if they agree, remember to write the final point
\subsection{Angle between two lines}
$$\cos\theta=\frac{a\cdot b}{|a||b|}$$
This can be used when lines are perpendicular as then
$$a\cdot b=|a||b|$$
\subsection{Find a missing x/y/z value of a point on a line}
Solve the equation for the two known points to find $\lambda$, then use that to find the other point
\subsection{Find the length of a vector/distance between two points}
Find the equation of the vector using the formula
$$\overrightarrow{AB}=\begin{pmatrix}x_b-x_a\\y_b-y_a\\z_b-z_a\end{pmatrix}$$
Then use
$$|AB|=\sqrt{x^2+y^2+z^2}$$
\subsection{Find the nearest point on a line to a point not on the line}
The shortest distance is always at a right angle, so you can use the fact that:
$$|a||b|=a\cdot b$$
Set up the equation of the line including $\lambda$, then solve the equation of the two of them being at right angles to find the value of $\lambda$ and so find the point
\subsection{Show that 3 points are collinear}
Show that the vector between the first and second points is a multiple of the vector between the second and third points
\subsection{Find the equation of a line from two points}
Find the direction vector from $\overrightarrow{AB}$ then arrange it either the form:
$$A+\lambda\Big(\overrightarrow{AB}\Big) \quad \textrm{or} \quad B+\lambda\Big(\overrightarrow{AB}\Big)$$
\subsection{Find the reflection of a point in a line}
Find the shortest distance vector from the point to the line, then draw the vector coming out the other side of the line (negative of the vector)
\subsection{Find the point after going a specific distance in the direction of a given vector}
\textbf{Example}:\\
\textit{What is the position vector 10 units in the direction
$\begin{pmatrix}1\\0\\3\end{pmatrix}$ from the point $\begin{pmatrix}2\\2\\2\end{pmatrix}$?}\\
\textcolor{red}{Convert the direction into a unit vector}
$$\sqrt{1^2+0^2+3^2}=2$$
This means that the vector is 2 units away from the origin, so to create the unit vector (1 unit away from the origin) multiply it by $\frac{1}{2}$
$$\frac{1}{2}\begin{pmatrix}2\\2\\2\end{pmatrix}$$
Then add 10 of these units to the initial vector point to get the answer
$$\begin{pmatrix}1\\2\\2\end{pmatrix}+10\times\frac{1}{2}\begin{pmatrix}2\\2\\2\end{pmatrix}=\begin{pmatrix}7\\2\\17\end{pmatrix}$$
\section{Integration}
\subsection{Integral of Trig squares}
{\renewcommand{\arraystretch}{2}
\begin{tabularx}{\textwidth}{|X|X|}
\hline
$f(x)$&$\int f(x) \ dx$\\
\hline
$\sec^2x$&On formula book\\
\hline
$\csc^2x$&On differentiation side of formula book\\
\hline
$\cot^2x$&Use the identity
$1+\cot^2x=\csc^2x$, then use known integral of $\csc^2x$ to integrate\\
\hline
$\sin^2x$&Use cos addition formula\\
\hline
$\cos^2x$&Use cos addition formula\\
\hline
$\tan^2x$&Use the identity $\tan^2x+1=\sec^2x$\\
\hline
\end{tabularx}}
\end{document}