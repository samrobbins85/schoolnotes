\documentclass{article}[18pt]
\usepackage{../../../../format}
\lhead{A Level Maths - C4}

\begin{document}
\begin{center}
\underline{\huge Coordinate Geometry}
\end{center}
\section{Parametric equations}
In parametric equations, coordinates x and y are expressed as functions.\\
$x=f(t)$ and $y=g(t)$ where t is the independent variable (called a parameter)
\subsection{Converting parametric equations into Cartesian form}
\textbf{Example 1}\\
$x=2t$\\
$y=t^2$\\
\\
$t=\dfrac{x}{2}$\\
\\
$y=\frac{1}{4}x^2$\\
\\
\textbf{Example 2}\\
$x=\sin(t)+2$\\
$y=\cos(t)-3$\\
\\
$\sin(t)=x-2$\\
$\cos(t)=y+3$\\
\\
$\sin^2\theta+\cos^2\theta=1$\\
$(x-2)^2+(y+3)^2=1$
\section{Parametric differentiation}
The chain rule can be used to differentiate parametric equations\\
$\dfrac{dy}{dx}=\dfrac{dy}{dt}\times\dfrac{dt}{dx}$\\
This can be used to find the gradient of normals and tangents.\\
\\
\textbf{Example}\\
$x=t^3+t$\\
$y=t^2+1$\\
\\
\textbf{Differentiate the x and y terms}\\
$\dfrac{dy}{dt}=2t$\\
$\dfrac{dx}{dt}=3t^2+1$\\
\\
\textbf{Invert the differential of x to fit the formula}\\
$\dfrac{dt}{dx}=\dfrac{1}{3t^2+1}$\\
\\
\textbf{Multiply together to make one fraction}\\
$\dfrac{dy}{dx}=\dfrac{2t}{3t^2+1}$

\end{document}