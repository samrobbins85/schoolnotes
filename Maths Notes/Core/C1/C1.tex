
\documentclass{article}[18pt]
\usepackage{../../../format}
\lhead{A Level Maths - C1}


\begin{document}
\begin{center}
\underline{\huge C1}
\end{center}
\section{Algebra and functions}
\subsection{Indices}
$a^m\times a^n=a^{m+n}$\\
$a^m\div a^n=a^{m-n}$\\
$(a^m)^n=a^{mn}$\\
$a^{-m}=\frac{1}{a^m}$\\
$a^{\frac{1}{m}}=\sqrt[m]{a}$\\
$a^{\frac{n}{m}}=\sqrt[m]{a^n}$\\
\section{Quadratic functions}
\subsection{Discriminant}
\begin{itemize}
\item $b^2>4ac$, Two real solutions
\item $b^2=4ac$, One repeated real solution
\item $b^2<4ac$, No real solution
\end{itemize}
\section{Sketching curves}
\subsection{Graph transformations}
\begin{itemize}
\item f(x)+a, $y$ coordinates increased by a
\item af(x), $y$ coordinates multiplied by a
\item -f(x), reflection in the $x$ axis
\item f(x+a), $x$ coordinates reduced by a
\item f(ax), $x$ coordinates divided by a
\item f(-x), reflection in the $y$ axis
\end{itemize}
\section{Coordinate geometry}
If two lines are perpendicular, the product of their gradients is -1.
\section{Sequences and series}
\subsection{Deriving the formula for the sum of an arithmetic series}
$$S_n=a+(a+d)+(a+2d)+...+(a+(n-1)d)$$
Reverse the sun
$$S_n=(a+(n-1)d)+(a+(n-2)d)+a(a+(n-3)d)+...+(a+d)+a$$
Add the two sums
$$2S_n=[2a+(n-1)d]+[2a+(n-1)d]+[2a+(n-1)d]+...+[2a+(n-1)d]$$
$$2S_n=n[2a+(n-1)d]$$
$$S_n=\frac{n}{2}[2a+(n-1)d]$$
\subsection{Forming a recurrence relation}
To create a recurrence relation formula from a formula given in n, substitute n+1 in place of n and rearrange.\\
\textbf{Example}
$$U_n=2^n+4n$$
$$U_{n+1}=2^{n+1}+4{n+1}$$
$$U_{n+1}=2U_n-4n+4$$
\subsection{Finding a and d}
Remember that the formula for the $\textrm{n}^\textrm{th}$ term may not be given in the form $a+(n-1)d$, and if it is not, it must be changed.\\
\textbf{Example}\\
$u_r=4r-7$\\
$u_1=4-7=-3=a$\\
$4r-7=-3+(r-1)d \quad \Rightarrow 4r-4=(r-1)d \quad \Rightarrow d=4$
\subsection{Finding the nth term from a recurrence relation}
To find the nth term, use a standard form, deducted from what is happening to the previous term.\\
\textbf{Example}\\
$U_{n+1}=2U_n-5$\\
The multiplier by 2 implies doubling, and so the nth term will contain $A\times2^n$, the subtraction of 5 suggests there will also be a constant term, $B$.\\
$$U_n=A\times2^n+B$$
Given $U_1=6$, A and B can be found to be $\frac{1}{2}$ and 5.


\end{document}