\documentclass{article}[18pt]
\usepackage[utf8]{inputenc}
\usepackage[margin=0.7in]{geometry}
\usepackage{parselines} 
\usepackage{amsmath}
\usepackage{titlesec}
\usepackage{pgfplots}
\usepackage{graphicx}
\usepackage[english]{babel}
\usepackage{fancyhdr}
\usepackage{amssymb}
\pgfplotsset{width=10cm,compat=1.9}
\usepackage{mwe}
\usepackage{float}
\usepackage{polynom}
\usepackage{tabu}
\usepackage{relsize}
\titlespacing\section{0pt}{14pt plus 4pt minus 2pt}{0pt plus 2pt minus 2pt}
\newlength\tindent
\setlength{\tindent}{\parindent}
\setlength{\parindent}{0pt}
\renewcommand{\indent}{\hspace*{\tindent}}
\usepackage{mathtools}
\pagestyle{fancy}
\fancyhf{}
\rhead{Sam Robbins 13SE}
\lhead{A Level Maths - C4}
\rfoot{Page \thepage}
\DeclarePairedDelimiter\abs{\lvert}{\rvert}
\newcommand{\R}{\mathbb{R}}
\pgfplotsset{ytick style={draw=none}}
\pgfplotsset{xtick style={draw=none}}

\begin{document}
\begin{center}
\underline{\huge Integration}
\end{center}
\section{Standard Results}
Standard results not on formula book:\\
\\
{\renewcommand{\arraystretch}{2}
\begin{tabu} to \textwidth { | X[c] | X[c] | }
 \hline
 \textbf{f(x)} & $\mathbf{\mathlarger{\int} f(x)\ dx}$ \\
 \hline
 $x^n$  & $\dfrac{x^{n+1}}{n+1}+c$ \\
\hline
$e^x$&$e^x+c$\\
\hline
$\dfrac{1}{x}$&$\ln|x|+c$\\
\hline
$\sin x$&$-\cos x$\\
\hline
$\cos x$&$\sin x$\\
\hline
$a^x$&$\dfrac{a^x}{\ln|a|}+c$\\
\hline
\end{tabu}
}
\section{Integration by substitution}
\textbf{Example}\\
$\mathlarger{\int}(3x+2)^4 \ dx$\\
\\
\textbf{Find a value for u}\\
$u=3x+2$\\
\\
\textbf{Substitute the value of u into the initial question}\\
$\mathlarger{\int}u^4 \ dx$\\
\\
\textbf{Find dx in terms of du}\\
$\dfrac{du}{dx}=3$\\
$du=3dx$\\
$dx=\frac{1}{3}du$\\
\textcolor{red}{Note - if an x term remains with the dx there may be a way to replace both the x term and dx}
\\
\textbf{Replace dx with du into the modified initial equation}\\
$\frac{1}{3}\mathlarger{\int}u^4 \ du$\\
\\
\textbf{Integrate using normal integration rules}\\
$\frac{1}{3}\mathlarger{\int}u^4 \ du=\frac{1}{15}u^5$\\
\\
\textbf{Substitute the value for u back in}\\
$\frac{1}{15}u^5=\frac{1}{15}(3x+2)^5$\\
\newpage
\section{Parametric integration}
$\mathlarger{\int}y \ dx$ gives the area under the function y.\\
\\
For parametric equations we use:\\
$\mathlarger{\int}y\dfrac{dx}{dt}dt$\\
\\
We have to integrate with respect to t or $\theta$ as it is the parameter.\\
Limits will sometimes need to be changed from x to t or $\theta$.

\subsection{Example}
$x=5t^2$\\
$y=t^3$\\
Find $\mathlarger{\int_1^2}y \ dt$\\
\\
\\
\textbf{Find $\mathbf{\dfrac{dx}{dt}}$}\\
$\dfrac{dx}{dt}=10t$\\
\\
\textbf{Substitute into the model $\mathbf{\mathlarger{\int}y\dfrac{dx}{dt} \ dt}$}\\
$\mathlarger{\int_1^2}t^3\times10t \ dt = \mathlarger{\int_1^2}10t^4 \ dt = \Big[2t^5\Big]^2_1=2\times2^5-2\times1^5=64-2=62$\\
\section{Integration using trig identities}
\subsection{Example 1 - sinf(x)cosf(x)}
$\mathlarger{\int}\sin(3x)\cos(3x) \ dx$\\
\\
\textbf{Use addition formula to simplify}\\
\textcolor{red}{$\sin2x=2\cos x\sin x$}\\
\textcolor{red}{$\sin3x\cos3x=\frac{1}{2}\sin6x$}\\
$\frac{1}{2}\mathlarger{\int}\sin6x \ dx$\\
\\
\textbf{Find a value for u}\\
$u=6x$\\
\\
\textbf{Find dx in terms of du}\\
$\dfrac{du}{dx}=6$\\
$dx=\frac{1}{6}du$\\
\\
\textbf{Substitute two conversions from x to u}\\
$\frac{1}{2}\times\frac{1}{6}\mathlarger{\int}\sin u \ du$\\
\\
\textbf{Use integration of sin formula}\\
$\frac{1}{12}\mathlarger{\int}\sin u \ du=-\frac{1}{12}\cos6x+c$
\newpage
\subsection{Example 2 - sinf(x)cosg(x)}
$\mathlarger{\int}\sin(3x)\cos(2x) \ dx$\\
\\
\textbf{Add together sin addition formula and sin subtraction formula}\\
\textcolor{red}{$\sin(A+B)=\sin A\cos B+\cos A\sin B$}\\
\textcolor{red}{$\sin(A-B)=\sin A\cos B-\cos A\sin B$}\\
\textcolor{red}{$\sin(A+B)+\sin(A-B)=2\sin A\cos B$}\\
\textcolor{red}{$\sin 3x\cos 2x=\sin(3x+2x)+\sin(3x-2x)$}\\
$\frac{1}{2}\mathlarger{\int}\sin5x+\sin x \ dx=\frac{1}{2}\mathlarger{\int}\sin5x \ dx+\frac{1}{2}\mathlarger{\int}\sin x \ dx$\\
\\
\textbf{Integrate using sin standard result}\\
$-\frac{1}{10}\cos5x-\frac{1}{1}\cos x+c$
\subsection{Example 3 - $\sin^2f(x)\cos^2f(x)$}
$\mathlarger{\int}\sin^2(x)cos^2(x) \ dx$\\
\textbf{Give $\mathbf{sin^2(x)cos^2(x)}$ in terms of $\mathbf{sin^2(x)}$}\\
\textcolor{red}{$2\sin(x)\cos(x)=\sin2x$}\\
\textcolor{red}{$4\sin^2(x)\cos^2(x)=\sin^22x$}\\
$\cos^2x\sin^2x=\frac{1}{4}\sin^22x$\\
\textbf{Sub simplified version}\\
$\mathlarger{\int}\frac{1}{4}\sin^22x \ dx$\\
\textbf{Give $\mathbf{sin^2x}$ in terms of $\mathbf{cosx}$}\\
\textcolor{red}{$2\sin^22x=\sin^22x+(1-\cos^22x)$}\\
\textcolor{red}{$2\sin^22x=1-(\cos^22x-\sin^22x)$}\\
\textcolor{red}{$2\sin^22x=1-\cos4x$} - Use of cos subtraction formula\\
$\sin^22x=\frac{1}{2}-\frac{1}{2}\cos4x$\\
\textbf{Sub simplified version}\\
$\frac{1}{4}\mathlarger{\int}\frac{1}{2}-\frac{1}{2}\cos4x \ dx$\\
$\frac{1}{8}\mathlarger{\int}1-\cos4x \ dx$\\
\textbf{Use cos integration standard result}\\
$\frac{1}{8}(x-\frac{1}{4}\sin4x)+c$\\
\\
$\frac{1}{8}x-\frac{1}{32}\sin4x+c$\\
\newpage
\section{Integration of partial fractions}
\subsection{Simple example}
$\mathlarger{\int}\dfrac{x-5}{(x+1)(x-2)}$\\
\textbf{Write as separate fractions}\\
$\dfrac{A}{x+1}+\dfrac{B}{x-2}$\\
\\
\textbf{Multiply out}\\
$A(x-2)+B(x+1)=x-5$\\
\\
\textbf{Solve}\\
B=-1\\
A=2\\
\\
\textbf{Substitute}\\
$\mathlarger{\int}\dfrac{2}{x+1}-\dfrac{1}{x-2} \ dx$\\
\\
\textbf{Separate and simplify}\\
$2\mathlarger{\int}\dfrac{1}{x+1} \ dx-\mathlarger{\int}\dfrac{1}{x-2} \ dx$\\
\\
\textbf{Use integration of $\mathbf{\dfrac{1}{x}}$ standard result}\\
$2\ln(x+1)-\ln(x-2)+c$\\
\\
\textbf{Simplify using laws of logs}\\
$\ln\abs*{\dfrac{(x+1)^2}{x-2}}+c$
\section{Integration by substitution - Fractional type}
$\mathlarger{\int}\dfrac{f'(x)}{f(x)} \ dx$ can be written as $\ln f(x)+c$\\
\section{Integration by parts}
This is the integration equivalent of the product rule\\
$\dfrac{d}{dx}(uv)=\dfrac{du}{dx}\times V+\dfrac{dV}{dx}\times u$\\
\\
$u\dfrac{dv}{dx}=\dfrac{d}{dx}(uv)-\dfrac{du}{dx}v$\\
\\
$\mathlarger{\int}u\dfrac{dv}{dx}dx=\mathlarger{\int}\dfrac{d}{dx}(uv)dx-\mathlarger{\int}\dfrac{du}{dx}v \ dx$\\
\\
$\mathlarger{\int}u\dfrac{dv}{dx}dx=uv-\mathlarger{\int}\dfrac{du}{dx}v \ dx$\\
\\
Easier to read version:\\
$\mathlarger{\int}uv' \ dx=uv-\mathlarger{\int}u'v \ dx$
\newpage
\subsection{Example}
$\mathlarger{\int}x\cos x \ dx$\\
$u=x$ - can be eliminated by differentiating\\
$v'=\cos x$\\
\\
$u'=1$\\
$v=\sin x$\\
\\
$\mathlarger{\int}x\cos x \ dx=x\sin x-\mathlarger{\int}1\times\sin x \ dx$\\
\\
$\mathlarger{\int}x\cos x \ dx=x\sin x+\cos x+c$\\
\subsection{Double application}
When integrating by parts once gives an integral that cannot be solved simply, a second application of integration is needed.\\
\\
A double application can sometimes lead to the integral having the negative version of the integral on the other side. In this case, add this integral to both sides, allowing both sides to be divided by 2, eliminating a term.
\section{Trapezium rule}
In C4 we may be asked to find the \% error between the trapezium rule approximation and an exact value.\\
\\
$$\text{\% error}=\dfrac{\text{Approx-Actual}}{\text{Actual}}\times 100$$
\section{Volume of revolutions}
This finds the volume of a graph when rotated round the x axis.\\
$$\text{Volume}=\pi\mathlarger{\int_a^b}y^2 \ dx$$
\subsection{Example}
$y=\sin2x$\\
\textit{Find the volume between $x=0$ and $x=\frac{\pi}{2}$ when rotated round the $x$ axis by $2\pi$}\\
\\
Volume = $\pi\mathlarger{\int_0^\frac{\pi}{2}}\sin^22x \ dx$\\
\\
Volume = $\pi\mathlarger{\int_0^\frac{\pi}{2}}\frac{1}{2}(1-\cos4x) \ dx$\\
\\
Volume = $\dfrac{\pi}{2}\mathlarger{\int_0^\frac{\pi}{2}}1 \ dx - \dfrac{\pi}{2}\mathlarger{\int_0^\frac{\pi}{2}}\cos4x \ dx$\\
\\
Volume = $\Big[\dfrac{\pi}{2}x\Big]^\frac{\pi}{2}_0-\Big[\dfrac{\pi}{8}\sin4x\Big]^\frac{\pi}{2}_0$\\
\\
Volume = $\dfrac{\pi^2}{4}$
\newpage
\section{Differential equations}
$\dfrac{dy}{dx}=f(x)\cdot g(y)$\\
\\
\textbf{Separate variables}\\
$\dfrac{dy}{g(y)}=f(x) \ dx$\\
\\
\textbf{Solve using integration}\\
$\mathlarger{\int}\dfrac{1}{g(y)} \ dy=\mathlarger{\int}f(x) \ dx$
\subsection{General solution}
General solutions have an unknown constant term
\subsection{Example}
$\dfrac{dy}{dx}=(1+y)(1-2x)$\\
\\
$\mathlarger{\int}\dfrac{1}{1+y} \ dy = \mathlarger{\int}1-2x \ dx$\\
\\
$\ln(1+y)=x-x^2+c$\\
\\
$1+y=e^{x-x^2}\times k$\\
\\
$y=ke^{x(1-x)}-1$
\subsection{Boundary conditions}
We can find a "particular solution" to a differential equation if we are given boundary conditions.\\
\\
You start by finding the general solution, then apply the boundary conditions.
\subsubsection{Example}
The rate of change of the volume of a container is with respect to time.
\end{document}