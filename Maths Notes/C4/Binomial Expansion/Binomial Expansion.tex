\documentclass{article}[18pt]
\usepackage[utf8]{inputenc}
\usepackage[margin=0.7in]{geometry}
\usepackage{parselines} 
\usepackage{amsmath}
\usepackage{titlesec}
\usepackage{pgfplots}
\usepackage{graphicx}
\usepackage[english]{babel}
\usepackage{fancyhdr}
\usepackage{amssymb}
\pgfplotsset{width=10cm,compat=1.9}
\usepackage{mwe}
\usepackage{float}
\usepackage{polynom}


\titlespacing\section{0pt}{14pt plus 4pt minus 2pt}{0pt plus 2pt minus 2pt}
\newlength\tindent
\setlength{\tindent}{\parindent}
\setlength{\parindent}{0pt}
\renewcommand{\indent}{\hspace*{\tindent}}

\pagestyle{fancy}
\fancyhf{}
\rhead{Sam Robbins 13SE}
\lhead{A Level Maths - C4}
\rfoot{Page \thepage}

\newcommand{\R}{\mathbb{R}}
\pgfplotsset{ytick style={draw=none}}
\pgfplotsset{xtick style={draw=none}}

\begin{document}
\begin{center}
\underline{\huge Binomial Expansion}
\end{center}
\section{Introduction to Binomial expansion}
Expansion can be done using the $(1+x)^n$ expansion, including with $(1+ax)^n$
\section{Negative powers}
\textbf{Example}
To expand $\dfrac{1}{1+x}$ turn it into $(1+x)^{-1}$ an use the formula from the book.\\
$1-x+x^2-x^3+x^4-x^5...$\\
As n is not a positive integer there will be no x coefficient equalling zero, meaning the expansion is infinite and convergent.\\
This gives valid values when $|x|<1$
\section{Fractional powers}
$\sqrt{1-3x}$\\
\textbf{Simplify}\\
$(1-3x)^{\frac{1}{2}}$\\
\\
\textbf{Find n and x}\\
$n=\frac{1}{2}$\\
$x=-3x$\\
\\
\textbf{Substitute into the formula}\\
$1+\frac{1}{2}\times-3x+\dfrac{\frac{1}{2}(\frac{1}{2}-1)}{1\times2}\times(-3x)^2$\\
\\
\textbf{Simplify}\\
$1-\frac{3}{2}x-\frac{9}{8}x^2$\\
\\
\textbf{Write conclusion}\\
Convergent and infinite when:
$|3x|<1$ $|x|<\frac{1}{3}$
\section{Applying $\mathbf{(1+x)^n}$ to $\mathbf{(a\pm bx)^n}$}
$(a\pm bx)^n$ can be rewritten as $a^n(1\pm\frac{b}{a}x)^n$
\subsection{Example}
\textit{Expand $\sqrt{4+x}$ to the $x^3$ term}\\
\textbf{Turn square root into power}\\
$(4-x)^{\frac{1}{2}}$\\
\\
\textbf{Rewrite with a 1 in the bracket}\\
$4^{\frac{1}{2}}(1+\frac{1}{4}x)^\frac{1}{2}$\\
\\
\textbf{Find n and x}\\
n=$\dfrac{1}{2}$\\
x=$\frac{1}{4}x$\\
\\
\textbf{Substitute into the formula}\\
\\
$2\Bigg[1+\dfrac{1}{2}\times\dfrac{1}{4}x+\dfrac{\frac{1}{2}(\frac{1}{2}-1)}{2!}\Bigg(\dfrac{1}{4}x\Bigg)^2+\dfrac{\frac{1}{2}(\frac{1}{2}-1)(\frac{1}{2}-2)}{3!}\Bigg(\dfrac{1}{4}x\Bigg)^3\Bigg]$
\newpage
\textbf{Simplify}\\
$2\Bigg[1+\dfrac{x}{8}-\dfrac{x^2}{128}+\dfrac{x^3}{1024}\Bigg]$\\
\\
$2+\dfrac{x}{4}-\dfrac{x^2}{64}+\dfrac{x^3}{512}$\\
\\
\textbf{Write conclusion}\\
\\
Valid if $\Big|\dfrac{x}{4}\Big|<1$ so valid if $|x|<4$
\section{Unknown coefficient type}
$(a+bx)^{-2}$ can be approximated by\\
$a(1+\frac{b}{a}x)^{-2}$\\
$\dfrac{1}{a^2}(1-2\frac{b}{a}x)$\\
\section{Fractional type}
Expand up to $x^3$ $\dfrac{1+x}{2+x}$\\
\\
\textbf{Re-Write using powers}\\
$(1+x)(2+x)^{-1}$\\
\\
\textbf{Ensure there is only a 1 in the bracket}\\
$2(1+\frac{1}{2}x)^{-1}$\\
\\
\textbf{Find n and x}\\
$n=-1$\\
$x=\frac{1}{2}x$\\
\\
\textbf{Substitute into the formula}\\
$\dfrac{1}{2}\Bigg(1+-1\times\dfrac{1}{2}x\Bigg)+\dfrac{-1(-1-1)}{2!}\Bigg(\dfrac{1}{2}(x)^2\Bigg)^2+\dfrac{-1(-1-1)(-1-2}{3!}\Bigg(\dfrac{1}{2}x\Bigg)^3$\\
\\
\textbf{Simplify}\\
$(1+x)\Bigg(\dfrac{1}{2}-\dfrac{1}{4}x+\dfrac{1}{8}x^2-\dfrac{1}{16}x^3\Bigg)$\\
$\dfrac{1}{2}+\dfrac{1}{4}x-\dfrac{1}{8}x^2+\dfrac{1}{16}x^3$\\
\\
\textbf{Write conclusion}\\
Valid if $x\neq2$\\
\newpage
\section{Approximating roots}
\textit{Find the expansion of $\sqrt{1-2x}$ up to $x^3$}\\
\textbf{Re-Write using powers}\\
$(1-2x)^{\frac{1}{2}}$\\
\\
\textbf{Find n and x}\\
$n=\dfrac{1}{2}$\\
$x=-2x$\\
\\
\textbf{Substitute into the formula}\\
$1+\Bigg(\dfrac{1}{2}\times-2x\Bigg)+\dfrac{\frac{1}{2}(\frac{1}{2}-1)}{2!}\times(-2x)^2+\dfrac{\frac{1}{2}(\frac{1}{2}-1)(\frac{1}{2}-2)}{3!}\times(-2x)^3$\\
\\
\textbf{Simplify}\\
$1-x+\dfrac{1}{2}x^2-\dfrac{1}{3}x^3$\\
\\
\textit{By substituting $x=0.01$, find a suitable approximation of $\sqrt{2}$}\\
\\
\textbf{Substitute values}\\
\\
$\sqrt{1-\dfrac{2}{100}}=1-\dfrac{1}{100}-\dfrac{(\frac{1}{100})^2}{2}-\dfrac{(\frac{1}{100})^3}{2}$\\
\\
\textbf{Simplify}\\
\\
$\sqrt{\dfrac{98}{100}}=\dfrac{\sqrt{98}}{10}=\dfrac{7\sqrt{2}}{10}$\\
\\
\textbf{Rearrange}\\
$\sqrt{2}\approx\dfrac{10}{7}\Bigg(1-\dfrac{1}{100}-\dfrac{(\frac{1}{100})^2}{2}-\dfrac{(\frac{1}{100})^3}{2}\Bigg)$\\
$\sqrt{2}=1.414213571$















\end{document}