\documentclass{article}[18pt]
\usepackage[utf8]{inputenc}
\usepackage[margin=0.7in]{geometry}
\usepackage{parselines} 
\usepackage{amsmath}
\usepackage{titlesec}
\usepackage{pgfplots}
\usepackage{graphicx}
\usepackage[english]{babel}
\usepackage{fancyhdr}

\pgfplotsset{width=10cm,compat=1.9}

\titlespacing\section{0pt}{14pt plus 4pt minus 2pt}{0pt plus 2pt minus 2pt}
\newlength\tindent
\setlength{\tindent}{\parindent}
\setlength{\parindent}{0pt}
\renewcommand{\indent}{\hspace*{\tindent}}

\pagestyle{fancy}
\fancyhf{}
\rhead{Sam Robbins 13SE}
\lhead{A Level Physics - Fields}
\rfoot{Page \thepage}


\begin{document}
\begin{center}
\underline{\huge Gravitational Fields Notes}
\end{center}
Gravitational field strength, $g$, is the force felt per unit mass on a unit mass placed at that point in a field.\\
$g=\dfrac{F}{m}$, unit=$NKg^{-1}$\\
The gravitational field is around all objects with mass and can be either \textbf{radial} or \textbf{uniform}.
\section{Introduction to Gravitational potential}
Definition: The potential at a point in a field is the work done in moving unit mass from infinity to that point.\\
\\
$V=\dfrac{\text{Work done}}{\text{Unit mass}}$ Unit: $JKg^{-1}$\\
In a radial field $v=\dfrac{-GM}{r}$\\
\begin{tikzpicture}[scale=0.5]
\begin{axis}[xmin=0,ymax=0,ticks=none,ylabel={v},xlabel={r},axis x line=top,axis y line=left,label style={font=\Huge},]
\addplot[color=black, domain=0:2]{-1/x};
\draw[ultra thin] (axis cs:0,\pgfkeysvalueof{/pgfplots/ymin}) -- (axis cs:0,\pgfkeysvalueof{/pgfplots/ymax});
\draw[ultra thin] (axis cs:\pgfkeysvalueof{/pgfplots/xmin},0) -- (axis cs:\pgfkeysvalueof{/pgfplots/xmax},0);

\end{axis}
\end{tikzpicture}
\\
Gravitational potential energy = $\Delta V\times m$\\
$V_o-V_s=\Delta V$ Unit:$JKg^{-1}$\\
$V_s$ = Potential on earth, $V_o$ = Potential in orbit.
\section{Potential gradient}
Potential gradient - Change in potential per unit change in distance
$$g=-\frac{\Delta V}{\Delta r}$$
\section{Newton's law of gravitation}
The force between two masses is attractive and is directly proportional to the product of the masses and is inversely proportional to the distance between them squared.
$$F=\frac{GM_1M_2}{r^2}$$
\section{Escape Velocity}
$$\frac{1}{2}mv^2=\frac{GMm}{r}$$
$$\frac{1}{2}v^2=\frac{GM}{r}$$
$$v^2=\frac{2GM}{r}$$
$$v=\sqrt{\frac{2GM}{r}}$$
\section{Satellite motion}
The force between a planet of mass M and the satellite of mass m is described by Newton's law of gravitation.\\
This force provides the \textbf{centripetal force} acting towards the centre of the orbit.
$$\frac{GMm}{r^2}=\frac{mv^2}{r}$$
$$GM=v^2r$$
$$\sqrt{\frac{GM}{r}}=v$$
$$v\propto r^{-\frac{1}{2}}$$
\section{Kepler's 3rd law}
$$F=\frac{GMm}{r^2}=\frac{mv^2}{r}=m\omega^2r$$
$$\frac{GMm}{r^2}=m\omega^2r$$
$$\frac{GM}{r^3}=\Big(\frac{2\pi}{T}\Big)^2$$
$$\frac{GM}{r^3}=\frac{4\pi^2}{T^2}$$
$$\frac{r^3}{T^2}=\frac{GM}{4\pi^2}$$
$$\textrm{As}\  r\uparrow \ T\downarrow$$
\section{Geostationary orbits}
Sometimes called geosynchronous orbits\\
$T=24h=86400s$\\
$r=42Mm=35857km$ above the earth\\
Geostationary - Orbiting in plane of the equator\\
Geosynchronous - 24h orbit inclined at an angle to the equator\\
\\
Geostationary orbits:
\begin{itemize}
\item Period of 24h
\item 36,000 km above the earth's surface
\item Circular
\item Equatorial (in plane of the equator)
\item Are in the same direction as the earth's rotation
\end{itemize}
\section{The time period of satellites}
$$\sqrt{\frac{GM}{r}}=\frac{2\pi r}{T}$$
$$\frac{GM}{r}=\frac{4\pi^2r^2}{T^2}$$
$$T^2=(\frac{4\pi^2}{GM})r^3$$
$$T^2\propto r^3$$
$$T=kr^{\frac{3}{2}}$$
$\log T=\frac{3}{2}\log r+\log k$ is in the form $y=mx+c$
This can also be used to fine the height that a satellite must be at to have any given orbit time.
















\end{document}