\documentclass{article}[18pt]
\usepackage[utf8]{inputenc}
\usepackage[margin=0.7in]{geometry}
\usepackage{parselines} 
\usepackage{amsmath}
\usepackage{titlesec}
\usepackage{pgfplots}
\pgfplotsset{width=10cm,compat=1.9}
\usepackage[english]{babel}
\usepackage{fancyhdr}

\titlespacing\section{0pt}{14pt plus 4pt minus 2pt}{0pt plus 2pt minus 2pt}
\newlength\tindent
\setlength{\tindent}{\parindent}
\setlength{\parindent}{0pt}
\renewcommand{\indent}{\hspace*{\tindent}}

\pagestyle{fancy}
\fancyhf{}
\rhead{Sam Robbins 13SE}
\lhead{A Level Physics - Fields}
\rfoot{Page \thepage}


\begin{document}
\begin{center}
\underline{\huge Physics Capacitor Notes}
\end{center}

\begin{obeylines}
\section{Introduction to Capacitors}
A capacitor is a device used to store charge, constructed of two parallel metal gates near each other with an insulator in between. One conductor gains electrons and the other one loses them. 

The charge on a capacitor with constant current is $Q=It$. Voltage across a capacitor is directly proportional to the current stored.

The unit of capacitance is the Farad, equal to one coulomb per volt. It will usually be given in $\mu F$, $nF$ or $pF$.
$ $
$C=\dfrac{Q}{V}$

\section{Uses of Capacitors}
\begin{itemize}
\item Smoothing circuits
\item Back-up power supplies
\item Timing circuits
\item Pulse producing circuits
\item Tuning circuits
\item Filter circuits (remove unwanted frequencies)
\end{itemize}

\section{Capacitor discharge through a fixed resistor}
As a capacitor discharges through a fixed resistor the current reduces to zero. This is because as charge reduces, voltage reduces. Because $Current = \dfrac{pd}{resistance}$ and resistance is constant current decreases as voltage decreases. Current and voltage decrease exponentially over time, giving their graph with time a curve. If the initial charge is Q, and after t the charge is 0.8 Q then after nt the charge will be $0.8^nQ$. 
This is because
$I=\dfrac{V}{R}=\dfrac{Q}{CR}$
$ $
$\Delta Q=-I \Delta t$
$ $
$\dfrac{dQ}{dt}=-\dfrac{Q}{CR}$
$ $
$\int\dfrac{dQ}{Q}=-\int\dfrac{1}{CR}dt$
$lnQ=-\dfrac{t}{CR}+A$
$ $
$Q=e^{-\dfrac{t}{RC}+A}=e^{-\dfrac{t}{RC}}e^A=Qe^{-\dfrac{t}{RC}}$

Time Constant = RC, unit=seconds
\newpage
\section{Charging a capacitor through a fixed resistor}
As the capacitor charge increases the charging current decreases. When the capacitor is fully charged, its PD=Source PD and the current is 0. The time constant for the circuit is the time it takes for the capacitor to reach 63\% charge.
Source PD = Resistor PD + Capacitor PD
$V_0=IR+\dfrac{Q}{C}$ at any time
Initial current $(I_0)=\dfrac{V_0}{R}$ assuming the capacitor is initially uncharged.
At time $t$ after the charging starts $I=I_0e^{\dfrac{-t}{RC}}$
$V_0=V_0e^{\dfrac{-t}{RC}}+\dfrac{Q}{C}$
$\dfrac{Q}{C}=V_0(1-e^{\dfrac{-t}{RC}})$
$V=\dfrac{Q}{C}$
$Q=CV_0(1-e^{\dfrac{-t}{RC}})$
At time t=0:
$e^{\dfrac{-t}{RC}}=1$ so Q=0 and V=0
As time $t \rightarrow \infty$:
$e^{\dfrac{-t}{RC}}=0$
$Q \rightarrow Q_0$
$V \rightarrow V_0$

\section{Dielectric action}	
The charge stored on the plates on a capacitor can be increased by inserting a \textbf{Dielectric} between the plates. These are electrically insulating materials that increase the charge storing ability. Polythene and waxed paper are examples of this.

When a dielectric is placed in a capacitor and connected to a battery the molecules become polarised, meaning that electrons are pulled slightly towards the positive plate. 

This causes more charge to be stored on the plates as:
\begin{itemize}
\item The positive side of the dielectric attracts more electrons from the battery onto the negative plate
\item The negative side of the dielectric pushed electrons back to the battery from the positive plate
\end{itemize}
\section{Relative permittivity}
Relative permittivity is the ratio of charge stored without a dielectric, compared to that with a dielectric.
Relative permittivitty $\epsilon _r=\dfrac{Q}{Q_0}$
Q = Charge stored when the area between the plates is completely filled with dielectric substances.
$Q_0$ = Charge without any dielectric.

For a fixed voltage $\epsilon _r = \dfrac{C}{C_0}$

The relative permittivity is also called the dielectric constant.

\section{Capacitor design}
Capacitance = $\dfrac{A \epsilon _0 \epsilon _r}{d}$
A= Surface area
D = Distance between plates
A large capacitance can be achieved by:
\begin{itemize}
\item Making the area as large as possible
\item Making the plate spacing as small as possible
\item Filling the space between the plates with a dielectric which has a relative permittivity as large as possible.
\end{itemize}
\newpage
\section{Polarisation mechanisms}
\subsection{Orientation polarisation}
For molecules where covalent bonds are formed between atoms of different elements. This forms a dipole as the electrons are shared unequally. When an electric field is applied the atoms are displaced in opposite directions, to align with the field.
\subsection{Ionic polarisation}
For substances where ions are held together by ionic bonds. Opposite direction displacement.
\subsection{Electronic polarisation}
Electrons in each atom are displaced relative to the nucleus when an electric field is applied. The electron distribution and the nucleus form a dipole.

\end{obeylines}












\end{document}