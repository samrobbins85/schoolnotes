\documentclass{article}[18pt]
\usepackage[utf8]{inputenc}
\usepackage[margin=0.7in]{geometry}
\usepackage{parselines} 
\usepackage{amsmath}
\usepackage{titlesec}
\usepackage{pgfplots}
\usepackage{graphicx}
\usepackage[english]{babel}
\usepackage{fancyhdr}
\usepackage{circuitikz}
\pgfplotsset{width=10cm,compat=1.9}
\usepackage{relsize}
\titlespacing\section{0pt}{14pt plus 4pt minus 2pt}{0pt plus 2pt minus 2pt}
\newlength\tindent
\setlength{\tindent}{\parindent}
\setlength{\parindent}{0pt}
\renewcommand{\indent}{\hspace*{\tindent}}

\pagestyle{fancy}
\fancyhf{}
\rhead{Sam Robbins 13SE}
\lhead{A Level Physics - Alternating Current}
\rfoot{Page \thepage}


\begin{document}
\begin{center}
\underline{\huge Alternating Current}
\end{center}
\section{Oscilloscopes}
$x$-axis - Time base\\
$y$-axis - Y Sensitivity\\
\\
The advantage of calculating peak to peak voltage over peak voltage is that it reduces the uncertainty in calculating the peak voltage.

\section{Alternating and direct current}
AC - Alternating Current - Current continuously changes direction\\
DC - Direct Current - Current flows in one(same) direction\\
\\
Frequency - The number of waves passing a point per second (Hz)\\
\section{Root mean square}
Root mean square values are used to calculate average values for alternating current that are not zero\\
\\
Peak voltage = $V_0 \quad$ Peak current=$I_0$\\
$$V_{RMS}=\dfrac{V_0}{\sqrt{2}}$$
$$I_{RMS}=\dfrac{I_0}{\sqrt{2}}$$\\
\\
Power, $P=IV$\\
\\
$P_{RMS}=I_{RMS}V_{RMS}=\dfrac{I_0}{\sqrt{2}}\times\dfrac{V_0}{\sqrt{2}}=\dfrac{P_0}{2}$
 
\end{document}
