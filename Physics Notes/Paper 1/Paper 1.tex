\documentclass[12pt]{article}
\usepackage{../../format}
\lhead{A Level Physics}

\begin{document}
\begin{center}
\underline{\huge Paper 1 Cheat Sheet}
\end{center}
\section{Measurements and their errors}
\textbf{Precision} - There is very little spread around the mean value\\
\textbf{Repeatability} - If the same experimenter repeats the investigation using the same method and equipment and obtains the same results\\
\textbf{Reproducibility} - If a different experimenter repeats the investigation, or uses a different experiment or technique, the same results are obtained\\
\textbf{Accuracy} - Close to the true value\\
{\renewcommand{\arraystretch}{2}
\begin{tabularx}{\textwidth}{|X|X|}
\hline
Combination&Operation\\
\hline
Adding or subtracting\newline $a=b+c$&Add the absolute uncertainties\newline $\Delta a=\Delta b+\Delta c$\\
\hline
Multiplying values\newline $a=b\times c$&Add the percentage uncertainties\newline $\epsilon a=\epsilon b+\epsilon c$\\
\hline
Dividing values\newline $a=\dfrac{b}{c}$&Add the percentage uncertainties\newline $\epsilon a=\epsilon b+\epsilon c$\\
\hline
Power rules\newline $a=b^c$&Multiply the percentage uncertainty by the power\newline $\epsilon a=c\times\epsilon b$\\
\hline
\end{tabularx}}
\newpage
\section{Particles and radiation}
\subsection{Constituents of the atom}
Protons and neurons in the centre, with shells of electrons around them
$$\textrm{Specific charge}=\frac{Q}{m}$$
\textbf{Isotope} - An atom with the same number of protons and electrons as an element, but a different number of neutrons
\subsection{Stable and unstable nuclei}
\subsubsection{The strong nuclear force}
\begin{tabular}{|c|c|}
\hline
$<0.5fm$&Repulsion\\
\hline
$0.5-3fm$&Attraction\\
\hline
$3fm+$&No force\\
\hline
\end{tabular}
\subsubsection{Alpha decay}
{\large
$$^A_ZX\rightarrow ^{A-4}_{Z-2}Y+^4_2\alpha$$
\subsubsection{Beta decay}
$$^A_ZX\rightarrow ^A_{Z+1}+^0_{-1}\beta+\overline{\nu}$$}
Neutrinos were hypothesised to allow for energy to be conserved in the interaction
\subsection{Particles, antiparticles and photons}



\end{document}