\documentclass[12pt]{article}
\usepackage[utf8]{inputenc}
\usepackage[margin=0.7in]{geometry}
\usepackage{titlesec}
\usepackage{graphicx}
\usepackage[english]{babel}
\usepackage{fancyhdr}
\usepackage{blindtext}
\usepackage{textcomp}
\pagestyle{fancy}
\fancyhf{}
\rhead{Sam Robbins 13SE}
\lhead{A Level Physics - Turning Points}
\begin{document}
\begin{center}
\underline{\huge The Discovery of photoelectricity}
\end{center}
\section{The ultraviolet catastrophe and black body radiation}
A black body is a theoretical object which is a perfect absorber and emitter of radiation, it would not reflect any radiation, nor allow it to be transmitted through.\\
The ultraviolet catastrophe is the situation where short wavelength radiation is incident on a black body, causing it to emit an infinite amount of power per unit wavelength.
\section{Planck's interpretation in terms of quanta}
Max Planck suggested that the energy from electromagnetic waves was quantised, rather than continuous, which solves the problem of the ultraviolet catastrophe. He formulated the equation:
$$E=hf$$
Where h is Planck's constant\\
\\
This theory was later used to explain photoelectricity.
\section{The failure of classical wave theory to explain observations on photoelectricity}
Predictions by Huygens' Wave theory:
\begin{itemize}
\item For a particular frequency, the energy of the wave depends on its amplitude and is spread evenly over a wave front. This would mean that each free electron in the metal surface would get a small amount of energy from each wavelength, so after a period of time would gain sufficient energy to escape.
\item There would be a time lag between the light being incident and the emission of an electron
\item Any frequency would be able to cause electron emission given enough time
\end{itemize}
These predictions are all wrong.
\section{Einstein's Explanation of photoelectricity}
Einstein proposed a particle theory of light, where electromagnetic waves are emitted in discrete amounts of energy of size $hf$. Einstein called a quantum of light a photon, and visualised the photoelectric effect as an interaction between one photon and one electron in the metal surface, with the photon giving all its energy to the electron, causing it to disappear.\\
He suggested that in order for an electron to leave the metal surface it must have enough energy to overcome the attractive electrical force holding it there, this energy is the work function, with symbol $\phi$. \\
If 
\begin{itemize}
\item $hf>\phi$ then the electron can escape from the surface
\item $hf<\phi$ then the electron gains energy, but does not escape the surface
\item $hf=\phi$ then the electron has enough energy to escape the metal but has zero kinetic energy once it has escaped. This frequency is called the threshold frequency, with symbol $f_0$
\end{itemize}
The kinetic energy the electron has once it has left the metal varies from zero to a maximum value as electrons have to do different amounts of work depending on how far they have to move through the metal.\\
He created the equation:
$$\frac{1}{2}mv^2=hf-\phi$$
\end{document}