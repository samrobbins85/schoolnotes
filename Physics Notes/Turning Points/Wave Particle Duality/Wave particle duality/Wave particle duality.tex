\documentclass[12pt]{article}
\usepackage[utf8]{inputenc}
\usepackage[margin=0.7in]{geometry}
\usepackage{titlesec}
\usepackage{graphicx}
\usepackage[english]{babel}
\usepackage{fancyhdr}
\usepackage{blindtext}
\usepackage{textcomp}
\pagestyle{fancy}
\fancyhf{}
\rhead{Sam Robbins 13SE}
\lhead{A Level Physics - Turning Points}
\begin{document}
\begin{center}
\underline{\huge Wave particle duality}
\end{center}
\section{de Broglie's hypothesis}
de Broglie created the equation for the energy of a photon:
$$E=\frac{hc}{\lambda}$$
And also deduced that photon momentum is:
$$p=\frac{h}{\lambda}$$
This caused de Broglie to put forward the idea that all particles had wave like properties.\\
The de Broglie wavelength is calculated using the equation
$$\lambda=\frac{h}{p}=\frac{h}{mv}$$
\section{Electron diffraction}
Calculation to find the wavelength of electrons from an electron gun:\\
\\
Set kinetic energy equal to eV
$$\frac{1}{2}mv^2=eV$$
Multiply both sides by m and rearrange
$$m^2v^2=2meV$$
Squarerooting
$$mv=\sqrt{2meV}$$
Combining with de Broglie's equation $\lambda=\frac{h}{mv}$
$$\lambda=\frac{h}{\sqrt{2meV}}$$
\\
\\
When electrons are diffracted they show a very similar pattern to the pattern when photons are diffracted, evidence for the wavelike nature of electrons.

\end{document}