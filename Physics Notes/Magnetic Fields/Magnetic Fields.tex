\documentclass{article}[18pt]
\usepackage[utf8]{inputenc}
\usepackage[margin=0.7in]{geometry}
\usepackage{parselines} 
\usepackage{amsmath}
\usepackage{titlesec}
\usepackage{pgfplots}
\usepackage{graphicx}
\usepackage[english]{babel}
\usepackage{fancyhdr}

\pgfplotsset{width=10cm,compat=1.9}

\titlespacing\section{0pt}{14pt plus 4pt minus 2pt}{0pt plus 2pt minus 2pt}
\newlength\tindent
\setlength{\tindent}{\parindent}
\setlength{\parindent}{0pt}
\renewcommand{\indent}{\hspace*{\tindent}}

\pagestyle{fancy}
\fancyhf{}
\rhead{Sam Robbins 13SE}
\lhead{A Level Physics - Fields}
\rfoot{Page \thepage}


\begin{document}
\begin{center}
\underline{\huge Magnetic Fields}
\end{center}
\section{Charged Particles in magnetic fields}
Force is always perpendicular to the direction - no change in speed to the electrons.\\
\\
If electrons are confined to a wire $F=BIl$\\
\\
The same equation applies for many charges where:\\
\\
$I=\dfrac{Q}{t} \quad l=vt$\\
\\
$F=B\times\frac{Q}{t}\times vt$\\
\\
$F=BQv$\\
\\
\subsection{The hall probe}
This is used to measure the strength of magnetic fields. It consists of a thin slice of semiconductor material with a constant current flowing through it.\\
\\
The negative charges go to the bottom of the material and so the top is positively charged, this creates a potential difference.\\
\\
$F_B=F_E$\\
\\
$BQV=\dfrac{VQ}{d}$\\
\\
$V=Bvd$\\
\\
$V\propto B$\\
\\
By measuring the voltage across the probe the magnetic field strength can be measured.




\end{document}