\documentclass{article}[18pt]
\usepackage{/home/sam/Documents/School_Notes/format}
\lhead{AS Level Physics - Particles and Radiation}

\begin{document}
\begin{center}
\underline{\huge Particles}
\end{center}
\section{Constituents of the atom}
\textbf{Specific charge} - The charge to mass ratio\\
\textbf{Isotope} - An atom with the same number of protons as an element but a different number of neutrons
\section{Stable and unstable nuclei}
\subsection{The strong nuclear force}
$0fm-0.5fm$ - \textbf{Repulsion}\\
$0.5fm-3fm$ - \textbf{Attraction}\\
$3fm+$ - \textbf{No force}
\subsection{Alpha decay}
An atom emits an alpha particle (Helium Nucleus/2 protons and 2 neutrons)\\
Reduces \textit{Mass Number} by \textbf{4} and \textit{proton number} by \textbf{2}
\subsection{Beta decay}
Neutron$\rightarrow$Proton+Electron+Neutrino\\
The neutrino was hypothesised to conserve energy
\section{Particles, antiparticles and photons}
For every particle there is a corresponding antiparticle (can be itself)\\
\\
{\def\arraystretch{1.5}
\begin{tabularx}{\textwidth}{|X|X|X|}
\hline
\textbf{Property}&\textbf{Particle}&\textbf{Antiparticle}\\
\hline
Mass&x&x\\
\hline
\textcolor{red}{Charge}&x&-x\\
\hline
Rest Energy&x&x\\
\hline
\textcolor{red}{Baryon Number}&x&-x\\
\hline
\textcolor{red}{Lepton Number}&x&-x\\
\hline
\textcolor{red}{Strangeness}&x&-x\\
\hline
\end{tabularx}}
\\
\\
When a particle and antiparticle collide they annihilate each other\\
A particle-antiparticle pair can be produced from energy
\section{Particle interactions}
Fundamental interactions:\\
{\def\arraystretch{1.5}

\begin{tabularx}{\textwidth}{|X|X|X|X|}
\hline
\textbf{Force}&\textbf{Affects}&\textbf{Gauge Boson}&\textbf{Range}\\
\hline
Gravitational&Mass&Graviton&Infinite\\
\hline
Electromagnetic&Charge&Photon&Infinite\\
\hline
Nuclear Strong&Quarks&Gluon(Pion)&$10^{-15}$m\\
\hline
Nuclear Weak&Leptons+Quarks&$W^+,W^-,Z^0$&$10^{-18}$m\\
\hline
\end{tabularx}}\\
\\
\newpage
The exchange particles provide the forces between elementary particles\\
Virtual photons are the exchange particle of the electromagnetic force\\
Examples of the weak interaction:
\begin{itemize}
\item $\beta^+$ Decay
\item $\beta^-$ Decay
\item Electron capture
\item Electron-proton collisions
\end{itemize}
\section{Classification of particles}
\subsection{Hadrons}
Hadrons are subject to the strong interaction\\
There are two types of hadrons:
\begin{itemize}
\item Baryon(3 Quarks)
\item Meson (Quark-Antiquark pair)
\end{itemize}
The baryon number is conserved in an interaction\\
All baryons will eventually decay into protons\\
Kaons can decay into pions\\
\subsection{Leptons}
Types of lepton(All have a lepton number of 1):
\begin{itemize}
\item Electron
\item Muon
\item Electron Neutrino (Approximated to massless)
\item Muon Neutrino (Approximated to massless)
\end{itemize}
Lepton number is conserved during an interaction\\
Muons decay into electrons\\
\subsection{Strange particles}
Produced through the strong interaction\\
Decay through the weak interaction
\newpage
\section{Quarks and antiquarks}
\subsection{Baryons}
{\def\arraystretch{1.2}
\begin{tabularx}{\textwidth}{|X|X|}
\hline
Proton&UUD\\
\hline
Neutron&DUD\\
\hline
\end{tabularx}}
\subsection{Mesons}
\subsubsection{Pions(All 0 Strangeness)}
{\def\arraystretch{1.5}
\begin{tabularx}{\textwidth}{|X|X|}
\hline
$\pi^0$&$U\bar{U}$ or $D\bar{D}$\\
\hline
$\pi^+$&$U\bar{D}$\\
\hline
$\pi^-$&$D\bar{U}$\\
\hline
\end{tabularx}}
\subsubsection{Kaons (All strange)}
{\def\arraystretch{1.5}
\begin{tabularx}{\textwidth}{|X|X|}
\hline
$K^+$&$U\bar{S}$\\
\hline
$K^-$&$\bar{U}S$\\
\hline
$K^0$&$D\bar{S}$\\
\hline
$\bar{K^0}$&$\bar{D}S$\\
\hline
\end{tabularx}}

\end{document}
