\documentclass{article}[18pt]
\usepackage{/home/sam/Documents/School_Notes/format}
\lhead{AS Level Physics - Particles and Radiation}

\begin{document}
\begin{center}
\underline{\huge Electromagnetic radiation and quantum phenomena}
\end{center}
\section{The photoelectric effect}
\textbf{The photoelectric effect - Metals liberate electrons when hit by photons}\\
\begin{tabularx}{\textwidth}{|X|X|}
\hline
\textbf{Observation}&\textbf{Explanation}\\
\hline
There is a minimum frequency required to liberate an electron, called the threshold frequency&At the minimum frequency KE=0 so the frequency must be Planck's Constant$\times$Frequency\\
\hline
Above the threshold frequency electrons are emitted immediately, regardless of intensity&Energy is immediately passed on\\
\hline
Intensity and electrons per second are directly proportional&1 photon liberates 1 electron\\
\hline
Below the threshold frequency no electrons are emitted, no matter the intensity&At the minimum frequency KE=0 so the frequency must be Planck's Constant$\times$Frequency\\
\hline
The electrons are emitted with a range of speeds unrelated to intensity&Through the metal the collisions for each electron vary, varying KE\\
\hline
Increasing frequency increases maximum speed but not number liberated per second&1 Photon liberates 1 electron\\
\hline
\end{tabularx}
\subsection{Why the photoelectric effect can't be explained by the wave model}
Waves transfer energy, it would be expected that the more intense the source the more energy the electrons would have.\\
Given enough time electrons should be emitted with waves
\subsection{Millikan's experiment}
\begin{tikzpicture}
\begin{axis}[axis lines=middle,scale=0.8,ylabel = KE,xlabel=Frequency,ylabel style={rotate=90,anchor=south},xlabel style={anchor=south},xmin=0,xmax=7,ymin=-5,ymax=7,xtick={2},xticklabels={$F_0$},ytick={-2},yticklabels={$-\Phi$}
]
\addplot[color=black,domain=0:7]{x-2};
\end{axis}
\end{tikzpicture}\\
$F_0$ - Threshold Frequency\\
$-\Phi$ - Negative work function\\
\textbf{Gradient} - Planck's Constant\\
Equation of line - $E_K=hf-\phi$ 
\newpage
\section{Collisions of electrons with atoms}
\textbf{Ionisation} - The adding or removal of electrons from an element\\
\textbf{Excitation} - The moving of an electron to a higher energy level\\
\subsection{How a fluorescent lamp works}
\begin{enumerate}
\item In a sealed unit with a phosphor coating on the inside of the glass. A small amount of mercury and an inert gas at a low pressure
\item When the voltage is switched on electrons move from the cathode to the anode
\item The electrons collide with the mercury, turning it into a vapour
\item Once a vapour, the moving electrons can then collide with the mercury gas atoms, exciting the bound electrons
\item The fired electron has undergone collisional excitation so has less energy
\item The mercury excitation produces UV rays
\item The phosphor coating is a chemical designed so it has UV excitation energy similar to the mercury excitation energy
\item The phosphor absorbs photons and emits visible light
\end{enumerate}
\section{Energy levels and photon emission}
\subsection{Absorption Spectra}
This happens when a cool material is placed between a hot light and the observer.\\
This causes dark lines in the spectrum\\
From this it can be inferred that atoms only absorb radiation of certain frequencies.\\
This technique can be used to identify different atoms\\
The difference in absorption is caused by different numbers of electrons\\
\subsection{Emission spectra}
This is caused when a gas is heated\\
The lines have the same wavelengths as absorption but the lines are light, rather than dark\\
This is caused by electrons becoming excited and so emitting an electron\\
\begin{tabularx}{\textwidth}{|X|X|}
\hline
\textbf{Observation}&\textbf{Reason}\\
\hline
Discrete lines&The electrons are absorbing/emitting photons\\
\hline
The spectrum does not depend on the isotope&The electrons are arranged in shells\\
\hline
The spectral lines get closer together as the energy increases&The photons need a certain amount of energy to excite an electron\\
\hline
\end{tabularx}
\subsection{The hydrogen atom}
The hydrogen atom is used as the example for energy level calculations. The ground state is the state of minimum energy, calculations can be performed for all the energy levels
\section{Wave-Particle Duality}
Electron diffraction suggests that particles possess wave properties. The photoelectric effect suggests electromagnetic waves have a particle nature.\\
The amount of diffraction changes when the momentum of the particle changes as it has a different de Broglie wavelength.
\end{document}
