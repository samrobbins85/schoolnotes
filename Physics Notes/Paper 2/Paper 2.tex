\documentclass[12pt]{article}
\usepackage{../../format}
\lhead{A Level Physics}
\usepackage{enumitem}
\setcounter{secnumdepth}{4}
\begin{document}
\begin{center}
\underline{\huge Paper 2 Cheat Sheet}
\end{center}
\section{Thermal physics}
\subsection{Thermal energy transfer}
\textbf{Specific heat capacity} - The energy required to raise the temperature of a unit mass of a given substance by one degree\\
\textbf{Specific latent heat}  - The energy required to change the state of a material without changing the temperature\\
\textbf{Temperature} - The average kinetic energy of the atoms or molecules in the system\\
\textbf{Heat} - Energy transfer due to a difference in temperature
\subsubsection{Continuous flow}
By dividing the specific heat capacity formula by t it can be found that
$$IV=mc\frac{\Delta T}{t}$$
This gives the power per second, where a mass $m$ flows in a time $t$
\subsection{Ideal gases}
{\renewcommand{\arraystretch}{2}
\begin{tabularx}{\textwidth}{|X|X|X|X|}
\hline
Law&Proportionality&Constant&Equation\\
\hline
Boyle's&$p\propto\dfrac{1}{v}$&Temperature, moles&$p_1v_1=p_2v_2$\\
\hline
Charles'&$V\propto T$&Pressure, moles&$\dfrac{v_1}{T_1}=\dfrac{v_2}{T_2}$\\
\hline
Gay-Lussac&$p\propto T$&Volume, moles&$\dfrac{p_1}{T_1}=\dfrac{p_2}{T_2}$\\
\hline
\end{tabularx}}
The two formulas on the formula book for the gas laws have $n$ moles and $N$ molecules
\subsubsection{Deriving Pressure volume work formula}
$$W=FS=F\times\Delta L=\frac{F}{A}\times A\Delta L=P\Delta V$$
\subsubsection{Types of masses}
\textbf{Molar mass} - The mass of a mole of a substance\\
\textbf{Molecular mass} - The mass of the molecules
\newpage
\subsection{Molecular kinetic theory model}
\subsubsection{Brownian motion as evidence for the existence of atoms}
\textbf{Brownian motion} - The random motion of smoke particles in a gas\\
As Newton's first law states that objects remain in motion until acted on by a force, the smoke particles should remain in motion, instead they move randomly, suggesting collisions with something else
\subsubsection{Explanation of relationships between p,V and T}
Increase pressure - More collisions, increase temperature. Same number of molecules, volume must decrease
\subsubsection{Empirical gas laws but theoretical kinetic theory}
By changing variables of a gas, the gas laws can be derived, however the kinetic theory is based on what else would be expected to be required to be constant.
\subsubsection{Derivation}
\textcolor{red}{Newton's $3^{\text{rd}}$ law - Every action has an equal and opposite reaction}\\
$\Delta mc=mc_{x1}--mc_{x1}=2mc_{x1}$\\
\\
\textcolor{red}{Use $\text{Velocity}=\dfrac{\text{Distance}}{\text{Time}}$}\\
Time=$\dfrac{\text{Distance}}{\text{Velocity}}=\dfrac{2l}{c_{x1}}$\\
\\
\textcolor{red}{Use force$=\dfrac{\text{Change in momentum}}{\text{time}}$}\\
$\text{Force}=\dfrac{\Delta mc}{\Delta t}=\dfrac{2mc{x1}}{2l/c_{x1}}=\dfrac{mc_{x1}^2}{l}$\\
\\
\textcolor{red}{Use Pressure$=\dfrac{\text{Force}}{\text{Area}}$}\\
\\
$p_1=\dfrac{mc_{x1}^2/l}{l^2}=\dfrac{mc_{x1}^2}{l^3}$\\
\\
\textcolor{red}{Expand for N particles}\\
$p=\Sigma p_n=p_1+p_2+p_3...+p_N$\\
\\
$p=\dfrac{mc_{x1}^2}{l^3}+\dfrac{mc_{x2}^2}{l^3}+\dfrac{mc_{x3}^2}{l^3}+\dfrac{mc_{xN}^2}{l^3}$\\
\\
$p=\dfrac{m}{l^3}(c_{x1}^2+c_{x2}^2+c_{x3}^2...+c_{xN}^2)$\\
\\
\textcolor{red}{The mean of all the squares of the velocities is written as $\mathlarger{\bar{c_x^2}}$}\\
\\
$\mathlarger{\bar{c_x^2}}=\dfrac{c_{x1}^2+c_{x2}^2+c_{x3}^2...+c_{xN}^2}{N}$\\
\\
$N\mathlarger{\bar{c_x^2}}=c_{x1}^2+c_{x2}^2+c_{x3}^2...+c_{xN}^2$\\
\\
\\
\textcolor{red}{Simplify expression for pressure}\\
$p=\dfrac{Nm\mathlarger{\bar{c_x^2}}}{l^3}$\\
\\
\textcolor{red}{Consider in 3 dimensions}\\
\\
$\mathlarger{\bar{c^2}=\bar{c_x^2}+\bar{c_y^2}+\bar{c_z^2}}$\\
\textcolor{red}{Average of mean square velocity for each dimension are equal}\\
\\
$\mathlarger{\bar{c_x^2}=\bar{c_y^2}=\bar{c_z^2}}$\\
\textcolor{red}{Simplify 3D formula}\\
\\
$\mathlarger{\dfrac{\bar{c^2}}{3}=\bar{c_x^2}=\bar{c_y^2}=\bar{c_z^2}}$\\
\\
\textcolor{red}{Simplify pressure formula}\\
\\
$p=\dfrac{1}{3}\times\dfrac{Nm\mathlarger{\bar{c^2}}}{l^3}$\\
\\
\textcolor{red}{Insert Density formula}\\
\\
$\rho=\dfrac{\text{Mass}}{\text{Volume}}=\dfrac{Nm}{l^3}$
\\
\textcolor{red}{Substitute into Pressure formula}\\
$$p=\frac{1}{3}\rho\mathlarger{\bar{c^2}}$$
\section{Fields and their consequences}
\subsection{Fields}
Similarities and differences between gravitational and electrostatic forces\\
\setitemize{noitemsep,topsep=-5pt,leftmargin=*}%Compress list
\begin{tabularx}{\textwidth}{|X|X|}
\hline
Similarities&Differences\\
\hline
\begin{itemize}
\item Inverse square laws
\item Use of field lines
\item Use of potential
\item Use of equipotentials
\end{itemize}& Masses always attract, but charges may attract or repel\\
\hline
\end{tabularx}
\newpage
\subsection{Gravitational fields}
\subsubsection{Gravitational field strength}
\includegraphics[width=8cm]{gravity.jpg}
\subsubsection{Gravitational potential}
Gravitational potential has a value of 0 at infinity, then reduces as it approaches the planet.\\
\textbf{Gravitational potential} - The work done in moving a unit mass from infinity to that point int he field\\
\textbf{Gravitational potential difference} - The work done in moving a unit mass from one point to another \\
\textbf{Equipotential} - The group of points with the same potential energy\\
\\
The sign is negative because a negative amount of work has to be done to move the object from infinity to earth because the object is attracted to earth.
\subsubsection{Orbits of planets and satellites}
\paragraph{Kepler's law}
$$F=\frac{GMm}{r^2}=\frac{mv^2}{r}\quad \therefore \frac{GM}{r}=v^2$$
$$v=\frac{s}{t}=\frac{2\pi r}{T}$$
$$v^2=\frac{4\pi^2r^2}{T^2}=\frac{GM}{r}$$
$$\frac{r^2}{T^2}=\frac{GM}{4\pi^2}$$
RHS is a constant
$$r^2\propto T^2$$
\paragraph{Escape velocity}
$$\frac{1}{2}mv^2=\frac{GMm}{r}\quad \therefore v=\sqrt{\frac{2GM}{r}}$$
$$g=\frac{GM}{r^2}\quad \therefore gr=\frac{GM}{r}$$
$$v=\sqrt{2gr}$$
\paragraph{Total energy of an orbiting satellite}
$$\textrm{Total energy=KE+GPE}$$
$$KE=\frac{1}{2}mv^2$$


\end{document}